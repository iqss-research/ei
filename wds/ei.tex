%%
%% EDIT in unix format only.  Don't put DOS characters in here,
%% or latex2html won't work.
%%
%% _sometimes_ this works to fix it:  perl -i.bak -e 's/\r//' -p t.tex
%%
\documentclass[11pt,titlepage]{article}
\usepackage[reqno]{amsmath}
\usepackage{amssymb}
\usepackage{verbatim}
\usepackage{epsf}
\usepackage{url}
\usepackage{html}
%\pagestyle{myheadings}
\htmladdtonavigation{
  \htmladdnormallink{%
    \htmladdimg{http://gking.harvard.edu/pics/home.gif}}
  {http://gking.harvard.edu/}}

\bodytext{ BACKGROUND="http://gking.harvard.edu/pics/temple.jpg"}
\setcounter{tocdepth}{3}
\renewcommand{\bibitem}{\vskip 2pt\par\hangindent\parindent\hskip-\parindent}

                                % dcolumn
\usepackage{dcolumn}
\newcolumntype{.}{D{.}{.}{-1}}
\newcolumntype{d}[1]{D{.}{.}{#1}}

\newcommand{\EI}{\ensuremath{{\mathfrak EI}}}
\newcommand{\EzI}{\ensuremath{{\mathfrak EzI}}}

\title{${\mathfrak EI}$: A Program for Ecological Inference}
\author{Gary King\\ Department of Government\\ Harvard
  University\thanks{\copyright\ Copyright 1995--2001, All rights
    reserved.  (Center for Basic Research in the Social Sciences, 34
    Kirkland Street, Harvard University, Cambridge MA
    02138\texttt{King@Harvard.Edu}, phone: (617) 495-2027.)  This
    document can be read on-line in hypertext format from
    \texttt{http://GKing.Harvard.Edu}. The program implements the
    procedures described in \emph{A Solution to the Ecological
      Inference Problem: Reconstructing Individual Behavior from
      Aggregate Data} (Princeton University Press, 1997).  ${\mathfrak
      EI}$ was written in the Gauss Programming Language, \copyright
    Copyright Aptech Systems, Inc., and uses Gauss' Constrained
    Maximum Likelihood Module written by Ronald J.\ Schoenberg.  For
    making available public domain Gauss code, I am grateful to David
    Baird (for an inverse cumulative normal procedure), Martin van der
    Ende (for an accurate cumulative bivariate normal procedure), and
    Simon Jackman (for a loess procedure).  For research assistance,
    my thanks goes to Kosuke Imai and Eric Dickson, and for research
    support, my thanks goes to the National Science Foundation
    (IIS-9874747), the National Institutes of Aging (P01 AG17625-01),
    and the World Health Organization.  EI is copyrighted, but you may
    copy and distribute this program provided that no charge is made
    and the copy is identical to the original.  To request an
    exception, please contact me.}}

\date{Version 1.7, 5 August 2001}
\newcommand{\Tiny}{\tiny}
\newcommand{\sump}{\sum_{i=1}^p}
\newcommand{\mean}{\frac{1}{p}\sump}
\newcommand{\ub}{\Dot{\beta}}
\newcommand{\ut}{\Dot{\theta}}
\newcommand{\bbeta}{{\mathfrak B}}
\newcommand{\btheta}{{\mathfrak T}}
\newcommand{\blambda}{{\mathfrak L}}
\newcommand{\bbetau}{\breve{\mathfrak B}}
\newcommand{\bV}{{\cal V}}
\newcommand{\sigmau}{\breve{\sigma}}
\newcommand{\Sigmau}{\breve{\Sigma}}
\newcommand{\rhou}{\breve{\rho}}
\newcommand{\psiu}{\breve{\psi}}
\newcommand{\Eu}{\breve{\text{E}}}
\newcommand{\Vu}{\breve{\text{V}}}
\newcommand{\Bb}{B^b}
\newcommand{\Bw}{B^w}
\newcommand{\NbD}{{N_i^{bD}}}
\newcommand{\NwD}{{N_i^{wD}}}
\newcommand{\NbR}{{N_i^{bR}}}
\newcommand{\NwR}{{N_i^{wR}}}
\newcommand{\NbN}{{N_i^{bN}}}
\newcommand{\NwN}{{N_i^{wN}}}
\newcommand{\tp}{{\mbox{\Tiny{+}}}}
\newcommand{\NpD}{{N_i^D}}
\newcommand{\NpR}{{N_i^{R}}}
\newcommand{\NpN}{{N_i^{N}}}
\newcommand{\Nbp}{{N_i^{b}}}
\newcommand{\Nwp}{{N_i^{w}}}
\newcommand{\Npp}{{N_i}}
\newcommand{\Nppp}{{N}}
\newcommand{\Nbpp}{{N^{b}}}
\newcommand{\Nwpp}{{N^{w}}}
\newcommand{\sumpN}{\sump\Npp}
\newcommand{\NsumpN}{\frac{1}{\Nppp}\sumpN}
\newcommand{\nbp}{{N_i^{bT}}}       % redo old definitions
\newcommand{\nwp}{{N_i^{wT}}}
\newcommand{\npp}{{N_i^{T}}}
\newcommand{\NbV}{{N_i^{bT}}}       % duplicate definitions with new name
\newcommand{\NwV}{{N_i^{wT}}}
\newcommand{\NpV}{{N_i^{T}}}

\newcommand{\xb}{\bar{X}}
\newcommand{\wb}{\bar{W}}
\newcommand{\tb}{\bar{T}}
\newcommand{\cx}{{\mathbb X}}
\newcommand{\cw}{{\mathbb W}}
\newcommand{\ct}{{\mathbb T}}
\newcommand{\cpij}{{\mathbb P}^*_{ij}}
\newcommand{\cpi}{{\mathbb P}^*_i}
\newcommand{\cwd}{{\dot{\mathbb W}}}
\newcommand{\ctd}{{\dot{\mathbb T}}}
\newcommand{\cxd}{{\dot{\mathbb X}}}
\newcommand{\cxb}{\bar{\mathbb X}}
\newcommand{\cz}{{\mathbb G}}
\newcommand{\ch}{{\mathbb H}}
\newcommand{\cm}{{\mathbb M}}
\newcommand{\E}{{\textbf{E}}}
\newcommand{\V}{{\textbf{V}}}
\newcommand{\C}{{\textbf{C}}}
\newcommand{\rE}{{\text{E}}}
\newcommand{\rV}{{\text{V}}}
\newcommand{\rC}{{\text{C}}}
\newcommand{\TN}{\text{TN}}
\newcommand{\BN}{\text{BN}}
\newcommand{\N}{\text{N}}
\renewcommand{\P}{\text{P}}
\newcommand{\D}{\textbf{D}}
\newcommand{\R}{\ensuremath{\textbf{R}}}
\newcommand{\Rr}{\ensuremath{\{\textbf{R}\diagdown r}\}}
\newcommand{\bC}{\ensuremath{\textbf{C}}}
\newcommand{\Cc}{\ensuremath{\{\textbf{C}\diagdown c}\}}
\newcommand{\one}{{\mathbf{1}}}
\newcommand{\Q}{\ensuremath{\overset{\vspace{3em}}{\text{\Huge ?}}}}
\newlength{\padsp}
\settowidth{\padsp}{$(\beta^w_i=\nwp/\Nwp)$}
\newcommand{\padbw}{\hspace*\padsp}
\newcommand{\bkappa}{\boldsymbol{\kappa}}

\newcommand{\hlink}{\htmladdnormallink}
\begin{document}
\maketitle
\tableofcontents

\section{Introduction}

This program provides a method of inferring individual behavior from
aggregate data that works in practice.  It implements the statistical
procedures, diagnostics, and graphics from the book \hlink{\emph{A
    Solution to the Ecological Inference Problem: Reconstructing
    Individual Behavior from Aggregate
    Data}}{http://gking.harvard.edu/preprints.shtml#sei} (Princeton:
Princeton University Press, 1997), by Gary King.  Please read the book
prior to trying this program (a sample chapter and other related
information is available at my \hlink{web
  site}{http://gking.harvard.edu/preprints.shtml#sei}).  Except where
indicated, all references to page, section, chapter, table, and figure
numbers in this document refer to the book.

\emph{Ecological inference}, as traditionally defined, is the process
of using aggregate (i.e., ``ecological'') data to infer discrete
individual-level relationships of interest when individual-level data
are not available.  As existing methods usually lead to inaccurate
conclusions about the empirical world, the ecological inference
\emph{problem} had been to develop a method that gives accurate
answers.  Ecological inferences are required in political science
research when individual-level surveys are unavailable (e.g., local or
comparative electoral politics), unreliable (racial politics),
insufficient (political geography), or infeasible (political history).
They are also required in numerous areas of major significance in
public policy (e.g., for applying the Voting Rights Act) and other
academic disciplines ranging from epidemiology and marketing to
sociology and quantitative history.  Most researchers using aggregate
data have encountered some form of the ecological inference problem.

Because the ecological inference problem is caused by the lack of
individual-level information, no method of ecological inference,
including that introduced in this book and estimated by this program,
will produce precisely accurate results in every instance.  However,
potential difficulties are minimized here by models that include more
available information, diagnostics to evaluate when assumptions need
to be modified, easy methods of modifying the assumptions, and
uncertainty estimates for all quantities of interest.  I recommend
reviewing Chapter 16 while using this program for actual research.

\section{Hardware and Software Requirements}

\EI\ is written in Gauss, and will run on any computer hardware and
operating system that runs Gauss.  The Gauss module CML (constrained
maximum likelihood, by Ronald J.\ Schoenberg) is also required.  No
other special hardware or software is required except the program that
accompanies this documentation.  A menu-oriented, stand-alone version
of this program, by Kenneth Benoit and me, is available at my
homepage, \hlink{\url{http://GKing.Harvard.Edu/stats.shtml}}
{http://gking.harvard.edu/stats.shtml}.  This version, \emph{\EzI: A(n
  easy) Program for Ecological Inference}, does not require Gauss, but
it is somewhat less flexible.

Gauss and CML are available for DOS/Windows, Windows, UNIX, and other
operating systems from Aptech Systems, Inc.; 23804 S.E. Kent-Kangley
Road; Maple Valley, Washington 98038; (206) 432-7855; FAX: (206)
432-7832; email: \texttt{sales@aptech.com}.  If you do not have a lot
of RAM, but you do have a lot of observations, consider turning on
Gauss' virtual memory feature.

\section{Installation}

If you received the program in a zip file, unzip it (with the public
domain program unzip).

\paragraph{On a DOS or Windows-based system:} Copy \texttt{ei.lcg} to
your \texttt{lib} subdirectory, such as
$\backslash$\texttt{GAUSS}$\backslash$\texttt{LIB}, and copy the
remaining files (\texttt{*.src}, \texttt{ei.dec}, \texttt{ei.ext}, and
\texttt{sample.asc}) to somewhere on the GAUSSPATH or src\_path, such
as the default $\backslash$\texttt{GAUSS}$\backslash$\texttt{SRC}.

\paragraph{On a UNIX-based system:}
If you have supervisor privleges, follow analogous steps as for DOS
systems.  If you do not have supervisor privileges, or do not wish the
program to be installed for all users, unzip ei.zip into a directory
and then specify ei in the library command with a path, e.g.:
\texttt{library /files/gking/ecinf/ei;}.

\paragraph{Information on Program Updates}
See the file \hlink{whatsnew}{/ei/whatsnew} for a list of the major
recent and planned changes to \EI, and the web page
\hlink{\url{http://GKing.Harvard.Edu/netmind.shtml}}
{http://GKing.Harvard.Edu/netmind.shtml} to be automatically notified
(via email) of future program updates.

\section{Overview}

As only four commands are required to use \EI\, the program can be
easily run interactively, or in batch mode as a regular Gauss program
(other Gauss commands can be used at any point if desired).  Each
command may also be used with many optional globals and subcommands.
An example of these commands are as follows:
\begin{verbatim}
library ei;                 @ initialize EI @
dbuf = ei(t,x,n,1,1);       @ run main prog., save results in dbuf @
call eigraph(dbuf,"tomog"); @ draw tomography graph @
v = eiread(dbuf,"betab");   @ extract precinct estimates from dbuf @
\end{verbatim}
In most applications, \texttt{eigraph} and \texttt{eiread} would
likely be run multiple times with different options chosen, and other
commands would be included with these four to read in the data
(\texttt{t}, \texttt{x}, and \texttt{n}).\footnote{For example, to
  read in the data from an ascii text file, you can include these two
  commands after the \texttt{library} command: \texttt{clear t,x,n;}
  and \texttt{loadvars sample.asc t x n}.  A detailed example appears
  on page \pageref{ei}.}

In this section, I describe these four commands through a very simple
use of \EI.  (Refer to the reference section below for further details
and more sophisticated uses.  That section also includes sample data
you can run with simple examples, and minor differences required when
running under Unix.)  For this purpose, and without loss of
generality, I use the running example from the book portrayed in Table
2.3 (page 31) and reproduced below in Table \ref{t:snotEI}.  This
example uses the fraction of the voting age population who are black
($X_i$), the fraction turning out to vote ($T_i$), and the number of
voting age people ($\Npp$) in each precinct ($i=1,\ldots,p$) to infer
the fraction of blacks who vote ($\beta_i^b$) and the fraction of
whites who vote ($\beta_i^w$), also in each precinct.
\begin{table}[t]
  \begin{center}
    \leavevmode
    \setlength{\extrarowheight}{4pt}
    \begin{tabular}{r<{  }@{}r@{}c@{}l@{}c@{}c@{}>{  }l}
      \textbf{Race of}&&\multicolumn{3}{c}{\textbf{Voting Decision}}\\
      \textbf{Voting Age} && &                   &&\\
      \textbf{Person} && \multicolumn{1}{c}{\underline{V}ote} && \underline{N}o vote &  \\\cline{2-5}
      \underline{b}lack &\vline& $\beta^b_i$ &\vline& $1-\beta_i^b$ &\vline& $X_i$ \\
      \cline{2-5}
      \underline{w}hite &\vline& $\beta^w_i$ &\vline& $1-\beta_i^w$ &\vline&$1-X_i$\\ \cline{2-5}
      && $T_i$       &&       $1-T_i$  && \\
    \end{tabular}
    \caption[Simplified Notation for Precinct $i$]{\em  Notation
      for Precinct $i$.  The goal is to infer the quantities of interest,
      $\beta_i^b$ (the fraction of blacks who vote) and $\beta_i^w$ (the
      fraction of whites who vote), from the aggregate variables $X_i$
      (the fraction of voting age people who are black) and $T_i$ (the
      fraction of people who vote), along with $\Npp$ (the number of
      voting age people).  This is Table 2.3 (page 31) in the book.}
    \label{t:snotEI}
  \end{center}
\end{table}

Of course, this is only an example; \EI\ can be used for any analogous
ecological inference.  It can be used for \emph{larger tables}, as
described in Sections 8.4 and 15.1 of the book (see procedure
\texttt{ei2} in the reference section).  \EI\ can also be used without
modification if $T_i$ is an average of individual-level
\emph{dichotomous variables} (as in the example), \emph{interval-coded
  discrete variables}, or fully \emph{continuous variables}, so long
as the variables are scaled to the [0,1] interval (see Section 14.3).

\begin{enumerate}
\item At the start of every program, use \texttt{library
    ei;} to initialize \EI.  (To reinitialize all globals
  between programs, use \texttt{eiset;} if desired.)
\item Run the main procedure, \texttt{dbuf = ei(t,x,n,1,1);}, which
  takes three $p\times 1$ vectors as \emph{inputs}: \texttt{t} (e.g.,
  the fraction of the voting age population turning out to vote);
  \texttt{x} (e.g., the fraction of the voting age population who are
  black); and \texttt{n} (e.g., the total number of people in the
  voting age population).  (The remaining two inputs are for optional
  covariates; for the basic model, set them each to 1 for no
  covariates.)  The \emph{output} of this procedure is \texttt{dbuf},
  a gauss data buffer.  A data buffer is a single entity that can be
  saved from gauss in a single file on disk, but which can include
  many different strings, vectors, matrices, or other elements (see
  the Gauss command \texttt{vput} in the Gauss manual for more
  information).  After running \texttt{ei}, it is a good idea to save
  \texttt{dbuf} on disk (with a Gauss command such as \texttt{save
    dbuf;}).  While \texttt{ei} is running, various intermediate
  results appear on the screen, such as iteration numbers, but no
  final results of substantive interest are printed.  As this step is
  by far the longest (about 2--5 minutes for 250 observations on an HP
  715/80 or Pentium 90), it is usually most convenient to run this
  separately from the remaining steps.
\item The output data buffer from \texttt{ei} includes a large variety
  of different results useful for understanding the results of the
  analysis. A minimal set of nonrepetitive information is stored in
  this data buffer, and a large variety of other information can be
  easily computed from it.  Fortunately, you do not need to know
  whether the information you request is stored or computed as both
  are treated the same.\footnote{Also stored in the output data buffer
    are the data input to this procedure, and all global options
    chosen during the analysis.  As such, this data structure is very
    convenient if you are in the habit of following \emph{the
      replication standard} by making publically available all the
    data and information necessary to replicate your published
    analyses.  In fact, the output data buffer is an automatically
    created and self-documented ``replication data set.'' To follow
    the replication standard, you only need to provide this \EI\ data
    buffer to the ICPSR or some other public archive; others will then
    have access to the data and information necessary to replicate
    your results (of course, they won't have any other information
    unless you decide to provide it).  To replicate an \EI\ analysis,
    you only need to load in the data buffer, \texttt{loadm dbuf;},
    and call this procedure: \texttt{dbufNew=eirepl(dbuf);}.  See Gary
    King, ``\hlink{Replication,
      Replication}{http://gking.harvard.edu/replrepl/replrepl.html},''
    \emph{PS: Political Science and Politics}, with comments from
    nineteen authors and a response, Vol.\ XXVIII, No.\ 3 (September,
    1995): 443--499.} To extract information from the data buffer, two
  procedures are available:
  \begin{enumerate}
  \item For \emph{graphics}, use \texttt{eigraph(dbuf,"name");}, where
    \texttt{dbuf} is the data buffer that is the output of
    \texttt{ei}, and \texttt{name} can be any of a long list of
    ready-made graphs.  For example, use \texttt{eigraph(dbuf,"fit");}
    to assess the fit of the model, \texttt{eigraph(dbuf,"tomog");} to
    print a tomography graph, or \texttt{eigraph(dbuf,"xgraph");} to
    display a scattercross graph.
  \item For \emph{numerical information}, use \texttt{v =
      eiread(dbuf,"name")}, where \texttt{v} is the item extracted,
    \texttt{dbuf} is the data buffer that is the output of
    \texttt{ei}, and \texttt{name} can be any of a long list of output
    possibilities.  For example, use \texttt{betab} for a vector of
    point estimates of $\beta_i^b$, \texttt{ci80w} for 80\% confidence
    intervals for $\beta_i^w$, or \texttt{sum} to print a summary of
    district-level estimates and information.
  \end{enumerate}

\end{enumerate}

\section{Advanced Topics: Bayesian Model Averaging}

\EI includes facilities to define and run multiple \EI models all at
once, and to combine them using formal Bayesian model averaging
procedures.  The procedure, \texttt{dbufdef =
eimodels\_def(dbufdef,num,t,x,n,zb,zw);}, enables you to define a
model specificiation (including all input variables
\texttt{t,x,n,zb,zw}, and all globals input to \EI), to assign it a
scalar integer model number, \texttt{num}, and to store it in a ``meta
data buffer,'' \texttt{dbufdef}.  The meta data buffer contains a set
of individual data buffers, each containing everything needed to
define one specification.  You can run \texttt{eimodels\_def} multiple
times to store as many models in \texttt{dbufdef} as needed (the first
time you run the procedure, set \texttt{dbufdef="";}).

Once all model specifications have each been defined and stored, use
\texttt{dbufrun = eimodels\_run(dbufdef);} to run \texttt{ei} on each
of the stored specifications and save the results in another meta data
buffer, \texttt{dbufrun}. The results of each \texttt{ei} run can be
read using \texttt{v = eiread(dbufrun);} and
\texttt{eigraph(dbufrun);} while specifying the model number with the
global variable, \texttt{\_EIMetaR} (default of this variable is 1
which reads the results of \texttt{ei} run for Model 1).

Finally, run \texttt{dbufavg = eimodels\_avg(dbufrun);} to use
Bayesian Model Averaging to combine the results from multiple models
with weights (the procedure calculates) indicating how much the data
supports each model.  The results of this model averaging can be read
using \texttt{v = eiread(dbufavg);} and \texttt{eigraph(dbufavg);}
where \texttt{dbufavg} is the output data buffer. To use this method,
the prior distribution for covariates should be specified for each
model using the global variables, \texttt{\_EalphaB} and
\texttt{\_EalphaW}, so that the resulting posterior distribution is
known to be proper.

% ------------------------------------------------------------------

\section{Reference}

\subsection{EI} \label{ei}
\markright{Reference: EI}

\paragraph{\underline{Format}:} \texttt{dbuf = ei(t,x,n,zb,zw);}, where
\texttt{t}, \texttt{x}, and \texttt{n} are $p\times 1$ vectors,
\texttt{zb} is a scalar 1 for no covariates or a $p\times k^b$ matrix
of covariates, and \texttt{zw} is the scalar 1 for no covariates or a
$p\times k^w$ matrix of covariates (do not include a constant term).
\texttt{dbuf} is an output data buffer.

\paragraph{\underline{Purpose}:}
\texttt{ei} gives observation-level estimates (and various related
statistics) of $\beta^b_i$ and $\beta_i^w$ given variables $T_i$ and
$X_i$ ($i=1,\dots,n$) in this accounting identity: $T_i = \beta^b_iX_i
+ \beta^w_i(1-X_i)$.  Results are stored in \texttt{dbuf}, a data
buffer that should be read with \texttt{eiread} or graphed with
\texttt{eigraph}.  See the model in Chapter 6, and extensions in
Chapter 9.

\paragraph{\underline{Example}:}
The following example uses the sample file \texttt{sample.asc} that
comes with \EI.  This file includes hypothetical data from 75
precincts in rows and three variables, \texttt{t}, \texttt{x}, and
\texttt{n}, in columns.  You may use any other variable names or
ASCII files and may include recodes or other Gauss commands in this
file.
\begin{verbatim}
new;                        @ clear workspace @
library ei;                 @ initialize libraries @
clear t,x,n;                @ clear all variables in dataset @
loadvars sample.asc t x n;  @ load variables from disk file @
dbuf = ei(t,x,n,1,1);       @ run ei @
save racevote=dbuf;         @ save dbuf in file racevote.fmt @
\end{verbatim}
The command \texttt{loadvars} is a keyword included with this program
that makes it easy to read variables in from an ASCII data set (see
page \pageref{loadvars} below).  If your data are in a Gauss data
file, use \texttt{subdatv} instead (see page \pageref{subdatv} below).
Because running \texttt{ei} takes several minutes, it is usually
easiest to write one program such as this, and to examine the results
interactively (or with a separate program) using \texttt{eiread} and
\texttt{eigraph}.

\paragraph{\underline{Globals}:}
\begin{description}
\item[\_EalphaB] (cols(Zb)$\times 2$) matrix of means (in the first
  column) and standard deviations (in the second) of an independent
  normal prior distribution on elements of $\alpha^b$.  If you specify
  \texttt{Zb}, you should probably specify a prior, at least with mean
  zero and some variance (default=\{.\}; which indicates no prior).
  (See Equation 9.2, page 170, to interpret $\alpha^b$).

\item[\_EalphaW] (cols(Zw)$\times 2$) matrix of means (in the first
  column) and standard deviations (in the second) of an independent
  normal prior distribution on elements of $\alpha^w$.  If you specify
  \texttt{Zw}, you should probably specify a prior, at least with mean
  zero and some variance (default=\{.\}; which indicates no prior).
  (See Equation 9.2, page 170, to interpret $\alpha^w$).

\item[\_Ebeta] Standard deviation of the ``flat normal'' prior on
  $\bbetau^b$ and $\bbetau^w$.  The flat normal prior is uniform
  within the unit square and dropping outside the square according to
  the normal distribution.  Set to zero for no prior (default).
  Setting to positive values probabilistically keeps the estimated
  mode within the unit square.  0.25 is a reasonable value to
  experiment with at first.

\item[\_Ebounds] 1 if set CML bounds on parameters automatically
  unless z's are included; 0 if don't use bounds; $k\times 2$ (where
  $k$ is the number of starting values) or $1\times 2$ matrix to
  indicate upper$\sim$lower bounds. (Do not confuse the bounds
  referred to here with the bounds on the quantities of interest.)
  Default=1.

\item[\_Ecdfbvn] Determines which procedure to use for computing the
  area of the bivariate normal distribution above the unit square: 1
  based on the Gauss function \texttt{CDFBVN}; 2 Martin van der Ende's
  method (based on D.R. Divgi, ``Calculation of the univariate and
  bivariate Normal integral,'' \emph{Annals of Statistics}, 1979,
  903-910, with additional options available for this method in the
  proc \texttt{cdfbvn\_div}); 3 Integration of log of the unit square;
  4 Direct integration on unit square; 5, fairly accurate and fast,
  based on direct integration on the unit square from a new Gauss
  internal procedure (DEFAULT); 6, most accurate but slow, based on a
  cdfbvn procedure by Alan Genz (using results from Drezner, Z.\ and
  G.O.\ Wesolowsky, 1989.\ ``On the computation of the bivariate
  normal integral,'' \emph{Journal of Statist. Comput. Simul.} 35:
  101--107).  See Appendix F.

  Option 5 (the default) appears to be the best tradeoff between speed
  and accuracy currently available (and so this global should not be
  changed to anything other than 6, which is more accurate but much
  slower, unless you have a good reason to do so).  However,
  fundamental progress remains to be made on methods of integrating
  the bivariate normal, as all currently available methods are
  innacurate and jump discontinuously and for very small values.
  Because of this, small values are truncated at the global
  \texttt{\_EcdfTol}, which you may wish to adjust.

\item[\_EcdfTol] Tolerance for the lncdfbvn function (when
  \texttt{\_Ecdfbvn=5}, its default), with smaller calculated values
  truncated at the value of this global (DEFAULT=2.220446e-11).  This
  can be any positive number, although lncdfbvn gets imprecise for
  small values.  Only set to smaller values if you think you need the
  precision, such as if most of your values of $T_i$ or $X_i$ are very
  small.

\item[\_Echeck] 1 check inputs and globals and give nice error
  messages if problems (default); 0 don't check, which saves some
  time.  There is little reason to choose 0 unless you are running a
  large number of estimations and you are certain all the inputs are
  correctly specified.  (Inessential global: not stored in dbuf.)

\item[\_EdirTol] direction tolerance for CML convergence.
  Default=0.0001.  Set to smaller values if most of your values of
  $T_i$ or $X_i$ are very small.

\item[\_EdoML] 1 do maximum likelihood (default); 0 don't do maximum
  likelihood, using instead the values of $\phi$ stored in
  \texttt{\_EdoML\_phi} and vcphi in \texttt{\_EdoML\_vcphi}.

\item[\_EdoML\_phi] if \texttt{\_EdoML}$=1$, this should include a
  vector of values of $\phi$ and will be used instead of the output of
  the likelihood maximization.  (This global is ignored unless
  \texttt{\_EdoML=1}.)

\item[\_EdoML\_vcphi] if \texttt{\_EdoML}=1, this should include a
  matrix of values of estimated variance matrix $V(\phi)$ and will be
  used instead of the output of the likelihood maximization procedure.
  (This global is ignored unless \texttt{\_EdoML=1}.)

\item[\_EdoSim] 1 do simulations (default); 0 don't do simulations;
  $-1$ don't do simulations or compute the maxlik variance (use this
  option for computing conditional log-likelihood of eta's).

\item[\_Eeta] Automatically includes $X_i$ in the inputs \texttt{Zb}
  and/or \texttt{Zw}.  The actual inputs \texttt{Zb} and \texttt{Zw}
  must be set to 1 if the default is changed.  Using this global is
  better than explicitly including $X_i$ in the inputs, because eiread
  and eigraph will be ``aware'' of the contents of \texttt{Zb} and
  \texttt{Zw}.  If you set this global, it is generally best to also
  set the priors \texttt{\_EalphaB} and \texttt{\_EalphaW}.  See
  Chapter 9, and the parameterization in Equation 9.2 (page 170).
  Options include:
  \begin{itemize}
  \item \texttt{\_Eeta=0} excludes $X_i$, which is equivalent
    to setting $\alpha^b=\alpha^w=0$ (default).
  \item \texttt{\_Eeta=1} sets \texttt{zb}=\texttt{x}, \texttt{zw}=1,
    which estimates $\alpha^b$ and fixes $\alpha^w=0$
  \item \texttt{\_Eeta=2} sets \texttt{zb}=1, \texttt{zw}=\texttt{x},
    which estimates $\alpha^w$ and fixes $\alpha^b=0$
  \item \texttt{\_Eeta=3} sets \texttt{zb}=\texttt{zw}=\texttt{x},
    which estimates $\alpha^b$ and $\alpha^w$.
  \item Set to a $4\times 1$ vector with elements \texttt{\_Eeta} =
    $\alpha^b\vert\alpha^w\vert\text{se}(\alpha^b)\vert\text{se}(\alpha^w)$
    to fix $\alpha^b$ and $\alpha^w$, and their standard deviations,
    during estimation.
  \item Finally, set to a $3\times 1$ vector $4|a|b$ to set
    \texttt{zb}=\texttt{x} and \texttt{zw}=1, to estimate $\alpha^b$,
    and fix $\alpha^w=$a and its standard error to b. Set to a
    $3\times 1$ vector $5|a|b$ to set \texttt{zb}=1 and
    \texttt{zw}=\texttt{x}, to estimate $\alpha^w$, and fix
    $\alpha^b=$a and its standard error to b.
  \end{itemize}
  
\item[\_EI\_vc] $M\times 2$ matrix ($M\geq 1$), each row of which
  represents instructions for one attempt to compute an estimated
  positive definite variance matrix of $\phi$.  The procedure exits
  after the first positive definite hessian is found.  Options to
  include in various rows are: \{1 0\} the usual numerical hessian
  computation (using Gauss's hessp.src proc); \{1 $d$\} use usual
  hessian procedure and then adjust eigenvalues together so they are
  greater than $d$; \{2 $f$\} use wide step lengths at $f$ fraction
  falloff in the likelihood function; \{3 $f$\} use quadradic
  approximation with falloff in likelihood function set at $f$; \{4
  0\} use a generalized inverse (to deal with singularity) and a
  generalized cholesky (to deal with non-positive definateness) based
  on work in progress by Jeff Gill and Gary King; \{5 0\} use wide
  step lengths but check that the gradients for each are correct (and
  if necessary search for better ones); \{-1 0\} avoid the computation
  of the variance covariance matrix in case of non-positive
  definiteness and use the singular value decomposition for the
  multinomial normal sampling (i.e. \_EisT has to be set to 0). In
  order to use this option, also make sure to define relatively narrow
  upper and lower bounds of the parameters by using \_Ebounds.
  DEFAULT=\texttt{\{1 0, 4 0, 2 0.1, 2 0.05, 3 0.1, 1 0.1, 1 0.2\}}.
  The variance computation only very rarely gets beyond the second
  try.

  When the likelihood surface is normal (i.e., quadratic), which is
  true asymptotically, all options produce identical results.  In
  practice, this procedure is useful for ensuring that a positive
  definite variance matrix can be found due to numerical, rather than
  theoretical or empirical, difficulties, as can happen when the mode
  of the truncated normal is far from the unit square due to
  imprecision in the function that computes the bivariate normal CDF.
  (Another, sometimes better, way to fix these numerical problems is
  to reduce the variances of the priors in \texttt{\_Erho} and
  \texttt{\_Esigma}.)  Because importance sampling is used after this
  procedure, different values of the variance matrix can produce
  identical estimates of the quantities of interest.  Be sure to
  verify that the simulations are being appropriately drawn from the
  estimated contours (see compare the right two figures in eigraph's
  \texttt{tomogS}).

\item[\_EIgraph\_bvsmth] smoothing parameter for nonparametric
  estimation; used only if \texttt{\_Enonpar}=1. Default=0.08.  (The
  same parameter controls the nonparametric bivariate density
  estimation for diagnostic purposes in \texttt{eigraph}.)

\item[\_EisChk] 0 to do nothing (default); 1 change \texttt{lnir} from
  the scalar mean importance ratio to a
  (\texttt{\_Esims}*\texttt{\_Eisn})$\times$(rows($\phi$)+1) matrix
  containing the log of the importance ratio as the first column and
  normal simulations of $\tilde{\phi}$ as the remaining columns.  Also
  changes \texttt{PhiSims} from the mean and standard deviation of the
  posterior phi's to a \texttt{\_Esims}$\times$rows($\phi$) matrix of
  normal simulations of phi.

\item[\_EiLlikS] 1 if save (\_Esims$\times 1$) the log-likelihoods
  evaluated for each simulation; 0 saves only the means of these
  likelihoods (default).  These can be used for computing the marginal
  likelihood.
  
\item[\_EisFac] factor to multiply by estimated variance matrix in the
  normal approximation for use in importance sampling, or set to $-1$
  to use normal approximation only or $-2$ to condition on the maximum
  posterior estimates.  Adjust this, \texttt{\_Eisn}, or
  \texttt{\_Eist} if eiread's \texttt{resamp} larger than 15 or 20.
  If this is set too low, estimation variability will not be
  sufficient and your confidence intervals may be too narrow; it must
  be greater than zero and should probably be at least one.  See
  Section 7.5. (Default=4).

\item[\_Eisn] factor to multiply by \texttt{\_Esims} to compute the
  number of normals to draw before resampling.  This is used to to try
  to get \texttt{\_Esims} samples from exact posterior.  Increase this
  or change \texttt{\_EisFac} or \texttt{\_EisT} if \texttt{resamp} is
  larger than 15 or 20.  Default=10.  See Section 7.5.

\item[\_EisT] 0 (default) to use multivariate normal density to draw
  random numbers for initial approximation for importance sampling; or
  if greater than 2, use the multivariate Student $t$ density, with
  degrees of freedom \texttt{\_Eist}.  Use this, \texttt{\_EisFac}, or
  \texttt{\_Eisn} if \texttt{resamp} is larger than 15 or 20.  See
  Section 7.5.

\item[\_EmaxIter] Maximum number of iterations for CML.  Default=500.

\item[\_EnonEval] Number of nonparametric density evaluations for each
  tomography line (default=11).  Only used if \texttt{\_EnonPar}=1.

\item[\_EnonNumInt] Number of points to evaluate for numerical
  integration in computing the denominator for the bivariate kernel
  density (default=50).  Only used if \texttt{\_EnonPar}=1.

\item[\_EnonPar] 0 do not run nonparametric model (default); 1 run
  nonparametric model.  (When choosing nonparametric estimation, only
  relevant options will be available under eigraph and eiread.)  See
  Section 9.3.2.

\item[\_EnumTol] Numerical tolerance.  A homogeneous precinct is one
  for which $X_i<$\texttt{\_EnumTol} or $X_i>$(1-\texttt{\_EnumTol}).
  Default is 0.0001.  Set to smaller values if most of your values of
  $T_i$ or $X_i$ are very small.

\item[\_Eprt] 0 print nothing; 1 print only final output from each
  stage; 2 also prints friendly iteration numbers etc (default); 3
  also prints all sorts of checks along the way.  Use eiread and
  eigraph instead of this global to see output.  (Inessential global:
  not stored in dbuf.)

\item[\_Eres] If items are \texttt{vput} into \texttt{\_Eres} before
  running \texttt{ei}, they are passed through into dbuf.  For
  example, identifiers for each aggregate unit would be useful in
  interpreting the results, or using them in subsequent analyses (try:
  \texttt{\_Eres=vput(\_Eres,caseid,"caseid")} before calling EI.  If
  a title is vput and given the name \texttt{titl}, the title is
  printed in convenient places.  See \texttt{eiread} for further
  information.  Do not use the name of any globals to this procedure
  or options listed under eiread(), or your variable will be lost.

\item[\_Erho[1]] Standard deviation of normal prior on $\phi_5$ for
  the correlation; set to 0 to fix $\phi_5$ to a second element,
  \texttt{\_Erho[2]}; set to $-1$ to estimate without a prior.
  Default=0.5.  \texttt{\_Erho} should be a scalar unless the first
  element is 0, in which case it should be a $2\times 1$ vector.  See
  Section 7.4.

\item[\_Eselect] Controls which observations are included in the
  estimation stage, including both likelihood maximization and
  importance sampling. All observations are included in the simulation
  stage unless you delete them from the data set before starting \EI.
  This allows users to base the truncated bivariate normal contours on
  a subset of observations that might be more representative (such as
  those for which $T_i$ is not 0 or 1).  Set to $p\times 1$ vector to
  of 1's to include and 0's to exclude individual observations.

\item[\_EselRnd] Set to scalar 1 to include all observations not
  already deleted by \texttt{\_Eselect} (default), or a scalar greater
  than 0 and less than 1 to randomly select this fraction of
  observations in the estimation stage.  This global is especially
  useful for speeding up estimation in very large datasets, since
  thousands of observations are not always needed for estimating
  $\phi$.  Since all observations will still be included in the
  simulation stage, precinct-level estimates of all quantities of
  interest will still be available.  (If used with EI2, each iteration
  of EI includes a different randomly selected set of observations.)

\item[\_Esigma] Standard deviation of an underlying normal
  distribution, from which a half normal is constructed as a prior for
  both $\sigmau_b$ and $\sigmau^w$.  Note: the expected value under
  this prior is \texttt{\_Esigma}$\sqrt{2/\pi}
  \approx$\texttt{\_Esigma}0.8.  Set to zero or negative for no prior.
  Default = 0.5.  See Section 7.4.

\item[\_Esims] Number of simulations. Default is 100.

\item[\_Estval] \emph{For gradient methods:} Scalar 1, use best guess
  starting values (default); or set to $k\times 1$ vector of starting
  values.  If \texttt{\_Eeta[1]}=0 (its default), $k=5$ with elements
  guesses of $\phi$, that is on the scale of estimation.  If you have
  starting values on the untruncated normal scale,
  $\psiu=\{\bbeta^b,\bbeta^w,\sigma_b,\sigma_w,\rho\}$, you can
  reparameterize as in this example:
  \texttt{\_Estval=eireparinv(.5|.5|.2|.2|-.1)}.  If
  \texttt{\_Eeta[1]}=1, 2, 4, or 5, $k=6$; if \texttt{\_Eeta}=3,
  $k=7$; and if covariates are used and rows(\texttt{\_Eeta})=4, then
  $k$ is 5 plus the number of covariates included, with Zb coming
  before Zw.

  \emph{For a grid search:} Set \texttt{\_Estval} to scalar 0 (with 5
  divisions per zoom), or to a scalar integer greater than or equal to
  3 for a grid search with this number of divisions per zoom.  (That
  is, the grid search procedure divides the parameter space into a
  number of divisions, evaluates the likelihood for every combination
  of values on all the parameters, chooses the region of highest
  likelihood, zooms in and repeats the procedure on the narrower
  parameter region.  This continues until differences in the
  parameters differ by the global \texttt{\_Edirtol}.

\item[\_EvTol] Numerical tolerance for the conditional variance
  calculation.  Must be greater than 0; Default is $1e-322$.
\end{description}

\paragraph{\underline{Output Global}:}
\texttt{\_Eres}, a data buffer containing the time of completion, if
procedure finishes.  If the procedure is aborted, \texttt{\_Eres} is a
data buffer containing the data and all items computed up to that
point.

\subsection{EI2}
\markright{Reference: EI2}

\paragraph{\underline{Format}:} \texttt{dbuf2 = ei2(V,dbuf,zb,zw);},
where \texttt{v} is a $p\times 1$ vector, \texttt{dbuf} is a data
buffer output from \texttt{ei()} or an earlier run of this procedure,
\texttt{zb} is a scalar 1 for no covariates or a $p\times k^b$ matrix
of covariates, and \texttt{zw} is the scalar 1 for no covariates or a
$p\times k^w$ matrix of covariates.  (Do not include a constant term.)
\texttt{dbuf2} is the output data buffer.

\paragraph{\underline{Purpose}:}
Gives observation-level estimates of parameters from $2\times C$
tables with $C>2$.  First decompose the $C$ outcome categories into a
series of dichotomous choices.  For example, in the $2\times 3$
running example in the book (see Table 2.2, page 31),
Democrat/Republican/No Vote can be decomposed into Vote/No vote (which
is analyzed as usual with \texttt{ei}) and Democrat/Republican, among
those who vote (which \texttt{ei2} can analyze given $V_i$, the
Democratic fraction of the vote, and the output from \texttt{ei}).  If
desired, covariates in either \texttt{ei} or \texttt{ei2} can be used
to avoid possibly incorrect independence assumptions.

In this running example, this procedure will estimate $\lambda^b_i$
and $\lambda_i^w$ (and various related statistics) given $V_i$
(Democratic fraction of the vote) and given an \emph{estimate} of
$x_i$ (black fraction of \emph{voters}) in this accounting identity:
$V_i = \lambda^b_ix_i + \lambda^w_i(1-x_i)$.  A first-stage estimate
of $\beta_i^b$ from \texttt{ei()} is used to estimate $x_i$, since
$x_i=\beta_i^bX_i/T_i$ (these first stage results should be studied
with eiread and eigraph before proceeding).  Multiple imputation is
used to incorporate the extra uncertainty due to $x_i$ being
estimated.  The details of this procedure are discussed in Section
8.4.

The separately imputed data buffers are stored in the output global
\texttt{\_ei2\_mta}.  The simulations from each data buffer are
combined in a single data buffer, \texttt{dbuf2}.  Both can be read
with \texttt{eiread} or graphed with \texttt{eigraph}. Because these
programs work for both \texttt{ei()} and \texttt{ei2()}, reading
estimates requires using the names ``betaB'' and ``betaW'' for the
parameters (even though they might be more appropriately called
``lambdaB'' and ``lambdaW'' for this particular example).  In
\texttt{dbuf2}, \texttt{x} are the mean posterior estimates of $x_i$,
and \texttt{x2} is a ($p\times\texttt{\_Esims}$) matrix of simulations
of $x_i$ (see eiread and eigraph for further information).  Diagnostic
outputs in this data buffer, such as \texttt{resamp}, are those from
the final imputed data set.

\paragraph{\underline{Example}:}
The following example references a file \texttt{sample2.asc}, included
with \EI, that includes data from 100 precincts in rows along with
four variables, \texttt{v}, \texttt{t}, \texttt{x}, and \texttt{n}, in
columns.
\begin{verbatim}
new;                          @ clear workspace @
library ei;                   @ initialize libraries @
clear v,t,x,n;                @ clear all variables @
loadvars sample2.asc v t x n; @ load variables from disk file @
dbuf = ei(t,x,n,1,1);         @ estimate betaB and betaW @
save racevote=dbuf;           @ save dbuf in file racevote.fmt @
dbuf2 = ei2(v,dbuf,1,1)       @ estimate lambdaB and lambdaW @
save raceDem=dbuf2;           @ save dbuf2 in file raceDem.fmt @
\end{verbatim}

For example, given the $2\times 3$ running example in the book (Table
2.2, Page 30), \texttt{eiread(dbuf,"betab")} are point estimates of
$\beta_i^b$, the fraction of blacks who vote;
\texttt{eiread(dbuf,"betaw")} are point estimates of $\beta_i^w$, the
fraction of whites who vote; \texttt{eiread(dbuf2,"betab")} are point
estimates of $\lambda_i^b$, the fraction of blacks who vote
Democratic, among those who vote; and \texttt{eiread(dbuf2,"betaw")}
are point estimates of $\lambda_i^w$, the fraction of whites who vote
Democratic, among those who vote.

\paragraph{\underline{Globals}:}
\begin{description}
\item[\texttt{ei} globals] All the globals from \texttt{ei} apply here
  too.

\item[\_EI2\_m] Number of data sets to multiply impute (must be less
than or equal to \texttt{\_Esims}). If it is set to -1, \texttt{ei2}
will use the posterior mean of $\beta^b$ to impute one data set. This
option should be used only when \texttt{ei2} has the computational
difficulty since the standard error is likely to be
underestimated. The point estimate will not be biased.) , default=4.

\end{description}

\paragraph{\underline{Output Global}:}
\texttt{\_ei2\_mta}, a ``meta-data buffer'' having as elements
separate data buffers named dbuf1, dbuf2, dbuf3,\ldots, one for each
of \texttt{\_ei2\_m} imputations.  Each is the output from a separate
imputation run.  If \texttt{ei2} fails, \texttt{\_ei2\_mta} contains
results from each of the imputation runs already completed.
\texttt{eiread} and \texttt{eigraph} understand when you give it a
meta-data buffer as an argument; by default it uses \texttt{dbuf1}
(this default can be changed by setting the \texttt{eiread} global
\texttt{\_EIMetaR} to the number of the imputation data buffer you wish
to analyze.)

For example, to save the meta-data buffer to a file for safe keeping,
and then examine the results from the first data buffer, add these
lines to the example above:
\begin{verbatim}
save MraceDem=_ei2_mta;        @ save meta-data buffer @
eiread(_ei2_mta,"sum");        @ show summary statistics @
graphon;                       @ open graphics window @
eigraph(_ei2_mta,"estsims");   @ plot simulations @
graphoff;                      @ close graphics window @
\end{verbatim}

\subsection{EIGRAPH} \label{eigraph}
\markright{Reference: EIGRAPH}

\paragraph{\underline{Format}:} \texttt{eigraph(dbuf,"name");}, where
\texttt{dbuf} is a data buffer that is the output of \texttt{ei} and
\texttt{name} is a string chosen from the options listed below.

\paragraph{\underline{Purpose}:}
Graphs results taken from the \texttt{ei} output data buffer (i.e.,
for $2\times 2$ tables) and for \texttt{ei2} data buffers ($2\times C$
tables).  See \texttt{eiread} for extracting numerical information.

\paragraph{\underline{Example}:}
This example assumes you have run \texttt{ei} and saved a data buffer
(called \texttt{racevote.fmt} in the example below).  \texttt{eigraph}
can be run interactively or via a batch file.  Here is a typical set
of commands that you might run interatively on a DOS-based machine.
(The calls to \texttt{graphclr;} prevent one graphic image being
overlaid on top of the next.)  Gauss for Unix requires you to open
graphics windows explicitly; see the reference item GRAPH on page
\pageref{graph} below.)
\begin{verbatim}
new;
library ei;              @ initialize ei @
loadm dbuf = racevote;   @ load in data buffer @
graphon;                 @ open graphics window @
eigraph(dbuf,"xgraph");  @ check info from bounds @
graphclr;                @ clear graphics screen @
eigraph(dbuf,"fit");     @ evaluate model fit @
graphclr;                @ clear graphics screen @
eigraph(dbuf,"tomog");   @ tomography plot @
graphclr;                @ clear graphics screen @
eigraph(dbuf,"post");    @ draw district-level posterior @
graphoff;                @ close graphics window @
\end{verbatim}

\paragraph{\underline{Options}:}
Choose any of the items below in place of \texttt{name} to see the
corresponding graph.  (Items that use true values from
individual-level data will only work if you have \texttt{vput} this
information into \texttt{\_Eres} prior to running \texttt{ei} under
the name \texttt{truth}.  These are useful for verifying the method if
individual-level data are available, but they are not useful for most
ecological inference applications, where this information is not
available.  The options are included here only because I needed them
while writing the book.)
\begin{description}
\item[beta] betaB and betaW

\item[betaB] density estimate (i.e., a smooth version of a histogram)
  of point estimates of $\beta^b_i$'s, with whiskers.

\item[betaW] density estimate (i.e., a smooth version of a histogram)
  of point estimates of $\beta^w_i$'s, with whiskers.

\item[betabw] lines,Tlines,bivar,Tbivar.  See Figure 10.8, page 214.

\item[bias] combination of biasB, biasW, TbiasB, TbiasW. See the
  points in Figure 13.2, page 238.

\item[biasB] $X_i$ by \emph{estimated} $\beta_i^b$.

\item[biasW] $X_i$ by \emph{estimated} $\beta_i^w$.

\item[bivar] \emph{estimated} $\beta_i^b$ by $\beta_i^w$.  See the
  right graph in Figure 13.6, page 243.

\item[boundX] combination of boundXB, boundXW, boundXB overlaid on
  TbiasB, boundXW overlaid on TbiasW.  See Figure 13.2, page 238.

\item[boundXB] $X_i$ by the bounds on $\beta_i^b$ (each precinct
  appears as one vertical line), see the lines in the left graph in
  Figure 13.2, page 238.

\item[boundXW] $X_i$ by the bounds on $\beta_i^w$ (each precinct
  appears as one vertical line), see the lines in the right graph in
  Figure 13.2, page 238.

\item[estsims] All the simulated $\beta_i^b$'s by all the simulated
  $\beta_i^w$'s.  The simulations should take roughly the shape of the
  mean posterior contours, except for those sampled from outlier
  tomography lines. Homogeneous precincts are excluded.

\item[fit] Combination of xtfit and tomogP

\item[fitT] Combination of xtfit and tomogT

\item[goodman] $X_i$ by $T_i$ plot with Goodman's regression line
  plotted. See Figure 4.1, page 60.

\item[lines] $X_i$ by $T_i$ plot with one \emph{estimated} line per
  precinct.  Homogenous precincts do not appear.  See Figure 10.8, page 214.

\item[movie] For \emph{each} observation ($i=1,\ldots,p$), one page of
  graphics appears with the posterior distribution of $\beta_i^b$ and
  $\beta_i^w$ (with whiskers drawn at each simulated value), a plot of
  simulated values of $\beta_i^b$ by $\beta_i^w$ from the tomography
  line) and the numerical values of $T_i$, $X_i$, $N_i$, and $i$.
  After the graph for each observation appears, the user can choose to
  view the next observation (hit return), jump to a specific
  observation number (type in the number and hit return), or stop (hit
  ``s'' and return).  See Figure 8.1, page 148.

\item[movieD] For \emph{each} observation ($i=1,\ldots,p$), one
  tomography plot appears (like tomogD) with the line for observation
  $i$ darkened.  After the graph for each observation appears, the
  user can choose to view the next observation (hit return), jump to a
  specific observation number (type in the number and hit return), or
  stop (hit ``s'' and return).  See Figure 5.1, page 81.

\item[nonpar] combination of tomogD and a nonparametric density
  estimation with contour plot and surface plot representations.  See
  Figure 9.9, page 195.

\item[post] combination of postB and postW.  See Figure
  10.4, page 208.

\item[postB] density estimate of (weighted) district quantity of
  interest, the posterior distribution for $B^b$.  See Figure
  10.4, page 208.

\item[postW] density estimate of (weighted) district quantity of
  interest, the posterior distribution for $B^w$.  See Figure
  10.4, page 208.

\item[prectB] plot of estimated $\beta^b_i$ by true $\beta^b_i$.  See
  Figure 10.5, page 210.

\item[prectW] plot of estimated $\beta^w_i$ by true $\beta^w_i$.  See
  Figure 10.5, page 210

\item[profile] Profile plots --- that is plots of the posterior of
  each element of $\phi$, holding constant all other values at their
  maxima.  The maximum posterior point, as found by CML is displayed
  as a vertical line from the max down to the bottom of the graph.
  After seeing these plots, you can zoom in by changing
  \texttt{\_eigraph\_pro} (as is useful for seeing how curved the
  likelihood is near the maximum).  (Kinks in the profile are due to
  imprecision in the cdfbvn function necessitating truncation by
  \texttt{\_EcdfTol}.)  This option is very useful for verifying
  whether the maximization procedure did indeed find the maximum.

\item[profileR] Profile plots of the cdfbvn function, (Equation 6.15,
  page 104).  These are plots of R by each element of $\phi$, holding
  constant all other values at their maxima.  The maximum posterior
  point, as found by CML is displayed as a vertical line from the max
  down to the bottom of the graph.  After seeing these plots, you can
  zoom in by changing \texttt{\_eigraph\_pro}.  Kinks in the profile
  are due to imprecision in the cdfbvn function necessitating
  truncation by \texttt{\_EcdfTol}.  This option is very useful for
  verifying whether the posterior is numerical stable near the
  maximum.

\item[ptile] combination of ptileB and ptileW.  See Figure
   10.7, page 213.

\item[ptileB] true percentile at which $\beta_i^b$ falls by estimated
  $\beta_i^b$.  See Figure 10.7, page 213.

\item[ptileW] true percentile at which $\beta_i^w$ falls by estimated
  $\beta_i^w$.  See Figure 10.7, page 213.

\item[results] combination of postB, postW, betaB, betaW.

\item[sims] combination of simsB and simsW.  See Figure 10.6, page
  212.

\item[simsB] simulations of $\beta_i^b$ by true $\beta_i^b$.  See
  Figure 10.6, page 212.

\item[simsW] simulations of $\beta_i^w$ by true $\beta_i^w$.  See
  Figure  10.6, page 212.

\item[TbiasB] $X_i$ by \emph{true} $\beta_i^b$.  See the points in the left
  graph in Figure 13.2, page 238.

\item[TbiasW] $X_i$ by \emph{true} $\beta_i^w$.  See the lines in the right
  graph in Figure  13.2, page 238.

\item[Tbivar] \emph{true} $\beta_i^b$ by $\beta_i^w$.  See Figure
  13.2, page 238.

\item[Tlines] $X_i$ by $T_i$ plot with one \emph{true} line per
  precinct.  Homogenous precincts do not appear.  See Figure 3.1, page
  41.

\item[tomog] tomography plot with ML contours. See Figure 10.2, page
  204.  Dashed lines are added for observations \texttt{\_Eselect}'d
  out of the estimation stage, and a note appears at the bottom left
  and top right corners if any observations are included for which
  $T_i=0$ or $T_i=1$ (since as tomography lines they are represented
  as barely visible points).  The global \texttt{\_EselRnd} is
  ignored.

\item[tomogCI] tomography plot with 80\% confidence intervals.
  Confidence intervals appear on the screen in red with the remainder
  of the tomography line in yellow.  On printed output, which is often
  easier to see than the screen display, the confidence interval
  portion is printed thicker than the rest of the line.  See Figure
  9.5, page 179.

\item[tomogCI95] tomography plot with 95\% confidence intervals.
  Confidence intervals appear on the screen in red with the remainder
  of the tomography line in yellow.  On printed output, which is often
  easier to see than the screen display, the confidence interval
  portion is printed thicker than the rest of the line.  See Figure
  9.5, page 179.

\item[tomogD] tomography plot with data only.  See Figure 5.1, page
  81.  Dashed lines are added for observations \texttt{\_Eselect}'d
  out of the estimation stage, and a note appears at the bottom left
  and top right corners if any observations are included for which
  $T_i=0$ or $T_i=1$ (since as tomography lines they are represented
  as barely visible points).  The global \texttt{\_EselRnd} is
  ignored.

\item[tomogE] tomography plot with estimated mean posterior
  $\beta_i^b$ and $\beta_i^w$ points.  Dashed lines are added for
  observations \texttt{\_Eselect}'d out of the estimation stage, and a
  note appears at the bottom left and top right corners if any
  observations are included for which $T_i=0$ or $T_i=1$ (since as
  tomography lines they are represented as barely visible points).
  The global \texttt{\_EselRnd} is ignored.

\item[tomogP] tomography plot with mean posterior contours.  Dashed
  lines are added for observations \texttt{\_Eselect}'d out of the
  estimation stage, and a note appears at the bottom left and top
  right corners if any observations are included for which $T_i=0$ or
  $T_i=1$ (since as tomography lines they are represented as barely
  visible points).  The global \texttt{\_EselRnd} is ignored.

\item[tomogS] combination of tomog, tomogp, tomogCI, and Tbivar (or
  estsims if truth isn't available).  See Figures 10.2 (page 204), 9.5
  (page 179), 7.1 (page 126).  Dashed lines are added for observations
  \texttt{\_Eselect}'d out of the estimation stage, and a note appears
  at the bottom left and top right corners if any observations are
  included for which $T_i=0$ or $T_i=1$ (since as tomography lines
  they are represented as barely visible points).  The global
  \texttt{\_EselRnd} is ignored.

\item[tomogT] tomography plot with true $\beta_i^b$ and $\beta_i^w$
  points.  See Figure 7.1 (page 126). Dashed lines are added for
  observations \texttt{\_Eselect}'d out of the estimation stage, and a
  note appears at the bottom left and top right corners if any
  observations are included for which $T_i=0$ or $T_i=1$ (since as
  tomography lines they are represented as barely visible points).
  The global \texttt{\_EselRnd} is ignored.

\item[truth] combination of post, precB, precW (compare truth to
  estimates at district and precinct-level).  See Figures 10.4 (page
  208) and 10.5 (page 210).

\item[xgraph] a scattercross with data plotted.  See Figure 12.1, page
  227.

\item[XgraphC] a scattercross with data plotted, with size
  proportional to $N_i$.

\item[xt] basic $X_i$ by $T_i$ scatterplot

\item[xtC] basic $X_i$ by $T_i$ scatterplot with circles sized
  proportional to $\Npp$ or some other variable defined by the global
  \texttt{\_eigraph\_circ}.

\item[xtfit] $X_i$ by $T_i$ plot with estimated $\rE(T_i|X_i)$ and
  conditional 80\% confidence intervals.  See Figure 10.3, page 206.

\item[xtfitg] xtfit with Goodman's regression line superimposed
\end{description}

\paragraph{\underline{Globals}:} These globals need
not be changed for the most common uses.
\begin{description}
\item[\_eigraph\_bb] string: xlabel for $\beta_i^b$ by $\beta_i^w$
  plots. Default=\texttt{"betaB"}

\item[\_eigraph\_bbhi] high end for betaB plots. Default=1.

\item[\_eigraph\_bblo] low end for betaB plots. Default=0.

\item[\_eigraph\_bvsmth] smoothing parameter for nonparametric density
  estimation; used only for option \texttt{nonpar}.  Default=0.08.
  (The same parameter is used in \texttt{ei} for nonparametric
  estimation.

\item[\_eigraph\_bw] string: ylabel for $\beta_i^b$ by $\beta_i^w$
  plots.  Default=\texttt{"betaW"}

\item[\_eigraph\_bwhi] high end for betaW plots. Default=1.

\item[\_eigraph\_bwlo] low end for betaW plots. Default=0.

\item[\_eigraph\_dbuf] if 1, create data buffer called
  \texttt{\_eigraph\_dbuf} with px, py, pz inputs to the contour plot
  when option \texttt{nonpar} is chosen.

\item[\_eigraph\_eval] number of equally spaced evaluations on each
  axis of the nonparametric density estimation. Default=31.

\item[\_eigraph\_loess] 1 if show simulated and fitted loess for
  xtfit; 0 if show fit only (default)

\item[\_eigraph\_pro] rows($\phi$)$\times 2$.  The first column is the
  lower limit and the second is the upper limit of the range in which
  eigraph's \texttt{profile} command uses for plotting.  The default,
  when set to missing, is to use eiread's \texttt{ebounds} option.

\item[\_eigraph\_smpl] Set this in the range (0,1] to randomly select
  this fraction of observations to use in tomography plots.  This is
  useful if $p$ is so large that it is difficult to see patterns
  because of the large number of lines.  Default=1.

\item[\_eigraph\_T] string: ylabel for xt plots. Default=\texttt{"T"}

\item[\_eigraph\_Thi] High end of T graphs. Default=1.

\item[\_eigraph\_Tlo] How end of T graphs. Default=0.

\item[\_eigraph\_thick] value to add to line thickness parameter
  (default=1)

\item[\_eigraph\_X] string: xlabel for xt plots. Default=\texttt{"X"}


\item[\_eigraph\_Xhi] High end of X graphs. Default=1.

\item[\_eigraph\_Xlo] How end of X graphs. Default=0.

\item[\_eigraphC] scalar: Multiply by circle size to change sizes.
  Default=1, which means do not change.  set to (0,1) to reduce circle
  size and to (1,$\infty$] to increase circle size.  Must be greater
  than zero.

\item[\_EIMetaR] If \texttt{dbuf} is a meta-data buffer (i.e., from the
  \texttt{\_ei2\_mta} output global of ei2), this global denotes which
  of the imputed data buffers stored in dbuf should be accessed when
  running \texttt{eigraph}.  Default=1.

\item[\_tomogClr] Colors for each contour drawn on tomography plots,
  default=\{ 12, 9, 10, 11, 13, 5 \}.

\item[\_tomogPct] Vector of percentage values at which to draw
  contours.  One contour is drawn for each element with this fraction
  of this distribution's volume falling within the drawn ellipse.
  Default=\{.5, .95\}; rows(\_tomogClr) must be $\geq$
  rows(\_tomogPct).
\end{description}


\subsection{EIMODELS} \label{eimodels}
\markright{Reference: EIMODELS}

\paragraph{\underline{Format}:} 
\texttt{dbufdef = eimodels\_def(dbufdef,num,t,x,n,zb,zw);}, where
\texttt{dbufdef} is an input meta data buffer which contains all the
model specifications previously defined as separate EI data buffers
(or \texttt{""} for the first run), and to which the current model
definition will be added, \texttt{num} is the scalar integer model
number to be assigned to the current model definition, and the rest
are defined as in \texttt{ei}: \texttt{t}, \texttt{x}, and \texttt{n}
are $p\times 1$ vectors, \texttt{zb} is a scalar 1 for no covariates
or a $p\times k^b$ matrix of covariates, and \texttt{zw} is the scalar
1 for no covariates or a $p\times k^w$ matrix of covariates (do not
include a constant term).  The output, \texttt{dbufdef}, is an updated
meta data buffer which contains the current model definition along
with the previously stored ones.

\texttt{dbufrun = eimodels\_run(dbufdef);}, where an input,
\texttt{dbufdef}, is the output meta data buffer from
\texttt{eimodels\_def} which contains multiple model definitions. The
output, \texttt{dbufrun}, is another meta data buffer, containing one
data buffer for each data buffer of model specifications in \texttt{dbufdef}.

\texttt{dbufavg = eimodels\_avg(dbufrun);}, where \texttt{dbufrun} is the
output meta data buffer from \texttt{eimodels\_run} containing the
results of \texttt{ei} run for multiple models. \texttt{dbufavg} is a
regular output data buffer.

\paragraph{\underline{Purpose}:}
\texttt{eimodels\_def} stores multiple model definitions in a meta
data buffer by adding one model at a time.  Each model can use
different covariates and/or different values of various global
variables used for \texttt{ei} estimation.  \texttt{eimodels\_def}
only stores model specifications; it does not run any models.  Results
are stored in \texttt{ndbuf2}, a meta data buffer which should be read
with \texttt{eiread} with the model number specified as the value of
the global variable, \texttt{\_EIMetaR}.

\texttt{eimodels\_run} takes the output meta data buffer from
\texttt{eimodels\_def} as an input and runs all the models stored in
that meta data buffer.  The output is another meta data buffer which
contains the results of \texttt{ei} runs for each model as a single
component data buffer. This output data buffer should be read with
\texttt{eiread} and \texttt{eigraph} with the model number specified
using the global variable, \texttt{\_EIMetaR} (the default for which
is 1, for the first model).

\texttt{eimodels\_def} and \texttt{eimodels\_run} are useful when you
have different numbers of \EI models to run but want to store the
results in one meta data buffer rather than saving them as many
separate data buffers. The two procedures can also be used for the
method called \textit{Bayesian Model Averaging} as explained below.

\texttt{eimodels\_avg} takes the output meta data buffer from
\texttt{eimodels\_run} as an input and implements Bayesian Model
Averaging over all the models stored in that data buffer. To use this
procedure, three global variables are required for each of the
component EI runs.  First, when a model includes covariates, the prior
distributions for $\alpha^a$ and $\alpha^b$ need to be specified for
Bayesian Model Averaging in order to ensure that the posterior
distribution is proper. The two global variables, \texttt{\_Ealpha\_B}
and \texttt{\_Ealpha\_W}, perform this function.  Second, \EI uses the
Laplace-Metropolis estimator (default) or the harmonic mean estimator,
for computing the marginal likelihoods. It is recommended that the
number of simulations, \texttt{\_Esims}, should be set to a relatively
large number in order to improve the precision of this estimation.  If
you use the harmonic mean estimator, \texttt{\_EiLikS} should be set
to 1 for each model so that the output data buffer from
\texttt{eimodels\_run} stores the values of the log-likelihood at each
simulation. These values are necessary to compute the marginal
likelihood for this method. Finally, the output data buffer should be
read with \texttt{eiread} and \texttt{eigraph}.

\paragraph{\underline{Example}:}
The following example uses the sample file \texttt{sample.asc} that
comes with \EI.  Consult section \ref{ei} for the description of data
and explanation of the commands to load the data set. The code first
stores the three different model definitions in a meta data buffer
using \texttt{eimodels\_def}, and then it runs \texttt{ei} for each of
the three models. Finally, it implements the Bayesian Model Averaging
to combine the results from the three models using
\texttt{eimodels\_avg}.
\begin{verbatim}
new;                                     @ clear workspace @
library ei;                              @ initialize libraries @
clear t,x,n;                             @ clear all variables in dataset @
loadvars sample.asc t x n;               @ load variables from disk file @

eiset;                                   @ clear for Model 1 @
_Eres = vput(_Eres, "Model 1", "titl");  @ print out model number @
_Eeta = 1;                               @ zb=x, zw=1 @
_Ealpha_B = {0 2};                       @ prior for betaB @
_Esims = 1000;                           @ number of simulation @
dbufdef = eimodels_def("",1,t,x,n,1,1);  @ save Model 1, the first model @

eiset;                                   @ clear for Model 2 @
_Eres = vput(_Eres, "Model 2", "titl");
_Eeta = 2;                               @ zb=1, zw=x @
_Ealpha_W = {0 2};                       @ prior for betaW @
_Esims = 1000;
dbufdef = eimodels_def(dbufdef,2,t,x,n,1,1); @ save Model 2 @

eiset;                                   @ clear for Model 3 @
_Eres = vput(_Eres, "Model 3", "titl");
_Eeta = 3;                               @ zb=x, zw=x @
_Ealpha_B = {0 2};                       @ prior for betaB @
_Ealpha_W = {0 2};                       @ prior for betaW @
_Esims = 1000;
dbufdef = eimodels_def(dbufdef,3,t,x,n,1,1); @ save Model 3 @

save rvdef = dbufdef;                    @ save ndbuf in file rvdef.fmt @
dbufrun = eimodels_run(dbufdef);         @ run ei on all the models @
save rvrun = dbufrun;                    @ save ndbres in file rvrun.fmt @

_EIMetaR = 1;
call eiread(dbufrun, "sum");             @ summary of ei run for Model 1 @

_EIMetaR = 3;
graphon;
call eigraph(dbufrun, "tomog");          @ tomography plot for Model 3 @
graphoff;

_EI_bma_prior = {0.2,0.4,0.4};           @ set prior model probabilities @
dbufavg = eimodels_avg(dbufrun);         @ Bayesian Model Averaging @
save rvavg = dbufavg;                    @ save dbres in file rvavg.fmt @

graphon;
call eiread(dbufavg,"sum");              @ summary for dbres @
call eigraph(dbufavg,"post");            @ draw district level posterior @
graphoff;                                @ close graphics window @
\end{verbatim} 

\paragraph{\underline{Globals}:}
\begin{description}

\item[\_EImodels\_save] set to ``filename'' if you want to save the
  results at each iteration in a data buffer on the disk (default =
  ``\mbox{ }''; for no save).

\item[\_EImodels\_est] scalar, if 1 the Laplace-Metropolis estimator0
will be used to estimate the marginal likelihood (default). If 2, the
harmonic mean estimator (an importance sampling scheme) will be used.
  
\item[\_EI\_bma\_prior] (($\#$ of models)$\times 1$) The discrete
  prior probability of each model in the ascending order of the model
  number (default = 0 which assigns the uniform prior). The elements
  of this vector must sum up to 1.
\end{description}


\subsection{EIREAD} \label{eiread}
\markright{Reference: EIREAD}

\paragraph{\underline{Format}:} \texttt{v = eiread(dbuf,"name");}, where
\texttt{dbuf} is the output from procedure \texttt{ei} or
\texttt{ei2}, \texttt{name} is the item you desire to be extracted or
computed from \texttt{dbuf} (see the list below), and \texttt{v} is
the output vector, matrix, or string.

\paragraph{\underline{Purpose}:}
To extract or calculate numerical information from the output data
buffer of \texttt{ei} (for $2\times 2$ tables) or \texttt{ei2} (for
$2\times C$ tables), from the ecological inference model.

\paragraph{\underline{Example}:}
This example assumes that you have first run \texttt{ei} and saved a
data buffer (called \texttt{racevote.fmt} in the example below).
\texttt{eiread} can be run interactively or via a batch file.  Here is
a typical set of commands that you might run, although many other
options to replace the items in quotes are available below.
\begin{verbatim}
new;
library ei;                     @ initialize ei @
loadm dbuf = racevote;          @ load in data buffer @
format/rd 7,4;                  @ nicely format output; 4 decimal places @
eiread(dbuf,"sum");             @ summary info @
betaw = eiread(dbuf,"betaw");   @ extract point estimates for betaW @
betabCI = eiread(dbuf,"ci80b"); @ extract betaB confidence intervals @
call eiread(dbuf,"goodman");    @ run Goodman's regression @
\end{verbatim}
The last example illustrates the Gauss ``call'' statement, which
discards the output of eiread, leaving only the results that are
explicitly printed to the screen by some eiread options.  To save
output from these procedures in an ascii text file, include commands
like these:
\begin{verbatim}
output file=filename.txt reset;  @ start saving to filename.txt @
  betaw;                         @ print previously saved output @
  eiread(dbuf,"betab");          @ print betaB results @
output off;                      @ close output file @
\end{verbatim}

\paragraph{\underline{Globals}:}
\begin{description}
\item[\_Eprt] if $\geq 1$, in addition to outputting results in
  \texttt{v}, print a display of the item to the screen (available for
  options whose description begins with an asterisk, below) (default);
  0 do not print.
  
\item[\_EIMetaR] If \texttt{dbuf} is a meta-data buffer (i.e., from
  the \texttt{\_ei2\_mta} output global of ei2; the output buffer from
  \texttt{eimodels\_def} and \texttt{eimodels\_run}), this global
  denotes which of the imputed data buffers (in case of
  \texttt{\_ei2\_mta}) or which of the stored model (in case of
  \texttt{eimodels\_def} and \texttt{eimodels\_run}) in dbuf should be
  accessed when running \texttt{eiread} and \texttt{eigraph}.  Default=1.
\end{description}

\paragraph{\underline{Options}:}
Choose from the following items for \texttt{name}.  The possibilities
include items stored as is in \texttt{dbuf} (and so could also be
retrieved with the Gauss command \texttt{vread}), calculated from
\texttt{dbuf}, special optional items that can be added ahead of time,
and additional items that can be computed but require the added items.
From the perspective of the user, items in each of these categories
can be treated identically, as all calculations are automatic.  Items
are not case sensitive.  (Items that use true values from
individual-level data will only work if you have \texttt{vput} this
information into \texttt{\_Eres} prior to running \texttt{ei} under
the name \texttt{truth}.  These are useful for verifying the method if
individual-level data are available, but they are not useful for most
ecological inference applications, where this information is not
available.  The options are included here only because I needed them
while writing the book.)

When used with an EI2 output data buffer (i.e., for $2\times C$
tables), \texttt{eiread} uses use the mean posterior estimate for
$X_i$ for some items; the correct multiply imputed values of $x_i$
(\texttt{x2}) are used only where specifically noted.

\begin{description}
\item[\_EalphaB] value of this global (priors on $\alpha^b$)

\item[\_EalphaW] value of this global (priors on $\alpha^w$)

\item[\_Ebeta] value of this global (priors on $\bbetau$)

\item[\_Ebounds] value of this global (bounds for CML, not quantities
  of interest).

\item[\_Ecdfbvn] value of this global (method of calculating CDF of
  the bivariate normal)

\item[\_EdirTol] value of this global (CML convergence tolerance)

\item[\_EdoML] value of this global (do maxlik)

\item[\_EdoML\_phi] value of this global (input $\phi$'s).  Only
  relevant if \texttt{\_EdoML}=0.

\item[\_EdoML\_vcphi] value of this global (input variance-covariance
  of $\phi$'s).  Only relevant if \texttt{\_EdoML}=0.

\item[\_Eeta] value of this global (expanded model specifications; see
  Chapter 9).

\item[\_EI\_bma\_prior] value of this global (prior model
  probabilities for Bayesian Model Averaging).

\item[\_EI\_vc] value of this global (variance matrix computation)

\item[\_EImodels\_save] value of this global (file name for
  \texttt{eimodels\_run}).

\item[\_EIgraph\_bvsmth] value of this global (smoothing parameter for
  nonparametric density estimation)

\item[\_EisChk] value of this global (check importance sampling)

\item[\_EisFac] value of this global (number to multiply by variance
  matrix by in importance sampling or $-1$ for normal approximation)

\item[\_EisN] value of this global (first stage importance sampling
  factor)

\item[\_EisT] value of this global (multivariate $t$ or normal for
  importance sampling)

\item[\_EmaxIter] value of this global (maximum iterations for CML)

\item[\_EnonEval] value of this global (number of nonparametric
  density evaluations for each tomography line).  Only relevant if
  \texttt{\_Enonpar}=1.

\item[\_EnonNumInt] value of this global (number of points to evaluate
  for numerical integration in computing the denominator for the
  bivariate kernel density).  Only relevant if \texttt{\_Enonpar}=1.

\item[\_EnonPar] value of this global (0, parametric or 1,
  nonparametric, estimation).

\item[\_EnumTol] value of this global (numerical tolerance for
  homogeneous and unanimous precincts)

\item[\_Erho] value of this global (prior on $\rho$).  See Section
  7.4.

\item[\_Eselect] value of this global (vector of 0/1 to delete/select
  observations during likelihood stage, or 1 to select all).

\item[\_EselRnd] value of this global (fraction of observations to
  select randomly).

\item[\_Esigma] value of this global (priors on $\sigmau_b$ and
  $\sigmau_w$).  See Section 7.4.

\item[\_Esims] value of this global (number of simulations).  See
  Appendix F.

\item[\_Estval] value of this global when \texttt{ei} was run
  (starting values).

\item[\_n] $p\times 1$: the original variable $N_i$ from the first
  stage EI analysis.  This works only for EI2.

\item[\_t] $p\times 1$: the original variable $T_i$ from the first
  stage EI analysis.  This works only for EI2.

\item[\_EvTol] value of this global (numerical tolerance for the
  conditional variance).

\item[\_x] $p\times 1$: the original variable $X_i$ from the first
  stage EI analysis.  This works only for EI2.

\item[\_Ez] $2\times 1$: number of covariates (see Chapter 9),
  including implied constant term for Zb$\vert$Zw; also clears and
  sets this in global memory.

\item[ABounds] $2\times 2$: aggregate bounds rows:lower,upper;
  columns:betab,betaw.  See Chapter 5.

\item[ABounds2] $2\times 2$: aggregate bounds for the $\lambda$'s from
  EI2, rows:lower,upper; columns:lambdaB,lambdaW.  See Chapter 5.

\item[AggBias] $4\times 2$: regressions of true $\beta_i^b$ and
  $\beta_i^w$ on a constant and $X_i$.  Requires \texttt{truth} to
  have been vput into \texttt{\_Eres} prior to running \texttt{ei}.
  See Table 11.1, page 219.

\item[aggs] \texttt{\_Esims}$\times 2$: simulations of district-level
  quantities of interest, $\hat{B}^b\sim\hat{B}^w$.  See Section 8.3.
  (Uses \texttt{x2} simulations when used with \texttt{ei2}).

\item[AggTruth] $2\times 1$: true district-level $B^b$ and $B^w$.
  Requires \texttt{truth} to have been vput into \texttt{\_Eres} prior
  to running \texttt{ei}.  See Chapter 2.

\item[beta] $p\times 2$ point estimates of $\beta_i^b$ (in the first
  column) and $\beta_i^w$ (in the second) based on their mean
  posterior.  See Section 8.2.  (Uses \texttt{x2} simulations when
  used with \texttt{ei2}.)

\item[betaB] $p\times 1$ point estimates of $\beta_i^b$ based on its
  mean posterior.  See Section 8.2.  (Uses \texttt{x2}
  simulations when used with \texttt{ei2}.)

\item[betaBs] $p\times$\texttt{\_Esims}: simulations of $\beta_i^b$.
  See Chapter 8.  (Uses \texttt{x2} simulations when used
  with \texttt{ei2}).

\item[betaW] $p\times 1$ point estimate of $\beta_i^w$ based on its
  mean posterior.  See Section 8.2.  (Uses \texttt{x2}
  simulations when used with \texttt{ei2}).

\item[betaWs] $p\times$\_Esims: simulations of $\beta_i^w$.  See
  Section 8.2.  (Uses \texttt{x2} simulations when used
  with \texttt{ei2}).

\item[bounds] $p\times 4$: bounds on $\beta_i^b$ and $\beta_i^w$,
  lowerB$\sim$upperB$\sim$lowerW$\sim$upperW.  See Chapter
  5.

\item[bounds2] $p\times 4$: bounds on $\lambda_i^b$ and $\lambda_i^w$,
  lowerB$\sim$upperB$\sim$lowerW$\sim$upperW.  See Chapter
  5, eqns 5.4.

\item[checkR] rows($\phi$)$\times 2$ matrix with rows corresponding to
  $\phi$, columns corresponding to slightly less$\sim$more (by amount
  \texttt{\_EdirTol}) than the MLEs, and each element indicating that
  the CDFBVN function is sufficiently precise (when 1) and
  insufficiently precise (when 0).  This function calculates the $R$
  portion of the likelihood function to make the comparison (Equation
  6.15, p.104).  See option R.

\item[CI50b] $p\times 2$: lower$\sim$upper 50\% confidence intervals
  for $\beta_i^b$.  See Section 8.2.  (Uses \texttt{x2}
  simulations when used with \texttt{ei2}).

\item[CI50w] $p\times 2$: lower$\sim$upper 50\% confidence intervals
  for $\beta_i^w$.  See Section 8.2.  (Uses \texttt{x2}
  simulations when used with \texttt{ei2}).

\item[CI80b] $p\times 2$: lower$\sim$upper 80\% confidence intervals
  for $\beta_i^b$.  See Section 8.2.  (Uses \texttt{x2}
  simulations when used with \texttt{ei2}).

\item[CI80bw] $p\times 4$: lowerB$\sim$upperB$\sim$lowerW$\sim$upperW
  80\% confidence intervals for $\beta_i^b$ and $\beta_i^w$.  See
  Section 8.2.  (Uses \texttt{x2} simulations when used
  with \texttt{ei2}).

\item[CI80w] $p\times 2$: lower$\sim$upper 80\% confidence intervals
  for $\beta_i^w$.  See Section 8.2.  (Uses \texttt{x2}
  simulations when used with \texttt{ei2}).

\item[CI95b] $p\times 2$: lower$\sim$upper 95\% confidence intervals
  for $\beta_i^b$.  See Section 8.2.  (Uses \texttt{x2}
  simulations when used with \texttt{ei2}).

\item[CI95bw] $p\times 4$: lowerB$\sim$upperB$\sim$lowerW$\sim$upperW
  95\% confidence intervals for $\beta_i^b$ and $\beta_i^w$.  See
  Section 8.2.  (Uses \texttt{x2} simulations when used
  with \texttt{ei2}).

\item[CI95w] $p\times 2$: lower$\sim$upper 95\% confidence intervals
  for $\beta_i^w$.  See Section 8.2.  (Uses \texttt{x2}
  simulations when used with \texttt{ei2}).

\item[coverage] $2\times 4$: confidence interval coverage; percent of
  true values within the 50\% and 80\% confidence intervals:
  50b$\sim$80b$\sim$50w$\sim$80w (1st row = means, 2nd = weighted
  means).  Requires \texttt{truth} to have been vput into
  \texttt{\_Eres} prior to running \texttt{ei}.  (Uses \texttt{x2}
  simulations when used with \texttt{ei2}).

\item[CsbetaB] $p\times 1$ confidence interval-based standard error of
  $\beta_i^b$.  (Uses \texttt{x2} simulations when used with
  \texttt{ei2}).

\item[CsbetaW] $p\times 1$ confidence interval-based standard error of
  $\beta_i^w$.  (Uses \texttt{x2} simulations when used with
  \texttt{ei2}).

\item[DataSet] Zb$\sim$Zw$\sim$x$\sim$t, used for input to
  \texttt{eiloglik}.  If \texttt{\_Eselect} is a scalar less than 1
  (for random selection of cases).  The global \texttt{\_EselRnd} is
  ignored.

\item[date] a string containing the date and time at which execution
  completed, as well as the version number and date of the program
  \EI\ that created the input data buffer.

\item[double] $2\times 1$ coefficients from a double regression.
  Requires an EI2 data buffer as input.  See Section 4.3.

\item[Goodman] $2\times 2$: row 1: Goodman's Regression coefficients,
  row 2: standard errors.  See Section 3.1.

\item[lnIR] If \texttt{\_EisChk}=1, this is a
  \texttt{\_Esims}*\texttt{\_Eisn}$\times$rows($\phi$)+1 matrix ,
  containing the log of the importance ratio as the first column and
  normal simulations of $\tilde{\phi}'$ as the remaining columns.  If
  \texttt{\_EisChk}=0, this is the scalar mean importance ratio
  (equivalent to \texttt{meanIR} below).

\item[LogLik] value of log-likelihood at the maximum (unnormalized)

\item[LogLikS] value of log-likelihood at the maximum (unnormalized)
  for each observation $i$.

\item[LLikSims] If \texttt{\_EiLlikS}=1, this is \texttt{\_Esims}$\times
  1$ vector of log-likelihood values for each simulation; otherwise it
  is a scalar mean of these.

\item[Maggs] $2\times 1$: point estimate of 2 district-level
  parameters, $\hat{B}^b$ and $\hat{B}^w$: meanc(aggs).  See Section
  8.3.  (Uses \texttt{x2} simulations when used with \texttt{ei2}).

\item[MeanIR] scalar log of the mean importance ratio

\item[mpPsiu] Mean Posterior of $\psiu$ (rather than MLEs).

\item[N] $p\times 1$: number of individual elements in precinct $i$,
  $\Npp$, an input to \texttt{ei}.

\item[Nb] $p\times 1$: denominator of \texttt{x} and \texttt{t},
  equal to \texttt{x.*n}, $\Nbp$ (e.g., number of blacks in the
  voting age population).

\item[Nb2] for \texttt{ei2} data buffers only,
  $p\times$\texttt{\_Esims}: denominator of \texttt{x2} and
  \texttt{V}, equal to \texttt{x2.*n}, $\Nbp$ (e.g., number of blacks
  in the voting age population).

\item[Nt] $p\times 1$: numerator of \texttt{t}, equal to
  \texttt{t.*n}, $N^t_i$ (e.g., number of people who turnout).

\item[NbN] $p\times1$: $\NbN$ (e.g., number of blacks who don't vote).
  Requires \texttt{truth} to have been vput into \texttt{\_Eres} prior
  to running \texttt{ei}.  See Chapter 2.

\item[NbT] $p\times1$: $\NbV$ (e.g., number of blacks who vote).
  Requires \texttt{truth} to have been vput into \texttt{\_Eres} prior
  to running \texttt{ei}. See Chapter 2.

\item[Neighbor] Freedman et al.'s neighborhood model point estimates
  (i.e., assuming $\beta_i^b=\beta_i^w$, with implied standard errors
  of zero).

\item[nobs] scalar: number of observations, $p$.

\item[Nw] $p\times 1$: equal to \texttt{(1-x).*n}, $\Nwp$ (e.g.,
  number of whites in the voting age population)

\item[Nw2] for \texttt{ei2} data buffers only,
  $p\times$\texttt{\_Esims}: equal to \texttt{(1-x2).*n}, $\Nwp$
  (e.g., number of whites in the voting age population)

\item[NwN] $p\times1$ $\NwN$: (e.g., number of whites who don't vote).
  Requires \texttt{truth} to have been vput into \texttt{\_Eres} prior
  to running \texttt{ei}.  See Chapter 2.

\item[NwT] $p\times1$: $\NwV$ (e.g., number of whites who Turnout).
  Requires \texttt{truth} to have been vput into \texttt{\_Eres} prior
  to running \texttt{ei}. See Chapter 2.

\item[Paggs] $2\times 2$: row 1: $\hat{B}^b$ and $\hat{B}^w$; row 2:
  standard errors.  See Section 8.3.  (Uses \texttt{x2}
  simulations when used with \texttt{ei2}).

\item[Palmquist] scalar: Palmquist's Inflation Factor.  See Equation
  3.14, page 52.

\item[ParNames] character vector of names for $\phi$
  (\texttt{\_cml\_parnames}).

\item[phi] maximum posterior estimates from CML.

\item[PhiSims] If \texttt{\_EisChk}==1, \texttt{PhiSims} is a
  \texttt{\_Esims}$\times$rows($\phi$) matrix of random simulations of
  $\phi$; otherwise, it is a rows($\phi$)$\times 2$ matrix of the
  means (in the first column) and standard deviations (in the second
  column) of the simulations (which are the mean and standard
  deviations of the posterior distribution of $\phi$).  See Section
  8.2.

\item[Pphi] $2\times 5$ maximum posterior estimates.  row 1: $\phi$,
  row 2: standard errors.  See Chapter 7.

\item[psi] reparameterized $\phi$ into ultimate truncated scale.  See
  Section 6.2.2.

\item[PsiTruth] $5\times 1$: true values of $\psi$ (i.e., on truncated
  scale).  Requires \texttt{truth} to have been vput into
  \texttt{\_Eres} prior to running \texttt{ei}. See Table 10.3, page
  207.

\item[psiu] $\psiu$, which was reparameterized from $\phi$ into
  untruncated scale.  See Equation 7.4, page 136.

\item[R] The sum of the log of the volume above the unit square under
  the bivariate normal, $R(\bbetau,\Sigmau)$.  This is the last piece
  of the likelihood function.  See checkR.

\item[Ri] The log of the volume above the unit square under the
  bivariate normal, $R(\bbetau,\Sigmau)$ for each $i$.  This is the
  last piece of the likelihood function, and will differ over $i$ only
  if covariates are included.  See checkR.

\item[resamp] number of resampling tries.  This number will range
  between 1 and \texttt{\_Esims} and is better if small.  If it is
  greater than 15--20, you can try adjusting \texttt{\_Eisn} or
  \texttt{\_EisFac} and rerunning \texttt{ei}.

\item[RetCode] CML return code.  zero means everything is ok.

\item[RNbetaBs] $p\times$\texttt{\_Esims}: randomly horizontally
  permuted simulations of $\beta_i^b$.  This is essentially equivalent
  to \texttt{betaBs} except that it randomly permutes estimation
  variation also.  (Uses \texttt{x2} simulations when used with
  \texttt{ei2}).

\item[RNbetaWs] $p\times$\texttt{\_Esims}: randomly horizontally
  permuted simulations of $\beta_i^w$.  This is essentially equivalent
  to \texttt{betaWs} except that it randomly permutes estimation
  variation also.  (Uses \texttt{x2} simulations when used with
  \texttt{ei2}).

\item[sbetaB] $p\times 1$ standard error for the estimate of
  $\beta_i^b$, based on the standard deviation of its posterior.  See
  Section 8.2.  (Uses \texttt{x2} simulations when used
  with \texttt{ei2}).

\item[sbetaW] $p\times 1$ standard error for the estimate of
  $\beta_i^w$, based on the standard deviation of its posterior.  See
  Section 8.2.  (Uses \texttt{x2} simulations when used
  with \texttt{ei2}).

\item[STbetaBs] $p\times$\_Esims: SORTED simulations of $\beta_i^b$
  (e.g., the 80\% confidence interval lower bound is
  \texttt{STbetaBs[int(0.1*\_Esims)]}).  See Section 8.2.
  (Uses \texttt{x2} simulations when used with \texttt{ei2}).

\item[STbetaWs] $p\times$\_Esims: SORTED simulations of $\beta_i^w$
  (e.g., the 80\% confidence interval upper bound is
  \texttt{STbetaWs[int(0.9*\_Esims)]}).  See Section 8.2.
  (Uses \texttt{x2} simulations when used with \texttt{ei2}).

\item[sum] prints a summary of district-level information

\item[t] $p\times 1$: outcome variable proportion, $T_i$ (e.g.,
  turnout); input to \texttt{ei}.  If dbuf is the output from ei2,
  then this option gives $V_i$ (e.g., Democratic fraction of the major
  party vote), an input to \texttt{ei2}.

\item[Thomsen] $2\times 1$: Estimates of $B^b|B^w$ from Thomsen's
  Ecological Logit Model.

\item[titl] string: a title with descriptive information. Must have
  been vput into \texttt{\_Eres} prior to running \texttt{ei}.

\item[truPtile] p x 2: percentile of sorted simulates at which the
  true value falls for $\beta_i^b$ and $\beta_i^w$.  Requires
  \texttt{truth} to have been vput into \texttt{\_Eres} prior to
  running \texttt{ei}. See Figure 10.7, page 213.  (Uses \texttt{x2}
  simulations when used with \texttt{ei2}).

\item[truth] $p\times 2$: true values of the precinct-level quantities
  of interest $\beta_i^b\sim\beta_i^w$.  Must have been vput into
  \texttt{\_Eres} prior to running \texttt{ei}.

\item[truthB] \texttt{truth[.,1]}, the true $\beta_i^b$.  Requires
  \texttt{truth} to have been vput into \texttt{\_Eres} prior to
  running \texttt{ei}.

\item[truthW] \texttt{truth[.,2]}, the true $\beta_i^w$.  Requires
  \texttt{truth} to have been vput into \texttt{\_Eres} prior to
  running \texttt{ei}.

\item[tsims] $100\times$\texttt{\_Esims}$+1$: simulations of $T_i$
  given $X_i$.  rows correspond to 100 values of $X_i$ equally spaced
  between zero and one, columns are $X_i$ and sorted simulations of
  $T_i|X_i$.  See Section 8.5.  (Uses \texttt{x2} simulations when
  used with \texttt{ei2}).

\item[VCaggs] $2\times 2$: variance matrix of 2 district-level
  parameters, $\hat{B}^b$ and $\hat{B}^w$.  See Section 8.3.  (Uses
  \texttt{x2} simulations when used with \texttt{ei2}).

\item[VCphi] global variance matrix of coefficients phi from gvc(). If
  $\texttt{\_EI\_vc}$ is set to \{-1 0 \}, then it returns the inverse of
  the variance matrix. 

\item[x] $p\times 1$: explanatory variable proportion, $X_i$ (e.g.,
  black voting age population); input to \texttt{ei}.  (for
  \texttt{ei2}, this is the mean posterior estimate of $x_i$, as,
  e.g., the black fraction of those voting).

\item[x2] for \texttt{ei2} data buffers only,
  $p\times\texttt{\_Esims}$: simulations of the explanatory variable
  proportion, $x_i$ (e.g., black fraction of voters).

\item[x2rn] for \texttt{ei2} data buffers only,
  $p\times\texttt{\_Esims}$: horizontally randomly permuted
  simulations of the explanatory variable proportion, $x_i$ (e.g.,
  black fraction of voters).  (This is useful because \texttt{x2} has
  only \texttt{\_EI2\_m} unique columns.)

\item[Zb] matrix of covariates for $\beta_i^b$ or 1 for none; as
  affected by \_Eeta; input to \texttt{ei}.  See Section 9.2.1.

\item[Zw] matrix of covariates for $\beta_i^w$ or 1 for none; as
  affected by \_Eeta; input to \texttt{ei}.  See Section
  9.2.1.
\end{description}

\subsection{EIREPL}
\markright{Reference: EIREPL}

\paragraph{\underline{Format}:} \texttt{dbuf=eirepl(dbufIN);}
where \texttt{dbufIN} is a Gauss data buffer output from
\texttt{ei()}, containing t, x, n, Zb, Zw, and all globals in standard
\texttt{ei} format, and where \texttt{dbuf} is an output data buffer.

\paragraph{\underline{Purpose}:}
The purpose of this proc is to make it very easy to replicate results
from an existing data buffer.  It extracts and sets all globals from
\texttt{dbufIN}, extracts the variables \texttt{t}, \texttt{x},
\texttt{n}, \texttt{Zb}, and \texttt{Zw}, and then runs \texttt{ei()}.
The original results stored in \texttt{dbufIN} are not used.  This
procedure has no globals.  Note that \texttt{ei} uses simulation to
approximate various quantities and so you should not expect to
replicate anything to the last decimal point.  If you wish replication
to more decimal points, increase \texttt{\_Esims} before running
\texttt{ei()} the first time or set Gauss's \texttt{rndseed}.

\paragraph{\underline{Example}:}
This example assumes that you have first run \texttt{ei} and saved a
data buffer (called \texttt{racevote.fmt} in the example below), or
you received \texttt{racevote.fmt} from someone else's analysis.
\begin{verbatim}
new;
library ei;              @ initialize ei @
loadm dbuf = racevote;   @ load in data buffer @
dbufNew = eirepl(dbuf);  @ replicate @
\end{verbatim}

\subsection{EISET}
\markright{Reference: EISET}

\paragraph{\underline{Format}:} \texttt{eiset;}

\paragraph{\underline{Purpose}:}
To reset all \EI\ globals to their defaults.

\subsection{GRAPHON, GRAPHONS, GRAPHCLR, GRAPHNO, GRAPHWAIT,\\
GRAPHOFF}
\markright{Reference: GRAPH} \label{graph}

\paragraph{\underline{Format}:}
\texttt{graphon;} to open a graphics window, \texttt{graphons;} to
open a small graphics window, \texttt{graphclr;} to clear a graphics
window already opened, \texttt{graphno;} to leave the graphics window
open and switch output to the text window, \texttt{graphwait;} to
pause between graphs, and \texttt{graphoff;} to wait for a key to be
pressed in the text window and then to close the graphics window.
(Use \texttt{wincloseall;} to close the graphics window without a
pause.)

\paragraph{\underline{Purpose}:} Use these commands for easy
interactive (or batch) window handling for windowing systems only
(such as X-windows under Unix or win95).  \texttt{graphclr;} should be
used between graphs on a dos system so that one graph is not overlaid
on top of the previous one.

\paragraph{\underline{Example}:} After running \texttt{ei} and
saving the data buffer on disk in filename \texttt{racevote.fmt}, a
typical interactive session might include the following:
\begin{verbatim}
new;
library ei;
loadm dbuf = racevote;           @ load in output from ei @
eiread(dbuf,"sum");              @ print summary info @
graphon;                         @ open graphics window @
eigraph(dbuf,"fit");             @ print graph @
graphclr;                        @ clear graphics window @
eigraph(dbuf,"bias");            @ print graph @
graphno;                         @ switch to text window @
eiread(dbuf,"Maggs");            @ print textual info @
graphclr;                        @ clear & switch to graphics @
eigraph(dbuf,"betabw");          @ print graph @
wincloseall;                     @ close graphics window @
\end{verbatim}

\subsection{LOADVARS} \markright{Reference: LOADVARS} \label{loadvars}

\paragraph{\underline{Format}:} \texttt{loadvars file var1
  var2\ldots}, where \texttt{file} is a string with a file name and is
followed by a list of valid Gauss variable names, one for each column
of \texttt{file}.

\paragraph{\underline{Purpose}:}
An ASCII data file is loaded from disk into the variable names listed.
The data file must be delimited by spaces, tabs, or commas.  It is
often convenient to have one row per observation and one variable per
column, but observations may wrap around to the next row if desired.
If used in a batch file, be sure to clear the variable names listed
prior to this procedure.  This procedure is included for the
convenience of users; if you are already familiar with other ways of
loading data into Gauss, they can also be used.

\paragraph{\underline{Example}:}
After including these lines at the top of your program, the variables
will be in memory and available for further use.  \texttt{sample.asc}
is a sample data file that comes with \EI.
\begin{verbatim}
new;
library ei;
clear t,x,n;                @ clear variables in dataset @
loadvars sample.asc t x n;  @ load in vars into memory @
\end{verbatim}

\subsection{SUBDATV}\markright{Reference: SUBDATV}\label{subdatv}

\paragraph{\underline{Format}:} \texttt{call subdatv("dataset",vrs);} where
\texttt{dataset} is a string with the filename of your Gauss data set,
and \texttt{vrs} is a character vector of variable names to read in or
zero to read them all in.

\paragraph{\underline{Purpose}:}
Reads in a Gauss data set into variable names in memory by the same
name.  If used in a batch program, be sure to \texttt{clear} all
variable names prior to issuing this command.  This procedure is
included for the convenience of users; if you are already familiar
with other ways of loading data into Gauss, they can also be used.

\paragraph{\underline{Example}:}
After including these lines at the top of your program, the variables
will be in memory and available for further use.
\begin{verbatim}
new;
library ei;
clear give,dog,a,bone;               @ clear variables in dataset @
call subdatv("C:\data\racedogs",0);  @ load all variables  @
\end{verbatim}

% ------------------------------------------------------------------
\section{Frequently Asked Questions}

\paragraph{How does EI relate to Goodman's regression and the method
  of bounds?}
The only commonly used methods before EI were Goodman's regression and
the method of bounds.  Goodman's regression worked when the
assumptions held but, as Leo Goodman made clear, it did not work when
the assumptions were wrong.  Within the Goodman framework, the data
alone provided no information about whether the assumptions were right
or wrong.  The method of bounds always gave correct ranges into which
the quantities of interest fell, but the ranges were often wider than
was desirable (only in part because the wrong method of computing them
was frequently used).

EI combines the two methods (hence resolving most controversies
between adherents of these two popular approaches) and adds some
additional features.  Instead of there being two situations, as under
Goodman's approach (i.e., the assumptions applied and the method
worked or they don't and it doesn't), we now have five, only the last
one of which is a problem for EI:
\begin{enumerate}
\item Under EI, if the assumptions are correct, you get the right
  answer.  For an example, see Chapter 10 or the Monte Carlos in
  Chapter 9.

\item If the assumptions are wrong, EI still does ``well'' (in the
  sense of small MSE or squared bias) when the bounds (and other
  information in the tomography lines) are sufficiently informative.
  An important point is that the degree to which the bounds are
  informative can easily be assessed from the aggregate data, and so
  the risks of making ecological inferences are largely known.  As one
  example, see Chapter 11.

\item If the assumptions are wrong and the bounds are not sufficiently
  informative, but the diagnostics are sufficiently informative, then
  the assumptions can easily be changed, and EI will do well.  The
  analyses reported in Figure 9.5 (p.\ 179) and the left graph in
  Figure 13.2 (p.\ 238) for aggregation bias and Figures 9.7 and 9.9
  (Pp.\ 187, 195) for distributional violations are examples.  The
  third assumption, no spatial autocorrelation, seems to have minor
  effects.

\item If the assumptions are wrong and the bounds and the diagnostics
  are not sufficiently informative, but the researcher has additional
  qualitative knowledge of the problem, then appropriate assumptions
  can be chosen.  In this case, either EI will do well, or the formal
  measures of uncertainty produced by EI (standard errors and
  confidence intervals, etc., which are based only upon the data and
  model) can be supplemented and expanded accordingly.  Since the
  ecological inference problem is about information that has been
  aggregated away, only by adding some information is it possible to
  make reliable inferences in general.  Qualitative information is of
  course subject to more interpretation and hence more uncertainty,
  but reliable inferences permit no other option other than to add
  assumptions or other information.  The book discusses a lot of ways
  to bring in qualitative information (see also Gary King, Robert
  Keohane, and Sidney Verba.  1994.  \emph{Designing Social Inquiry:
    Scientific Inference in Qualitative Data}. Princeton University
  Press).

\item If the assumptions are wrong and the bounds and the diagnostics
  are not sufficiently informative, and the researcher has no time or
  resources to collect additional qualitative information, then EI
  will perform poorly.  An example of data like this appear in Figure
  9.2 (p.\ 163).  Even in this worst case scenario, and the others, EI
  will be more robust than Goodman's.  By this I mean that the maximum
  amount of bias from EI is capped at a fixed and knowable level, in
  contrast to Goodman's approach.  The dotted line (corresponding to
  $\tau=0$ for the default model) in Figure 9.6 (p.\ 180) shows that
  bias in EI estimates increases with the degree of aggregation bias
  for small levels of aggregation bias; at some point, however, the
  maximum bias maxes out and increases no further.  The point at which
  the error maxes out depends on the data.  Under Goodman's approach,
  the error linearly increases without limit as aggregation bias
  increases.
\end{enumerate}
The likelihood of the first four cases coming up relative to the fifth
(as compared to the likelihood of the assumptions applying vs not
applying under Goodman's) summarizes the advantage of EI.  Basically
what EI does is to chip off pieces of Goodman's worst case (the
assumptions not applying).  The benefits of EI will therefore quite
obviously depend on the area and application and how much effort is
put into collecting qualitative information.

\paragraph{In what precise statistical sense is EI ``Robust''?}
As aggregation bias increases, Goodman's regression becomes biased
without limit.  The bias in EI, in contrast, has a maximum value that
is a function of the bounds.  This is easy to see in Figure 9.6,
p.180; note how the mean absolute error for the dotted line at the
top, for example, is linear in the low aggregation bias region (where
$\alpha<0.5$), but for higher levels of aggregation bias it levels
off; in contrast, the mean absolute error for Goodman's and any method
that does not incorporate the bounds will increase linearly without
limit.  

For a formal definition of robustness, consider the \emph{breakdown
  point} --- the smallest fraction of observations that you'd have to
alter to make your quantity of interest arbitrarily far from the
truth.\footnote{D.L.\ Donoho and P.J.\ Huber. 1983. ``The Notion of
  the Breakdown Point,'' in Peter J.\ Bickel, Kjell A.\ Doksum, and
  J.L.\ Hodges, Eds., \emph{A Feschrift for Erich L.\ Lehmann},
  Belmont, CA: Wadsworth; F.\R. 1971. ``A General Qualitative
  Definition of Robstness,'' \emph{Annals of Mathematical
    Statisticgs}, 42 (December): 1887--1896; and Jonathan N. Wand,
  Kenneth W.\ Shotts, Jasjeet S.\ Sekhon, Walter R. Mebane, Jr.,
  Michael C.\ Herron, and Henry E. Brady, 2002, ``The Butterfly Did
  It: The Aberrant Vote for Buchanan in Palm Beach County, Florida,''
  \emph{American Political Science Review}, forthcoming.}  In
Goodman's regression only one observation is enough, and so it has a
breakdown point of $1/n$.  For EI, even changing all the observations
could not send the estimates out of the unit square, and so its
breakdown point is $>n$.

\paragraph{How does EI relate to the neighborhood model?}
The neighborhood model was designed for the purpose of critiquing the
Goodman model, and is not thought of as a plausible method of
producing ecological inferences in its own right (see pp.\ 43--44).
The neighborhood model's advantage is that it is in the class of
models that are consistent with the method of bounds for each precinct
and thus cannot be rejected solely from information in the aggregate
data (this class of models is described on p.\ 191).  As such, it can
be seen as a special case of EI.  Its disadvantage is its assumption
that $\beta_i^b=\beta_i^w$, which is of course appropriate in some
cases and far off in others, but whichever it is it assumes the answer
to the question being asked.  A consequence of the neighborhood model
assumption having no error term and supposedly holding exactly is that
its standard errors are always zero, which is unreasonable.  If,
somewhat more reasonably, the neighborhood model assumption is
\emph{approximately} plausible for a particular application, then it
is best to run EI with priors suitably adjusted to reflect this
information.  If priors are strong enough and not substantially
contradicted by the likelihood, EI will give similar point estimates
to the neighborhood model, but it will give more reasonable (nonzero)
standard errors and confidence intervals for the precinct-level
quantities.

\paragraph{Can EI give the misleading answers?}
Yes.  Nothing in \EI, this manual, or \emph{A Solution to the
  Ecological Inference Problem} promises to give you the correct
answer every time without thought.  ``The method'' proposed in the
book is \emph{not} what comes spinning out of \EI\ with all globals
set at their defaults.  Appropriate inferences, according to the
argument put forward in the book, require full use of the diagnostics,
to evaluate the amount of information lost in aggregation (such as how
wide the bounds are for different groups of observations) and how well
the model fits and its assumptions apply.  Since assumptions about
joint distributions for $\beta_i^b$ and $\beta_i^w$ cannot be rejected
if they merely have positive mass over \emph{any} curve that connects
the bottom left and top right points of a tomography plot (p.191),
there is no way to make certain inferences about individual level
behavior from aggregate data alone.  The only solution to this
fundamental lack of information is to bring in some of the vast array
of qualitative information available to most social scientists about
the problems we study --- including ethnographies, particiant
observations, partial survey data, journalistic accounts, historical
studies, prior quantitative research, and the like --- the full range
of data collection schemes used in modern social science.
Interpreting qualitative information in the context of statistical
inference is of course open to more interpretation and ambiguity than
formal statistical tests, but stopping at quantitative data,
especially for this problem, is insufficient.

Considerable thought, analysis, and qualitative information may be
necessary to settle on the right version of the model to run.  \EI\
includes dozens of global variables that govern the main parts of the
model; combinations of these globals can produce estimates from
millions of possible specifications, even given identical input
variables.  The choice among these models requires the same degree of
reasoned analysis and reanalysis, checking assumptions, and rerunning
that the appropriate use of any method does.  The actual method of
ecological inference proposed in the book requires careful attention
to each item in the checklist provided in the concluding chapter
(Chapter 16); since several of the items require the user to consult
qualitative evidence and other substantive knowledge about the
problem, this program alone implements only part of the proposed
method.  Moreover, even with considerable thought, some misinformation
or lack of information can sometimes lead to incorrect estimates;
Chapter 9 provides extensive examples of precisely what can go wrong
and under what conditions.  If you have an example where you suspect
that EI does not recover the truth, then one of the problems discussed
in that chapter is likely at fault, and so you might consider some of
the alternative approaches and model extensions also given there.

Finally, if you are comparing EI results to an external source of
information to judge the ``truth'', consider whether the external
source may be biased.  For example, an estimate from survey data is
just an estimate and not necessarily be better than an ecological
inference.  One of the best academic surveys, the National Election
Studies, overestimates turnout by 8-10\% and vote for incumbent House
candidates by about 8\%.  (Even the NES's ``voter validation
studies,'' which check each respondent's turnout from public records,
contain errors.)  Other surveys, especially about controversial
issues, or politically or personally sensitive topics, often generate
larger biases.  The point is that every source of information,
ecological and individual, comes with some potential biases or errors.

See also the questions below on the advantages of EI, computational
problems, statistical fit, and standard error interpretation.

\paragraph{How do I understand EI standard errors?}
\EI\ standard errors (and other uncertainty estimates, such as
confidence intervals, etc.) are logically very similar to, and can be
interpreted analogously to, standard errors for least squares
regression (LS) coefficients:
\begin{enumerate}
\item In LS, the variance of predicted values as estimates of $y_i$,
  $V(\hat y_i+\epsilon)=V(x_ib)+\sigma^2$ do not go to zero as $n$
  gets larger.  In \EI, the variances of the precinct-level parameters
  ($V(\beta^b_i)$ and $V(\beta^w_i)$) do not go to zero as $n$ gets
  larger.

\item In LS, $V(\hat y)$, i.e.\ the ``sample'' variance of $\hat y_i$
  over $i$, does not go to zero as $n$ gets larger.  In \EI,
  $V(\beta^b)$ and $V(\beta^w)$, i.e. the ``sample'' variance of
  $\beta^b_i$ and $\beta^w_i$ over $i$, does not go to zero as $n$
  gets larger.

\item In LS, $V(b)$ goes to zero as $n$ gets larger (where $b$ is
  essentially any scalar function of $\hat y_i$, for all $i$, such as
  a LS coefficient).  In \EI, $V(B)$ goes to zero as $n$ gets
  larger (where B is the weighted average of the $\beta_i$'s over all
  precincts).

\item In LS, the number of explanatory variables we include tends in
  common practice to be an increasing function of the number of
  observations we have, and so we don't see very small standard errors
  unless there's a mistake.  In \EI, covariates are only sometimes
  used to correct for some types of aggregation bias and the number
  included is, in practice, independent of the number of observations
  and the number of quantities that may be of interest.

\item In LS, the variance of a prediction is a function of estimation
  uncertainty (sampling error, $V(b)$, where $b$ is the regression
  coefficient) and fundamental uncertainty ($V(\epsilon_i)$).  In \EI,
  $V(\beta^b_i)$ and $V(\beta^w_i)$ are functions of estimation
  uncertainty ($V(\psi)$, where psi are the 5 parameters of the
  truncated bivariate normal), and fundamental uncertainty ($\Sigma$,
  which is composed of three elements of $\psi$).

\item In LS, it is standard practice is to include in the formal
  computation of the standard error only estimation and fundamental
  variability and to exclude uncertainty due to the possibility of
  omitted variable bias, endogeneity, measurement error, selection
  bias, etc.  It is possible to include these other possible problems
  in the computation of the standard error, but few computer programs
  allow it.  In \EI, the standard errors include only estimation and
  fundamental uncertainty, and exclude possible violations of the 3
  model assumptions that the model in an application may not be robust
  to (i.e., serious violations of no aggregation bias, truncated
  bivariate normality, and spatial independence).  Only by using the
  diagnostics and bringing in additional qualitative information, not
  represented in the aggregate data, can one add assessments of
  uncertainty to the formal measures given by the program.  Of course,
  many of the standard problems of regression do not affect ecological
  inferences.
\end{enumerate}

\paragraph{How do I get a quick list of options and globals?}  Use the
Gauss help facility.  As long as the library command (\texttt{library
  ei;}) has been issued, you can ask for help on any of the
EI commands.  For example, in DOS, you can type Alt-H,H,eiread to get
all of eiread's options.

\paragraph{What computational details should I check?}
Make sure that the maximization procedure worked properly.  For
example, make sure that eiread's \texttt{retcode} is zero.  Is the
variance matrix from the normal approximation reasonable?  Most
importantly, check to make sure the simulations are coming from the
mean posterior contours by comparing eigraph's \texttt{tomogp} to
\texttt{estsims} (these are both in \texttt{tomogS}); verify that
eigraph's \texttt{post} gives unimodal distributions.  Was the
importance sampling successful?  Check to make sure that eiread's
\texttt{resamp} is not much larger than 20 or so (it is not fatal if
it is larger, but its worth checking).

\paragraph{What statistical issues should I check?}
It is essential to verify that the model fits the data.  First, look
at eigraph's \texttt{fit}: for the X by T graph on the left, verify
that the $E(T_i|X_i)$ line passes through the middle of the points,
and the 80\% confidence intervals capture around 80\% of the points,
vertically for each value of $X$ on the horizontal axis.  For the
tomography plot of the right of the \texttt{fit} graph, verify that
the contours capture the place from where the ``emissions'' come from
(roughly speaking, where the lines are crossing).  Then check
eigraph's \texttt{tomogS} and verify that the maximum likelihood
contours (on the top left, the same as the tomography plot in the
\texttt{fit} graph) and mean posterior contours (on the top right)
both fit the data in roughly the same way.  If there are problems
here, see the next question in this FAQ.

Other issues to check are whether there outliers or multiple modes. Is
there aggregation bias?  Check the results at the aggregate level
(eigraph's \texttt{post}; eiread's \texttt{Paggs}) and the precinct
level (e.g., eigraph's \texttt{beta} or eiread's \texttt{betaB} and
\texttt{betaW}); are these consistent with your qualitative knowledge?
Verify that the relationship between $\beta_i^b$ and $X_i$ and between
$\beta_i^w$ and $X_i$ are consistent with your understanding of your
substantive problem (see eigraph's \texttt{boundX}).  Is there survey,
qualitative, or other external information you could have used but
didn't?  Chapter 16 provides a complete checklist that should be used
for every serious application.

\paragraph{What global values should I set?  Which EI model should I run?}
EI will enable you to run any of a large set of related models.  No
one model from this set or any other will work all the time.  The
program defaults are chosen for speed and simplicity, not for what
will be most reasonable empirically.  For the latter, I would read
Section 9.2.3 (especially Figure 9.6) and consider globals like these:
\begin{verbatim}
  _Eeta=3;
  _EalphaB=0~0.25;
  _EalphaW=0~0.25;
\end{verbatim}

\paragraph{What do I do if my model doesn't fit the data?}  There are
several approaches, depending on the problem.  Don't miss the first
item.
\begin{enumerate}
\item The most important technical problem with lack of fit occurs
  because of numerical issues related to imprecision in the lnCDFbvn
  function used in the likelihood function (the volume above the unit
  square under the bivariate normal, Equation 7.2, p.134).  This
  imprecision can induce artificial local maxima in the likelihood
  function, leading to convergence at the wrong parameter values.  It
  can also create artificial maxima higher than the correct global
  maximum.  These problems occur most often when the maximization
  routine is looking far from optimal value.  A good way to fix both
  problems is to give the program starting values in the region of the
  right answer (set \texttt{\_Estval}), and constrain the search to a
  region that includes these values (set \texttt{\_Ebounds}).  One way
  to find better starting values, it is to pick the parameter values
  by looking at a tomography plot, as if we were on the truncated
  scale.  That is, identify the region from where the ``emissions''
  are coming (roughly, where the tomography lines are crossing) and
  record the coordinates for betaB and betaW, the width of the
  emissions at the central point, and then a likely correlation (or
  0).  Then transform these onto the scale of estimation with
  Equations 7.4 (p.136), or use \texttt{eireparinv} to do this
  automatically.  Then set \texttt{\_Ebounds} to regions around those
  starting values --- not too narrow because you determine the answer
  (a concern if the parameters turn out to be maximized at boundary
  values), and not too wide because you may run into numerical
  problems.  If this does not work, it will be helpful to narrow
  \texttt{\_Ebounds}.  The new grid search procedure is especially
  helpful here (set \texttt{\_Estval} to 0 or $>5$).  If it is an
  especially difficult problem, you may need to change the tolerance
  of the lncdfbvn function, with \texttt{\_EcdfTol}.

\item If you have very small values of $T$, see the FAQ question
  below.

\item One common problem is coding errors, or small precincts, for
  which $T_i$ is very close to 0 or 1 (look at the corners of
  eigraph's \texttt{tomog} for a count of these); if these values are
  outliers, they can have a disproportionate effect on the likelihood
  results, despite the fact that in many applications $T$ only gets to
  the corners when there are data errors.  To delete them from the
  estimation stage but include them in the simulation stage, you could
  set
  \begin{center}
    \texttt{\_Eselect=(t.$>$0.001).and(t.$<$0.999);}
  \end{center}
  or perhaps an even narrower range would be wise.  You could also
  delete them from the data set to skip both stages.

\item Do you suspect extensive aggregation bias?  Perhaps you should
  try the globals \texttt{\_Eeta=3;} \texttt{\_EalphaB=0$\sim$0.1;}
  and \texttt{\_EalphaW=0$\sim$0.1;} to start (see Chapter 9).

\item Does eigraph's \texttt{tomog} suggest multiple modes?  Consider
  specifying \texttt{Zb} or \texttt{Zw} coded to pick up the modes.

\item Is eiread's \texttt{resamp} much larger than 20?  If so, you
  might try using a $t$ distribution as the first approxiation for
  importance sampling by setting \texttt{\_EisT} to 3 or higher, or
  adjusting \texttt{\_EisFac} (usually downwards or set to $-1$,
  especially if \texttt{tomog} fits but \texttt{tomogP} does not) or
  \texttt{\_Eisn} (always upwards) (e.g., you might try
  \texttt{\_EisFac=1} or \texttt{\_EisFac=$-1$}).
  
\item If eigraph's \texttt{estsims} does not look approximately like
  \texttt{tomogp}, or if the graphs in \texttt{post} are bimodal, you
  need to do something.  You may try a different method of computing
  the variance matrix.  You could also narrow the variance of the
  priors on $\sigma_b$, $\sigma_w$, and $\rho$ by setting
  \texttt{\_Esigma} and \texttt{\_Erho}.  Or more simply, you could
  use the maximum likelhood solution and set \texttt{\_EisFac=-2;}.

\item If the relationship between $X_i$ and $\beta_i^b$ or $\beta_i^w$
  does not correspond to your substantive knowledge of the problem,
  consider setting \texttt{\_Eeta=3} and adding a prior on $\alpha^b$
  and $\alpha^w$ (with \texttt{\_EalphaB} and \texttt{\_EalphaW}).

\item If you have additional information in the form of survey or
  qualitative evidence, you could change the priors, add covariates in
  \texttt{Zb} or \texttt{Zw}, or divide the data set.
\end{enumerate}
See Chapters 9 and 16 for more detailed suggestions.

\paragraph{What do I do if the Hessian is not positive definite?}  If
you have covariates, make sure you didn't include a constant term or
variables that are very highly collinear.  Whether you included
covariates or not, what often helps is to narrow the variance of the
prior on $\sigma_b$, $\sigma_w$, and $\rho$ by setting
\texttt{\_Esigma} and \texttt{\_Erho} to smaller numbers than their
defaults.  Following this action is also sometimes helpful in getting
the model to fit appropriately, if \EI\ took several tries to compute
a valid variance matrix.  You should be careful, of course, that a
non-positive definite Hessian doesn't indicate an impossible
estimation problem.  Also be careful whenever \texttt{eiread}'s
\texttt{ghactual} is other than \{0 1\} (if it is, see the global
\texttt{\_EI\_vc}).

\paragraph{What do I need to know about studying voter transition
  rates?}  The model applies as usual, but note how the quantities of
interest are parameterized.  In the running example, it is the
fraction of blacks who vote, not the fraction of people who are both
black and who vote.  If the latter parameterization seems more natural
for voter transition applications (e.g., the fraction of people who
vote Democratic at time 1 and time 2), you need to translate the
output of \EI.

\paragraph{Why do I get slightly different results every time?}
\EI\ uses random simulation to approximate various statistics.  This
entails no compromise, and it makes computations easy that would
otherwise be impossible.  The only issue is ensuring that you have
sufficient random draws for the level of precision (number of
significant digits) you desire.  For example, if you need 2 digits to
the right of the decimal point for a table you plan to present, you
can check that the number of simulations is set high enough by running
the analysis twice and verifying that the first two digits do not
change.  If they do change, increase \texttt{\_Esims} and try again
until the two runs give the same answer to two digits.  (If you really
want exactly the same numbers for each replication, set the Gauss
random number seed with the Gauss command \texttt{rndseed} before
running \EI, although this would only be useful for testing.)  See
Chapter 8.

\paragraph{How do I get this program to run faster?}  A
variety of \EI\ options can significantly increase the speed of the
program.  In the order in which we suggest you try them, these are,
\begin{enumerate}
\item If you have many observations, you can use a random subset in
  the first stage (by setting \texttt{\_EselRnd} to the fraction of
  observations to include).  Setting this global still results in all
  observations still being included in the second stage, the result
  being that estimates will still be available for all observations.
  If a large $n$ is the issue, then this is the best way to speed
  estimation without much cost.
  
\item You can use the asymptotic normal approximation, and eliminate
  the importance sampling refinement, by setting
  \texttt{\_EisFac}$=-1$ or to go even faster use
  \texttt{\_EisFac}$=-2$ to exclude estimation uncertainty.

\item Choose good starting values (by setting \texttt{\_Estval}).  For
  example, if you have many similar analyses to run, do a typical
  analysis and use those numbers as starting values for the others.
  If you have a guess as to the values of $\phi$ on the untruncated
  scale, you can use the following command (documented only in
  \texttt{eirepar.src}) to set the starting values: e.g.,
  \texttt{\_Estval=eireparinv(0|0|-1|.4|.1)}.

\item Buying enough RAM so you do not need to take advantage of the
  virtual memory feature of Gauss is helpful.  Or, just buy a really
  fast computer!

\item If you are willing to live with lower precision, you can draw
  fewer simulations by setting \texttt{\_Esims} to a number smaller
  than the default (100).

\item You can set \texttt{\_EdirTol} to a larger number.  The default
  is 0.0001; you can try 0.001 or larger.  You could also reduce the
  maximum number of iterations with \texttt{\_EmaxIter}.

\item If you are running many analyses, run one analysis to make sure
  that the globals are properly defined and then set
  \texttt{\_Echeck=0} to turn off global checking.  This will only
  save a small bit of time and so is only helpful if you are running
  many analyses.  (If you turn off checking, and you define a global
  incorrectly, \EI\ will not give a pretty error message.)

\item For EI2, set \texttt{\_ei2\_m}$=-1$.
\end{enumerate}

\paragraph{Why is the program slower when using covariates?}
\EI\ needs to evaluate the cumulative bivariate normal distribution,
which is computationally intensive.  If no covariates are used, only
one set of computations need to be done for the entire data set.  If
any covariates are used, a new computation must be computed for every
observation.

\paragraph{What are the parameter names printed during maximization?}
Without covariates, \texttt{Zb0} and \texttt{Zw0} are estimates of
$\phi_1$ and $\phi_2$ respectively.  With covariates, \texttt{Zb1},
\texttt{Zb2},\ldots, refer to estimates of the elements of the vector
$\alpha^b$ and \texttt{Zw1}, \texttt{Zw2},\ldots, are estimates of the
elements of $\alpha^w$.  The symbols \texttt{sigB}, \texttt{sigW}, and
\texttt{rho} refer to estimates of $\phi_3$, $\phi_4$, and $\phi_5$.
Finally, the global \texttt{\_Eeta} allows $\alpha^b$ and $\alpha^w$
to be set to fixed values, which are \texttt{etaB} and \texttt{etaW}
(the default for is zero for each).

\paragraph{What should I look for during the iterations?} The iterations
will converge when direction vector (which, roughly speaking, is the
number of digits of precision in locating the maximum) gets close to
all zeros.  (If you prefer knowing when it will end, choose the grid
search procedure by setting \texttt{\_Estval=0}).

\paragraph{What about data with very small or large values of $T$?}
Data with many small or large $T_i$ cause no problems with the method
in theory.  Moreover, data like these produce estimates with very
narrow bounds (since the tomography lines cut off the top right or
bottom left corner of the plot).  This is good news for estimation,
although if the difference between, say, $\beta_i^b=0.01$ and
$\beta_i^b=0.005$ is substantively large (as for mortality rates),
what might otherwise be considered ``narrow'' bounds (such as
[0,0.05]) could in some cases still be too wide for particular
substantive purposes, so be careful about interpretation.  To read the
graphs, it is helpful to zoom in on the relevant regions, by setting
globals such as \texttt{\_eigraph\_Xlo}, \texttt{\_eigraph\_BBhi},
etc.  Be sure to check and probably reduce the global
\texttt{\_EnumTol}, since the default value (0.0001) would define some
data sets as unanimous ($T_i=0,1$) for too many observations.  Beware
also of other numerical problems with data like these (by verifying
the fit of the contours to the data with eigraph's \texttt{tomogS} and
\texttt{fit}), since the contours must be fit to a very small area of
the tomography plot and so calculating the cdf of the truncated
bivariate normal can be imprecise.  It is often necessary to use
better starting values (see \texttt{\_Estval}) and search constraints
(\texttt{\_Ebounds}) than the defaults.  The grid search procedure can
also be helpful (\texttt{\_Estval}).  Since odds are you want to
distinguish between very small differences in the quantities of
interest, you will probably also wish to increase the number of
simulations substantially (\texttt{\_Esims}).

\paragraph{How do I make inferences in larger tables?}
For $2\times C$ tables, see the \texttt{ei2} in the reference section
of this manual (see Section 8.4).  For general $R\times C$ tables, you
can partition the data into a series of $2\times 2$ (or $2\times C$)
tables, each of which this program can analyze.  The results can then
be combined to produce estimates of your larger table.  See Section
15.1 for details and the rest of Chapter 15 for more general
procedures.  (We're working on an easier approach for $R\times C$
tables.)

\paragraph{What do I do if I have computational problems with EI2?}
Try \texttt{\_Ei2\_m=-1}.  This may make the standard errors slightly
too small, but it will make the procedure numerically more reliable
and about four times faster.  This is the procedure described at the
start of Section 8.4.1 (pp 151--2).

\paragraph{Why does eiread report some precinct-level estimates
  as missing values?}  When $X_i=0$ (in the running example, this
indicates no blacks in the precinct), $\beta_i^b$ (the fraction of
blacks who vote) is undefined, and its estimate is set at missing.
Similarly, if $X_i=1$, $\beta_i^w$ is set to missing.  In
\texttt{ei2}, if $T_i=0$ from the previous stage then $V_i$ should be
undefined and both $\beta_i^b$ and $\beta_i^w$ should be missing
(although it is best to remove observations for which $T_i=0$ from the
\texttt{ei2} run.)

\paragraph{How do I test whether $B^b-B^w$=0?} You can get simulations
of this difference (racially polarized voter turnout in the running
example) by doing: \texttt{aggs=eiread(dbuf,"aggs");} and
\texttt{diff=aggs[.,1]-aggs[.,2];}.  Then you can use \texttt{diff} as
you might normally use simulations of $\beta_i^b$.  For example, a
point estimate of the aggregate difference is \texttt{meanc(diff)}, a
standard error is \texttt{stdc(diff)}.  You can also see the full
posterior distribution by plotting the histogram with \texttt{call
  hist(diff,7)} or a kernel density estimate with \texttt{call
  dens(diff)}.

\paragraph{How do I make inferences about aggregated continuous variables?}
If the individual-level variables corresponding to $X_i$ and/or $T_i$
are continuous but restricted to the the unit interval (such as the
fraction correct on a test) or rescaled to this interval, then this
program can be used without any changes.  Only the interpretation of
the estimates change.  See Section 14.3.

\paragraph{How many observations do I need?}  The answer
depends on the problem.  For example, one observation with $X_1=0$
will give you a correct answer for $\beta_1^w$ and a standard error of
zero.  On the other hand, two million observations with $X_i=0$, won't
be enough to estimate $B^b$.  Another way to think about this question
is that the basic model has five parameters and (like a regression
model with five parameters) you would probably want at least 30--50
observations or so.  Problems for which the parameters are more highly
correlated (like regression problems with high collinearity among the
explanatory variables) will require additional observations to achieve
the same level of confidence.  (That is, check the tomography plot to
ascertain what kind of problem you are working with.)  You will also
want additional observations if your substantive problem demands
answers with more precision.  Of course, if you can selectively add
observations that are especially informative, then there may be great
power to be had from collecting just a bit more data.

\paragraph{Has this method been used in Court?}
A federal district court judge declared that \emph{A Solution to the
  Ecological Inference Problem} ``is the best method currently
available to measure racial bloc voting'' [i.e., whether blacks and
whites vote differently]
% and ``Dr. King the world's foremost
% expert in the statistical analysis of racial bloc voting.''
(\emph{Mallory, et al., v.\ State of Ohio, et al.}, USDC Southern
District Court of Ohio, case no.\ C-2-95-381.)  This decision was
upheld on appeal before the three judges of the Circuit Court, and
similar language was used to describe the method (in \emph{Mallory, et
  al.\ v.\ State of Ohio, et al.}, United States Court of Appeals,
Sixth Circuit, No.\ 97-4425, 4/13/99).  It was also accepted by the
court in \emph{Bill Jones vs.  Ca Democratic Party} and in a variety
of others.  In each case, EI was unchallenged by experts on the
opposing side.  Since EI starts with Goodman's regression, which had
previously been declared as the appropriate method by the Supreme
Court, and adds information known to be true (i.e., the bounds), EI
has been relatively uncontroversial in these situations.

\paragraph{What if I find a bug?}  First try to replicate the problem
with the current version of the program (available at
\hlink{\url{http://GKing.Harvard.Edu}}{http://gking.harvard.edu}).  If the
problem still exists, copy down \emph{exactly} what you see on the
screen when the program bombs, and email it along with any changes in
globals to
\hlink{\texttt{kbenoit@tcd.ie}}{mailto:kbenoit@tcd.ie} if
you think the error was in the \EzI\ front end or
\hlink{\texttt{King@Harvard.Edu}}{mailto:king@harvard.edu} if the
error appears to be in \EI.  Comments and suggestions for changes are
welcome.
\end{document}
