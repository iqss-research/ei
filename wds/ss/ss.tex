\documentclass[11pt,titlepage]{article}
\usepackage[reqno]{amsmath}
\usepackage{amssymb}
\usepackage{epsf}
\usepackage{url}

% === additional commands/packages/settings ===
\usepackage{graphicx,psfrag,natbib}
\usepackage{setspace}
\usepackage{vmargin}
\setpapersize{USletter}

% === dcolumn package ===
\usepackage{dcolumn}
\newcolumntype{.}{D{.}{.}{-1}}
\newcolumntype{d}[1]{D{.}{.}{#1}}

% === newcommands from sei.tex ===
\newcommand{\EI}{\ensuremath{{\mathfrak EI}}}
\newcommand{\Tiny}{\tiny}
\newcommand{\sump}{\sum_{i=1}^p}
\newcommand{\mean}{\frac{1}{p}\sump}
\newcommand{\ub}{\Dot{\beta}}
\newcommand{\ut}{\Dot{\theta}}
\newcommand{\bbeta}{{\mathfrak B}}
\newcommand{\btheta}{{\mathfrak T}}
\newcommand{\blambda}{{\mathfrak L}}
\newcommand{\bbetau}{\breve{\mathfrak B}}
\newcommand{\bV}{{\cal V}}
\newcommand{\sigmau}{\breve{\sigma}}
\newcommand{\Sigmau}{\breve{\Sigma}}
\newcommand{\rhou}{\breve{\rho}}
\newcommand{\psiu}{\breve{\psi}}
\newcommand{\Eu}{\breve{\text{E}}}
\newcommand{\Vu}{\breve{\text{V}}}
\newcommand{\Bb}{B^b}
\newcommand{\Bw}{B^w}
\newcommand{\NbD}{{N_i^{bD}}}
\newcommand{\NwD}{{N_i^{wD}}}
\newcommand{\NbR}{{N_i^{bR}}}
\newcommand{\NwR}{{N_i^{wR}}}
\newcommand{\NbN}{{N_i^{bN}}}
\newcommand{\NwN}{{N_i^{wN}}}
\newcommand{\tp}{{\mbox{\Tiny{+}}}}
\newcommand{\NpD}{{N_i^D}}
\newcommand{\NpR}{{N_i^{R}}}
\newcommand{\NpN}{{N_i^{N}}}
\newcommand{\Nbp}{{N_i^{b}}}
\newcommand{\Nwp}{{N_i^{w}}}
\newcommand{\Npp}{{N_i}}
\newcommand{\Nppp}{{N}}
\newcommand{\Nbpp}{{N^{b}}}
\newcommand{\Nwpp}{{N^{w}}}
\newcommand{\sumpN}{\sump\Npp}
\newcommand{\NsumpN}{\frac{1}{\Nppp}\sumpN}
\newcommand{\nbp}{{N_i^{bT}}}
\newcommand{\nwp}{{N_i^{wT}}}
\newcommand{\npp}{{N_i^{T}}}
\newcommand{\NbV}{{N_i^{bT}}}
\newcommand{\NwV}{{N_i^{wT}}}
\newcommand{\NpV}{{N_i^{T}}}
\newcommand{\xb}{\bar{X}}
\newcommand{\wb}{\bar{W}}
\newcommand{\tb}{\bar{T}}
\newcommand{\cx}{{\mathbb X}}
\newcommand{\cw}{{\mathbb W}}
\newcommand{\ct}{{\mathbb T}}
\newcommand{\cpij}{{\mathbb P}^*_{ij}}
\newcommand{\cpi}{{\mathbb P}^*_i}
\newcommand{\cwd}{{\dot{\mathbb W}}}
\newcommand{\ctd}{{\dot{\mathbb T}}}
\newcommand{\cxd}{{\dot{\mathbb X}}}
\newcommand{\cxb}{\bar{\mathbb X}}
\newcommand{\cz}{{\mathbb G}}
\newcommand{\ch}{{\mathbb H}}
\newcommand{\cm}{{\mathbb M}}
\newcommand{\E}{{\textbf{E}}}
\newcommand{\V}{{\textbf{V}}}
\newcommand{\C}{{\textbf{C}}}
\newcommand{\rE}{{\text{E}}}
\newcommand{\rV}{{\text{V}}}
\newcommand{\rC}{{\text{C}}}
\newcommand{\TN}{\text{TN}}
\newcommand{\BN}{\text{BN}}
\newcommand{\N}{\text{N}}
\renewcommand{\P}{\text{P}}
\newcommand{\D}{\textbf{D}}
\newcommand{\R}{\ensuremath{\textbf{R}}}
\newcommand{\Rr}{\ensuremath{\{\textbf{R}\diagdown r}\}}
\newcommand{\bC}{\ensuremath{\textbf{C}}}
\newcommand{\Cc}{\ensuremath{\{\textbf{C}\diagdown c}\}}
\newcommand{\one}{{\mathbf{1}}}
\newcommand{\Q}{\ensuremath{\overset{\vspace{3em}}{\text{\Huge ?}}}}
\newlength{\padsp}
\settowidth{\padsp}{$(\beta^w_i=\nwp/\Nwp)$}
\newcommand{\padbw}{\hspace*\padsp}
\newcommand{\bkappa}{\boldsymbol{\kappa}}

\newcommand{\bb}{\beta^{\text{bad}}}
\newcommand{\bg}{\beta^{\text{good}}}

\title{How to Analyze Second-Stage Ecological Regressions: Extensions
  to Herron and Shotts}

\author{Christopher Adolph\thanks{Ph.D. candidate, Department of
    Government, Harvard University. (Center for Basic Research in the
    Social Sciences, 34 Kirkland, Cambridge MA 02138;
    \texttt{chris.adolph.name}, \texttt{cadolph@Fas.Harvard.Edu}).}
\and %
Gary King\thanks{Professor of Government, Harvard University and
  Senior Science Advisor, World Health Organization (Center for Basic
  Research in the Social Sciences, 34 Kirkland Street, Harvard
  University, Cambridge MA 02138; \texttt{http://GKing.Harvard.Edu},
  \texttt{King@Harvard.Edu}, (617) 495-2027).}  }

\begin{document}
\maketitle
\begin{abstract}
  
\end{abstract}
We have taken the opportunity afforded us to comment on Herron and
Schotts' (hereinafter HS) article because of its potential to affect
the way a considerable body of practical research is conducted, and
because of HS's interesting, creative, and productive ideas.  HS's
paper is based on the suggestions in three paragraphs in King (1997:
289--90), along with the growing number of empirical papers that have
adopted some of the suggestions herein.  We thought it might be useful
to start with these paragraphs:
\begin{quotation}
  If a second stage analysis is conducted, least squares regression
  should probably not be used in most cases, even though it may not be
  particularly misleading.  The best first approach is usually to
  display a scatterplot of the explanatory variable (or variables)
  horizontally and (say) an estimate of $\beta_i^b$ or $\beta_i^w$
  vertically.  In many cases, this plot will be sufficient evidence to
  complete the second stage analysis.
  
  If it proves useful to have more of a formal statistical approach,
  and many of the actual values of $\beta_i^b$ fall near zero or one,
  then some method should be used that takes this into account.  The
  data could be transformed, via a logit or probit transformation, or
  a model like the one in Section 9.2.2 could be applied by replacing
  $X_i$ with the explanatory variables.  Whatever method is chosen,
  the researcher should be careful to include the fact that some
  estimates of $\beta_i^b$ are more uncertain than others.
    
  In practice, a weighted least squares linear regression may be
  sufficient in many applications, with weights based on the standard
  error of $\beta_i^b$ (or other quantity of interest).  Researchers
  should be careful in applying this simplified method here, and
  should verify its assumptions with scatterplots.  The problem is
  that, according to the model, the variance of $\beta_i^b$ over $i$
  is the sum of two quantities: its estimation variance, which varies
  over the observations (and is an output of the basic ecological
  inference model described in this book), and its fundamental
  variability not explained by the aggregate explanatory variables.
  That is, even if $\beta_i^b$ were known with certainty in every
  precinct, we would not expect the variability in it from precinct to
  precinct to be perfectly explained by any set of measured
  explanatory variables.  A reasonable approach can be kludged by
  first running a (homoskedastic) least squares regression and
  computing the variance of the residuals, which is a rough measure of
  the total variability of $\beta_i^b$ over precincts.  Then, for each
  observation, compute the variance, from which the weight will be
  determined, by subtracting the estimation variance from the total
  variance.  Finally, run a weighted least squares regression.  This
  is not as theoretically elegant a procedure as the more formal set
  up in Section 9.2.2, but it is simple, relatively robust, and
  probably complete enough to be of use in many applications.
\end{quotation}  

HS add to these paragraphs by pointing out that Bayesian estimates,
such as those from EI, regress to the mean.  This ``shrinkage''
property produces optimal estimates, in terms of mean square error.
Thus, we agree with HS that the best possible estimate of $\beta^b_i$,
under HS's assumptions, is that produced by EI.  But HS also make the
interesting and correct point that using Bayesian estimates with this
property as dependent variables in least squares regression can, under
some circumstances, produce biased coefficient estimates.  This
central point of HS's paper is important contribution and one that we
attempt to build on in this paper.  (HS's paper is model of good
science; we were able to replicate all of their simulations easily
from the information in their paper.)



3.  HS's suggested procedure is to provide a bias correction for the
slope of the second stage regression but to leave the constant term
unchanged.  The correction is based on linear approximations that draw
on classical linear regression theory.  The problems we describe below
with applying this procedure in the ecological inference situation are
due to the failure of the linearity assumption in modeling an
inherently nonlinear and bounded relationship, and the lack of a
correction in the constant term.

4. We find that correcting the slope but not the constant is
considerably worse than an unadjusted regression of $\hat\beta_i^b$ on
$Z$ and every other procedure discussed in HS's paper and the
literature we have examined.  The regression line from HS's partially
adjusted method often misses the point cloud of true $\beta_i^b$
points in most simulations by a wide margin.  It even misses the
observed points, $\hat\beta_i^b$.

5.  We extend HS's partial adjustment procedure, within their linear
regression framework, by developing an adjustment for the constant
term also.  Our fully adjusted (FA) second stage regression method
naturally extends HS's framework.  It also dominates HS's procedure,
and so we do not further consider HS's original procedure.

6.  We also compared our fully corrected procedure to unadjusted
linear regression and found the following four modal situations likely
to arise in practice. Since these situations are easily detectable
from the aggregate data, users will know which situation they are in
and can take the appropriate action.

(a) When ecological data have very wide bounds, EI (and any method of
ecological inference) will be highly model dependent and sensitive to
assumptions.  In many applications in this situation, no ecological
inference should be conducted unless one has good reason to believe
the model assumptions.  If one goes ahead anyway, then it is in this
almost exactly the same as the linear regression situation and the
correction works: it is in this situation (only) in which the full
adjustment corrects appropriately for attenuation bias and dominates
the unadjusted procedure.

(b) When the bounds are at least somewhat informative, we are in the
situation when we would be more likely to trust ecological inferences
using EI (or another method that takes into account the information in
the precinct-level bounds).  In this situation, we find that the
unadjusted procedure usually does as well or better than the fully
corrected procedure.

(c) When some observations have wide bounds and others have narrow
bounds, and $hat\beta_i^b$ is approximately a linear function of
$Z_i$, an unadjusted weighted least squares (WLS) regression will
often substantially improve estimates as compared to the unadjusted
regression or the fully corrected version of the HS procedure.  This
is contrary to HS's claims that WLS would not make a difference; what
they missed by applying a linear regression framework to problem with
bounds and nonlinearity is that the degree of attenuation is directly
related to the width of the bounds, and so the weights are correlated
with the attenuation bias and can be effective in adjusting for it.

(d) When the relationship is nonlinear, as is sometimes observably the
case because of the bounds, then any linear second stage procedure can
produce impossible results.  In this situation, a scatterplot or an
appropriate nonlinear procedure would be better.  The fully adjusted
procedure in this situation often produces more out of bounds
predictions than the unadjusted procedure.  (In this case, WLS is also
inappropriate, both because the LS assumption of linearity does not
hold, and because $\hat\beta_i^b$'s at the extremes have standard
errors of zero or nearly so.  Giving extra weight to these
observations tends to bias the estimate of the slope downwards.)

7.  At least in some applications, the HS and other corrections
discussed here may make little difference.  We replicate Burden and
Kimball's study of split-ticket voting, and find that none of the
proposed corrections affected the substantive conclusions of the
original study.

8.  The Monte Carlo procedure used to generate results by HS
artificially induces immense attenuation bias due to random
measurement error in $Z$, quite apart from any attenuation due to
Bayesian shrinkage error in the EI estimate.  This can be seen because
the setup of the model from which they draw their simulations implies
that $\gamma=1$ and yet the estimates from their simulations of the
true $\beta_i^b$ (i.e., without any attenuation bias in the dependent
variable at all) on $Z$ indicates that the estimate is drastically
biased. They do not present the estimates, but in our replication we
found that $\hat\gamma\approx 0.05$.  In contrast, our simulation
procedure tends to recover $\hat\gamma\approx\gamma$ to the extent
that the relationship between $\beta_i^b$ and $Z_i$ is linear.

Proof that their simulation method assumes that alpha=0 and gamma_R=1:

E(betab)=E(alpha+gamma_R*Z+nu)
        =alpha+gamma_R*E(Z)
        =alpha+gamma_R*E(betab+tau)
        =alpha+gamma_R*betab,

and so therefore it must be the case that alpha=0 and gamma=1 when we
regress the true betab on Z.

9.  As indicated in King (1997) and quoted above ``The best first
approach is usually to display a scatterplot of the explanatory
variable (or variables) horizontally and (say) an estimate of
$\beta_i^b$ or $\beta_i^w$ vertically.  In many cases, this plot will
be sufficient evidence to complete the second stage analysis.''
Indeed \textbf{show is the data!} is a good general motto of any
statistical analysis, especially those with complex nonlinear and
bounded variables such as those resulting from making ecological
inferences.  In fact, the various adjusted and unadjusted linear
regressions might be best thought of as descriptive devices that are
essentially convenient summaries of this kind of scatterplot.

If a more formal statistical approach seems desirable, then a good
method must go beyond classical linear regression theory.  It must
take into account (a) the nonlinear nature of the problem, (b) the
bounded nature of the second stage dependent variable with the width
of the bounds varying over observations, (c) the heteroskedasticity,
and (d) the effect of any possible logical inconsistency of the first
and second stages of the analysis (technically known as
``uncongeniality''; Meng, 1994).  At present, the only model that has
been proposed with all these properties is the extended EI model that
allows covariates to be included as part of the EI estimation
procedure.  HS are correct that this extended model sometimes makes
estimation only weakly identified, but that is only when $X$ is
included among the covariates.  In other cases, especially when $Z$ is
unrelated to $X$, the model will often be strongly identified and so
can be used if desired.  Furthermore, Imai and King (2002) demonstrate
how to compute first differences and other quantities from the
extended EI model, and they report on software extensions to make this
possible.

\end{document}

Meng, X.L.\ 1994a. ``Multiple-imputation Inferences with Uncongenial
Sources of Input,'' \emph{Statistical Science}, 9, 4: 538--573.

