\documentclass[12pt]{scrartcl} 
\usepackage[left=1in,right=1in,top=1in,bottom=1in]{geometry} 
\usepackage{booktabs}
\usepackage{dcolumn}
\usepackage{setspace}
\usepackage{graphicx}
\usepackage{fullpage}
\usepackage{paralist}
\usepackage{natbib}
\usepackage{appendix}
\usepackage{amsmath}
\usepackage{amsfonts}
\usepackage{amssymb}
\usepackage{color}
\usepackage{times}
\usepackage{topcapt}
\usepackage{booktabs}
\usepackage{rotating}
\usepackage{sectsty}
\usepackage{morefloats}
\usepackage{url}
\sectionfont{\large}
\subsectionfont{\normalsize}
\subsubsectionfont{\small}

\title{An Analysis of the Arizona Independent Redistricting Commission
  Congressional District Map}
%
\author{Gary King\thanks{Alfred J.
    Weatherhead III University Professor, Institute for Quantitative
    Social Science, 1737 Cambridge Street, Harvard University,
    Cambridge MA 02138; \url{http://GKing.harvard.edu},
    king@harvard.edu, (617) 500-7570.}  \and Benjamin
  Schneer\thanks{Ph.D.\ student, Institute for Quantitative Social
    Science, 1737 Cambridge Street, Harvard University, Cambridge MA
    02138, bschneer@fas.harvard.edu} }

%\date{January 25, 2011}

\begin{document}
\maketitle

\doublespacing

\section{Introduction}

We have been retained by the Arizona Independent Redistricting
Commission to analyze data from the congressional district maps drawn
for the 2011-2012 redistricting cycle. In this report, we estimate the
extent of racially polarized voting, determine the electability of the
minority groups' candidates of choice, and (through a new method we
introduce in this project) evaluate whether stronger districts could
have been drawn.  We perform this analysis for both the benchmark map
and proposed map, which allows for a judgment of whether the proposed
map has a retrogressive effect.  Our analysis is based on the
quantitative information; much supplementary qualitative information
will appear in other documents submitted by the Commission.

% A summary of our findings is as follows:

% \begin{itemize}
% \item Finding One
% \item Finding Two
% \item Finding Three
% \end{itemize}

\subsection{Methodological Approach}

The goal of the present analysis is to provide evidence about the
voting behavior of different ethnic groups. Direct evidence of this
behavior is unavailable because of the secret ballot. Thus, we use
modern methods of ecological inference to estimate (rather than
determine) individual voting behavior from publicly available
VTD-level (i.e., precinct) votes for each of the candidates from
aggregated election results as well as population figures from the US
Census.  We summarize here the methods of ecological inference we use.

In 1953, two methods of ecological inference were introduced --- the
method of bounds \citep{duncan1953} and ecological regression
\citep{goodman1953}. A special case of the method of bounds is known
as ``homogeneous precinct analysis,'' which had been used in many
court cases: this approach seeks out ethnically homogeneous precincts
(100\% black or 100\% white, or 100\% Hispanic) because for those
precincts we know for certain the voting behavior of one ethnic group.
The assumption behind this method is that the voting behavior observed
in homogeneous precincts is identical to that in other areas. The
advantage of this method is that it yields completely certain
information about some subgroups of voters in some areas; the
disadvantage is that the relatively few who live in homogeneous
precincts may turnout in different numbers and vote for different
candidates in starkly different ways than the vast majority of the
population who live in (at least partially) heterogeneous areas.  The
method of bounds is more general than homogeneous precinct analysis
because the former can also provide some information about
heterogenous precincts; it does this in the form of ``bounds'' or
ranges into which the fraction of a minority group voting for a
particular candidate must fall.

The second method of ecological inference -- ecological regression --
ignores the information revealed by the method of bounds and its
special case of homogeneous precinct analysis. Instead, ecological
regressions gathers hints from statistical information across all
precincts. For example, if we find that in areas with more African
Americans that more votes are cast for the Democrats, then we may be
willing to infer that it is the African Americans who are voting for
the Democrats. The advantage of this approach is that it uses some
information from all precincts. The disadvantage is that the
information can be highly misleading: For example, also consistent
with the same evidence would be that the (liberal) whites who live in
areas with high African Americans concentrations are the ones who are
producing more votes for the Democrats. In fact, as an indication of
the common problems with this method, ecological regression, unlike
the method of bounds, regularly gives impossible answers -- such as
the percent of Hispanics voting for the Democrats of 160\% or $-54$\%.

The method of bounds (or homogeneous precinct analysis) and ecological
regression dominated the academic literature and courtroom expert
testimony from 1953 until 1997 when \citeauthor{king1997}'s EI
approach was introduced. This approach was the first to combine the
deterministic information from the method of bounds with the
statistical information from ecological regression into a single set
of estimates. Thus, it uses the statistical information from all
precincts, the certain information from homogeneous precincts, and
other deterministic information known for certain from other precincts
(given in the form of ranges of estimates).  Impossible estimates are
never produced by this methodology, and all information from all
precincts are used in the analysis. Like any indirect method of
revealing information that the secret ballot hides, EI is also
uncertain to a degree, but it uses more information than any other
previous approach. 

Since EI was introduced, a variety of other methods have been
developed in the academic literature, virtually all of which follow
King's practice of including deterministic and statistical information
in the same model.  For example, \citet{king2001} extends King's
method to arbitrarily large numbers of ethnic groups and candidates;
we use this method in our work as well as King's original method. Some
of these other methods have been collected in the edited volume by
\citet{king2004}.

In practice, when we have data we can use to validate the methods,
ecological regression and homogeneous precinct analysis tend to be
fairly inaccurate in many situatoins. Studies have shown that
uncertainty remains with King's and other subsequent methods, but the
estimates are normally superior. Differences among the new methods
that include both deterministic and statistical information are, in
comparison, relatively minor.

In this case, we use these newer methods to give estimates of the
proportion of each ethnic group that votes and, among those voting,
the proportion who vote for each of the candidates. We also use King's
``tomography plots'' and other diagnostic methods that help us discern
when adjustments in the methods need to be made, and how much
uncertainty remains.  

We also use the results from the ecological inference analysis here to
measure what we call \emph{district performance}, which is the percent
of the voting age population which a given ethnic group needs in order
to receive a specified share of the vote --- in this case we use 50 or
55 percent as the specified thresholds. We do this under the
assumption that voters in a racial group in the district will have the
same voting behavior as voters in a racial group from adjacent
districts.  This directly measures what would happen if the district
lines were redrawn in order to move minority voters into (or out of) a
district of interest.

\subsection{Measuring Voting Strength}

To measure voting strength in the new districts, we follow the
venerable practice of breaking down votes from available statewide
races.  In particular, we examined elections in the state of Arizona
from 2004 through 2010. These include the 2004 US Presidential
Election, 2006 Arizona Secretary of State Election, 2008 US
Presidential Election, 2010 Arizona Mine Inspector Election and 2010
Arizona Secretary of State Election.  All of these are state or
national elections (known in voting rights parlance as ``exogenous''
elections); changes in district lines are unlikely to influence the
vote choice of individual voters in these elections. Minority
candidates ran for office in four of these five elections.
Specifically, Hispanic candidates served as the Democratic candidate
in the 2006 Arizona Secretary of State Election and 2010 Arizona Mine
Inspector Election; an African-American candidate served as the
Democratic candidate in the 2008 US Presidential Election; a Native
American was the Democratic candidate in the 2010 Secretary of State
Election. The 2004 US Presidential Election did not include a minority
candidate; however, as we show below, the Democratic Presidential
candidate appears to be the clear candidate of choice for minority
voters in most districts.  

Below, we present results for each election separately, as well as an
average.  As would be expected based on prior history of electoral
returns in Arizona, and in other states, the results and our
conclusions are fairly similar across these different statewide races.

\section{Analysis of Congressional Districts}

We used our approach to ecological inference to determine the extent
of racially polarized voting in Arizona as well as the electability of
each district's candidate of choice among minority voters. For each
congressional district in the existing (i.e., ``benchmark'') and
proposed map, we estimated the turnout and each minority group and the
proportion of each minority group's vote for each candidate. 

To illustrate our approach, we describe the analysis for the vote
totals for a single election occurring in a single congressional
district.  Thus, Table~\ref{smine_cvap_cd_3_ex} displays estimates of
how each racial group in the proposed Congressional District 3 voted
in the 2010 election for Arizona Mine Inspector. Note that in each
table the total population in the last column and the total votes in
the last row are {\it observed} while the internal cells are {\it
  estimated} by our methodology. The bottom row labeled Total Pop
gives the population vote totals computed by aggregating over all VTDs
in the district.  Similarly, the column labeled Total CVAP lists the
share of citizen voting age population comprised by each racial group
in the district.  Again, this is calculated by simply aggregating over
all VTDs in the district.

\begin{table}[ht]
\begin{center}
\caption{Mine Inspector 2010 CD 3 (Proposed)}
\label{smine_cvap_cd_3_ex}
\begin{tabular}{lccccc}
  \hline
Racial Group & Turnout & D Vote & R Vote & S Vote & Total CVAP\\ 
  \hline
White & 0.36 & 0.35 & 0.65 & 0.00 & 0.43 \\ 
  Hispanic & 0.30 & 0.90 & 0.10 & 0.00 & 0.44 \\ 
  Native American & 0.22 & 0.90 & 0.09 & 0.01 & 0.04 \\ 
  Black & 0.13 & 0.50 & 0.48 & 0.02 & 0.05 \\ 
  Other & 0.21 & 0.55 & 0.44 & 0.01 & 0.05 \\ 
  Total Pop & 0.31 & 0.61 & 0.39 & 0.00 &  \\ 
   \hline
\end{tabular}
\end{center}
\end{table}

The internal cells in the table contain estimates from ecological
inference. For example, we estimate that 90\% (0.90 in the table) of
Hispanic citizens in CD 3 voted for the Democratic Candidate and 10\%
for the Republican candidate (and 0\% for any third party candidates).
Furthermore, we know that Hispanics made up 44\% of the district's
total population (CVAP) and we estimate that they turned out for the
election at a rate of 30\%.  Looking at the district as a whole, we
observe that the Democratic candidate received 61\% of the vote. Thus,
on the basis of the quantitative information from the 2010 Mine
Inspector election in the proposed CD 3, the Democratic candidate
seems to be the candidate of choice for Hispanic voters and this
candidate comfortably carried the district despite the existence of
some level of polarized voting --- that is, lower support among white
voters who we estimate voted for the Democrat at a rate of only 35\%
compared to the Hispanics 90\%.

This example illustrates the method of analysis for a single race in a
single district. We conducted this form of analysis for each district
in each election of interest. In the sections that follow, we use our
estimates for each district and election to assess the level of racial
polarization and ability to elect the candidate of choice.  The
results presented here use the Citizen Voting Age Population (CVAP)
totals. We present the same set of estimates using Voting Age
Population (VAP) totals in the digital appendix.  We have included
tables illustrating the estimates for each district and each race in
the appendix of this report.

\subsection{Sources of Uncertainty}

There are several sources of uncertainty in these estimates including
estimation uncertainty, fundamental uncertainty, and model dependence.
\emph{Estimation uncertainty} occurs because we have a limited set of
observations (precincts) to use in estimating how each racial group
votes.  \emph{Fundamental uncertainty} is the idea that even with a
large amount of data there is some degree of randomness in the
electoral process. A candidate runs an especially good campaign; the
weather makes it difficult for one candidate's supporters to get to
the polls; an outside group runs an effective set of attach ads; A
scandle hits one of the candidates; this is the stuff of politics and
it is modeled in a well-known way as fundamental variability.
Finally, \emph{model dependence} is the degree to which changes in
modeling assumptions affect our estimates.  Models are required in
ecological inference because some information is destroyed in the
process of aggregation and kept from the analyst due to the secret
ballot.

In different types of statistical problems, each of these sources of
uncertainty can play difference roles and to differing degrees.
Ecological inferences of course have all three sources of uncertainty,
but as it happens, it is the model dependence caused by having to
estimate voter preferences in the presence of the secret ballot that
accounts for the overwheming fraction of overall uncertainty.  Model
dependence dwarfs the other two components. Thus, we tune our methods
of assesssing uncertainty to focusing on model dependence.  

To convey the degree of uncertainty in the estimates of turnout and
vote share, we use ``tomography plots'' to accompany the tables.
These plots give all information in the data without making any
statistical modeling assumptions, as well as summarizing the available
statistical information in the data. This enables us to evaluate
directly how much information is available in the data and how much
is imposed by the statistical model.

One tomography plot is needed for each estimate in the corresponding
table. For one example, the tomography plot below analyzes the
information and uncertainty in the data and our analyses with respect
to the percent of Whites (horizontally) and non-Whites (vertically)
who vote for the Hispanic candidate in District 3, in each precinct.
If this information were known for a precinct, it would appear in the
plot as a single dot, and the set of precincts as a set of dots.
Because of the secret ballot, we cannot know the exact point with
precision; what the plot shows is that the information hidden from us
by the secret ballot is directly quantifiable: it turns each
precinct's dot into a line.  We can think of the dot as being smeared
into a line. That smearing represents a loss of information, but much
information is retained (and that is uniquely captured by our methods
of ecological inference).

\begin{figure}[htb]
\begin{centering}
\includegraphics[scale=.5]{pl_whitevote_h_3.png}
\caption{This tomography plot displays the white vote for the Hispanic Candidate in 2010 Mine Inspector Election}
\end{centering}
\label{tomog}
\end{figure}

For example, consider the bold green line in the plot. We know, based
on the observed data, that the point in this plot (representing where
the White and non-White vote share is) for this precinct must be some
point on the line, but we do not know exactly where on the line this
point falls. For this line, we know that the fraction of Whites that
vote for the Hispanic candidate must fall somewhere between 0.16 (16
percent) and 0.23 (23 percent). We get these numbers by projecting the
line downwards to the horizontal axis. If instead we project the line
to the left (vertical) axis, we can see that, for this particular
precinct, the range of possible values for the percent of non-Whites
voting for the Hispanic candidate could be anywhere from 0\% to 100\%.
That estimate is better than the method of ecological regression,
which often gives answers outside that interval, but still we can see
that this precinct is informative with respect only to Whites, not
non-Whites. In this way, each line captures exactly what we do and do
not know about the voting behavior in each precinct. Our statistical
method uses all this available information. Lines that are relatively
steep in this particular tomography plot, convey a lot of information
about the percent of Whites that vote for the Hispanic candidate.
Lines that are relatively flat convey a great deal of information
about the percent of non-Whites voting for the Hispanic candidate.
Lines that cut off the top right or bottom left corner of the plot are
informative about both quantities.

The tomography plots reflect what we know about the racial composition
of each precinct. If a precinct contains more than 65\% of a
particular racial group (which we use as an arbitrary cutoff for
graphical clarity), then the line on the tomography plot that
corresponds to that precinct is color coded to represent the majority
group (see the legend in the plot). If no groups comprise 65\% or more
of the precinct, then no color code is assigned.  Finally, the
tomography plots also reflect the results of the ecological inference
statistical estimation. The point estimate (i.e., the exact point on
the line that we estimate as the vote share for each group) as well as
the confidence intervals are colored yellow. Taken together, these
describe the overall estimate of how racial groups voted in the
district as a whole. For example, in the tomography plot for District
3 we estimate that white voters voted for the Hispanic candidate at a
substantially lower rate than did their non-white counterparts in the
district.  The point of the tomography plots is to convey the overall
uncertainty and certainty in the available data, and how our
statistical estimator uses that information to produce an estimate.
The uncertainty estimates here are far more informative and
information rich than sampling based confidence intervals or standard
errors.

For the sake of conciseness, tomography plots for each estimate are
included in a digital appendix.

\subsection{Racially Polarized Voting}
We assess the degree of racially polarized voting in both the benchmark and proposed districts using Tables~\ref{smine_cvap_cd_3} through~\ref{sos10_cvap_cd_7_benchmark} in the appendix, which displays the estimates for proposed CDs 3 and 7 as well as benchmark CDs 4 and 7. 

The proposed CDs 3 and 7 both demonstrate a high degree of polarized voting between whites and the ``main minority'' Hispanics. For both districts, the clear candidate of choice for the main minority across all five elections is the Democratic candidate; however, white voters do not appear to prefer the main minority candidate of choice. For example, as Table~\ref{smine_cvap_cd_7} illustrates, 90\% of Hispanic voters in CD 7 preferred the Democratic candidate in the 2010 Arizona Mine Inspector while there was an even split among white voters for this candidate. A similar trend holds across all other elections for both CD 3 and CD 7 in the proposed map.

\begin{figure}[!h]
\begin{centering}
\includegraphics[scale=.8]{cvap_cd_polarization.pdf}
\caption{CVAP Congressional Districts: Racial Polarized Voting}
\end{centering}
\label{cvap_cd_polarization}
\end{figure}

The benchmark map's congressional districts reveal a similar level of polarization. For example, as Table~\ref{smine_cvap_cd_7_benchmark} illustrates, 89\% of Hispanic voters favored the Democratic candidate in the benchmark CD 7 while only 33\% of whites voters voted for the minority candidate of choice.

Figure~\ref{cvap_cd_polarization} displays the level of racially polarized voting across all races and also provides an easy graphical means of comparing racial polarization between the proposed and benchmark districts.

In each graph, the horizontal axis represents the main minority (Hispanic) vote minus the majority (white) vote for each district. Thus, districts located farther to the right exhibit higher levels of polarized voting. The vertical axis tracks the share of the main minority (Hispanic) population in each district. The orange squares represent the location of the benchmark districts while the blue numbers represent the proposed districts.

As is evident from figure~\ref{cvap_cd_polarization}, the level of polarization from benchmark to proposed is strikingly similar across all races. Proposed CD 3 exhibits a level of polarization almost identical to benchmark CD 7. Proposed CD 7 exhibits a slight but insubstantial level of polarization over the benchmark CD 4. Furthermore, the proposed districts maintain a similar level of main minority population as compared to the benchmark.

\subsection{Electability of Candidate of Choice}
We assess the electability of the candidate of choice in both the benchmark and proposed districts using Tables~\ref{smine_cvap_cd_3} through~\ref{sos10_cvap_cd_7_benchmark} in the appendix, which displays the estimates for proposed congressional districts 3 and 7 as well as benchmark congressional districts 4 and 7.

Figure~\ref{cvap_cd} displays the results of our analysis; this figure is particularly crucial because it helps provide answers to the questions that are at the heart of whether the proposed districts are retrogressive or not. In particular, for each election the figure captures (1) the {\it estimated} level of support  among the main minority group in each congressional district candidate for the candidate of choice; (2) the {\it observed} electability of the candidate of choice in each congressional district; and, (3) the degree to which the proposed districts represent improvement or retrogression along these two dimensions.

\begin{figure}[!h]
\begin{centering}
\includegraphics[scale=.8]{cvap_cd.pdf}
\caption{CVAP Congressional Districts}
\end{centering}
\label{cvap_cd}
\end{figure}

For each graph in figure~\ref{cvap_cd}, the horizontal axis captures the estimated two-party vote share for the candidate of choice among the main minority population. The vertical axis represents the observed two-party vote share for the candidate of choice in the district. Thus, when a district falls in the top right quadrant of a graph it means that that (1) the main minority in the district prefers the candidate of choice and (2) the district as a whole would have elected the candidate of choice. 

As the figure illustrates, across all five individual elections (and when averaging across the five elections as in the graph in the bottom right corner) proposed CDs 3 and 7 fall in the top right quadrant -- these districts demonstrate a clear candidate of choice and maintain the ability to elect that candidate. 

Furthermore, the graphs reveal that for each election the proposed districts perform as well or better than the benchmark districts in terms of both the existence of a clear main minority candidate of choice as well as the ability to elect this candidate. Based on these criteria, the proposed districts do not appear to be retrogressive when compared to the benchmark districts.

%\subsection{The Potential for Stronger Districts: A Counterfactual Analysis}

%\begin{figure}[htb!]
%\begin{centering}
%\includegraphics[scale=.8]{cvap_cd_performance_ratio.pdf}
%\caption{CVAP Congressional Districts}
%\end{centering}
%\label{cvap_cd_performance_ratio}
%\end{figure}

\section{Analysis of Legislative Districts}

\subsection{Racially Polarized Voting}

\begin{figure}[ht]
\begin{centering}
\includegraphics[scale=.8]{cvap_ld_polarization.pdf}
\caption{CVAP Legislative Districts}
\end{centering}
\end{figure}

\subsection{Electability of Candidate of Choice}

\begin{figure}[ht]
\begin{centering}
\includegraphics[scale=.8]{cvap_ld.pdf}
\caption{CVAP Legislative Districts}
\end{centering}
\end{figure}

\subsection{The Potential for Stronger Districts: A Counterfactual Analysis}

\begin{figure}[ht]
\begin{centering}
\includegraphics[scale=.8]{cvap_ld_performance_ratio.pdf}
\caption{CVAP Legislative Districts}
\end{centering}
\end{figure}

\section{Conclusion}

\singlespacing

\newpage
\bibliographystyle{apsr}

\bibliography{az_bib}

\appendix
\renewcommand*\appendixpagename{\section*{Appendix}}
\appendixpage

\section{Ecological Inference Estimates: CVAP CD}
\subsection{Proposed Map}

% latex table generated in R 2.13.1 by xtable 1.6-0 package
% Thu Jan 26 11:22:20 2012
\begin{table}[htb]
\begin{center}
\caption{Mine Inspector 2010 CD 3 (Proposed)}
\label{smine_cvap_cd_3}
\begin{tabular}{lccccc}
  \hline
Racial Group & Turnout & D Vote & R Vote & S Vote & Total CVAP \\ 
  \hline
White & 0.36 & 0.35 & 0.65 & 0.00 & 0.43 \\ 
  Hispanic & 0.30 & 0.90 & 0.10 & 0.00 & 0.44 \\ 
  Native American & 0.22 & 0.90 & 0.09 & 0.01 & 0.04 \\ 
  Black & 0.13 & 0.50 & 0.48 & 0.02 & 0.05 \\ 
  Other & 0.21 & 0.55 & 0.44 & 0.01 & 0.05 \\ 
  Total Pop & 0.31 & 0.61 & 0.39 & 0.00 &  \\ 
   \hline
\end{tabular}
\end{center}
\end{table}

% latex table generated in R 2.13.1 by xtable 1.6-0 package
% Thu Jan 26 11:22:20 2012
\begin{table}[htb]
\begin{center}
\caption{Mine Inspector 2010 CD 7 (Proposed)}
\label{smine_cvap_cd_7}
\begin{tabular}{lccccc}
  \hline
Racial Group & Turnout & D Vote & R Vote & S Vote & Total CVAP \\ 
  \hline
White & 0.33 & 0.50 & 0.50 & 0.00 & 0.39 \\ 
  Hispanic & 0.20 & 0.90 & 0.10 & 0.00 & 0.42 \\ 
  Native American & 0.13 & 0.67 & 0.30 & 0.03 & 0.03 \\ 
  Black & 0.24 & 0.80 & 0.20 & 0.01 & 0.10 \\ 
  Other & 0.26 & 0.72 & 0.26 & 0.01 & 0.06 \\ 
  Total Pop & 0.25 & 0.68 & 0.32 & 0.00 &  \\ 
   \hline
\end{tabular}
\end{center}
\end{table}


% latex table generated in R 2.13.1 by xtable 1.6-0 package
% Thu Jan 26 11:22:20 2012
\begin{table}[htb]
\begin{center}
\caption{US President 2004 CD 3 (Proposed)}
\label{pres04_cvap_cd_3}
\begin{tabular}{lccccc}
  \hline
Racial Group & Turnout & D Vote & R Vote & S Vote & Total CVAP \\ 
  \hline
White & 0.36 & 0.45 & 0.55 & 0.00 & 0.43 \\ 
  Hispanic & 0.41 & 0.69 & 0.30 & 0.00 & 0.44 \\ 
  Native American & 0.44 & 0.67 & 0.31 & 0.02 & 0.04 \\ 
  Black & 0.17 & 0.35 & 0.60 & 0.05 & 0.05 \\ 
  Other & 0.22 & 0.33 & 0.62 & 0.05 & 0.05 \\ 
  Total Pop & 0.37 & 0.57 & 0.42 & 0.01 &  \\ 
   \hline
\end{tabular}
\end{center}
\end{table}

% latex table generated in R 2.13.1 by xtable 1.6-0 package
% Thu Jan 26 11:22:20 2012
\begin{table}[htb]
\begin{center}
\caption{US President 2004 CD 7 (Proposed)}
\label{pres04_cvap_cd_7}
\begin{tabular}{lccccc}
  \hline
Racial Group & Turnout & D Vote & R Vote & S Vote & Total CVAP \\ 
  \hline
White & 0.44 & 0.49 & 0.50 & 0.00 & 0.39 \\ 
  Hispanic & 0.30 & 0.76 & 0.23 & 0.01 & 0.42 \\ 
  Native American & 0.21 & 0.48 & 0.46 & 0.06 & 0.03 \\ 
  Black & 0.22 & 0.61 & 0.36 & 0.03 & 0.10 \\ 
  Other & 0.12 & 0.40 & 0.52 & 0.08 & 0.06 \\ 
  Total Pop & 0.34 & 0.60 & 0.39 & 0.01 &  \\ 
   \hline
\end{tabular}
\end{center}
\end{table}


% latex table generated in R 2.13.1 by xtable 1.6-0 package
% Thu Jan 26 11:22:20 2012
\begin{table}[htb]
\begin{center}
\caption{US President 2008 CD 3 (Proposed)}
\label{pres08_cvap_cd_3}
\begin{tabular}{lccccc}
  \hline
Racial Group & Turnout & D Vote & R Vote & S Vote & Total CVAP \\ 
  \hline
White & 0.50 & 0.40 & 0.59 & 0.01 & 0.42 \\ 
  Hispanic & 0.42 & 0.79 & 0.20 & 0.01 & 0.44 \\ 
  Native American & 0.37 & 0.76 & 0.21 & 0.03 & 0.04 \\ 
  Black & 0.21 & 0.52 & 0.39 & 0.09 & 0.05 \\ 
  Other & 0.30 & 0.55 & 0.36 & 0.09 & 0.05 \\ 
  Total Pop & 0.43 & 0.59 & 0.40 & 0.01 &  \\ 
   \hline
\end{tabular}
\end{center}
\end{table}

% latex table generated in R 2.13.1 by xtable 1.6-0 package
% Thu Jan 26 11:22:20 2012
\begin{table}[htb]
\begin{center}
\caption{US President 2008 CD 7 (Proposed)}
\label{pres08_cvap_cd_7}
\begin{tabular}{lccccc}
  \hline
Racial Group & Turnout & D Vote & R Vote & S Vote & Total CVAP \\ 
  \hline
White & 0.50 & 0.51 & 0.48 & 0.01 & 0.39 \\ 
  Hispanic & 0.24 & 0.83 & 0.15 & 0.02 & 0.42 \\ 
  Native American & 0.24 & 0.53 & 0.39 & 0.08 & 0.03 \\ 
  Black & 0.48 & 0.83 & 0.15 & 0.03 & 0.10 \\ 
  Other & 0.49 & 0.61 & 0.34 & 0.05 & 0.06 \\ 
  Total Pop & 0.38 & 0.64 & 0.34 & 0.02 &  \\ 
   \hline
\end{tabular}
\end{center}
\end{table}


% latex table generated in R 2.13.1 by xtable 1.6-0 package
% Thu Jan 26 11:22:20 2012
\begin{table}[htb]
\begin{center}
\caption{Secretary of State 2006 CD 3 (Proposed)}
\label{sos06_cvap_cd_3}
\begin{tabular}{lccccc}
  \hline
Racial Group & Turnout & D Vote & R Vote & S Vote & Total CVAP \\ 
  \hline
White & 0.27 & 0.40 & 0.58 & 0.02 & 0.43 \\ 
  Hispanic & 0.26 & 0.77 & 0.21 & 0.02 & 0.44 \\ 
  Native American & 0.30 & 0.73 & 0.22 & 0.05 & 0.04 \\ 
  Black & 0.18 & 0.40 & 0.47 & 0.14 & 0.05 \\ 
  Other & 0.23 & 0.41 & 0.43 & 0.16 & 0.05 \\ 
  Total Pop & 0.26 & 0.58 & 0.39 & 0.03 &  \\ 
   \hline
\end{tabular}
\end{center}
\end{table}

% latex table generated in R 2.13.1 by xtable 1.6-0 package
% Thu Jan 26 11:22:20 2012
\begin{table}[htb]
\begin{center}
\caption{Secretary of State 2006 CD 7 (Proposed)}
\label{sos06_cvap_cd_7}
\begin{tabular}{lccccc}
  \hline
Racial Group & Turnout & D Vote & R Vote & S Vote & Total CVAP \\ 
  \hline
White & 0.32 & 0.46 & 0.52 & 0.02 & 0.39 \\ 
  Hispanic & 0.17 & 0.76 & 0.20 & 0.04 & 0.42 \\ 
  Native American & 0.16 & 0.46 & 0.37 & 0.17 & 0.03 \\ 
  Black & 0.18 & 0.52 & 0.38 & 0.10 & 0.10 \\ 
  Other & 0.14 & 0.46 & 0.35 & 0.20 & 0.06 \\ 
  Total Pop & 0.23 & 0.56 & 0.40 & 0.04 &  \\ 
   \hline
\end{tabular}
\end{center}
\end{table}


% latex table generated in R 2.13.1 by xtable 1.6-0 package
% Thu Jan 26 11:22:20 2012
\begin{table}[htb]
\begin{center}
\caption{Secretary of State 2010 CD 3 (Proposed)}
\label{sos10_cvap_cd_3}
\begin{tabular}{lccccc}
  \hline
Racial Group & Turnout & D Vote & R Vote & S Vote & Total CVAP \\ 
  \hline
White & 0.37 & 0.34 & 0.66 & 0.00 & 0.43 \\ 
  Hispanic & 0.31 & 0.87 & 0.13 & 0.00 & 0.44 \\ 
  Native American & 0.23 & 0.91 & 0.08 & 0.01 & 0.04 \\ 
  Black & 0.14 & 0.48 & 0.50 & 0.02 & 0.05 \\ 
  Other & 0.23 & 0.65 & 0.34 & 0.01 & 0.05 \\ 
  Total Pop & 0.32 & 0.60 & 0.40 & 0.00 &  \\ 
   \hline
\end{tabular}
\end{center}
\end{table}

% latex table generated in R 2.13.1 by xtable 1.6-0 package
% Thu Jan 26 11:22:20 2012
\begin{table}[htb]
\begin{center}
\caption{Secretary of State 2010 CD 7 (Proposed)}
\label{sos10_cvap_cd_7}
\begin{tabular}{lccccc}
  \hline
Racial Group & Turnout & D Vote & R Vote & S Vote & Total CVAP \\ 
  \hline
White & 0.34 & 0.49 & 0.51 & 0.00 & 0.39 \\ 
  Hispanic & 0.19 & 0.88 & 0.12 & 0.00 & 0.42 \\ 
  Native American & 0.15 & 0.73 & 0.24 & 0.03 & 0.03 \\ 
  Black & 0.26 & 0.79 & 0.20 & 0.01 & 0.10 \\ 
  Other & 0.26 & 0.68 & 0.30 & 0.01 & 0.06 \\ 
  Total Pop & 0.26 & 0.66 & 0.34 & 0.00 &  \\ 
   \hline
\end{tabular}
\end{center}
\end{table}


\subsection{Benchmark Map}
\clearpage

% latex table generated in R 2.13.1 by xtable 1.6-0 package
% Thu Jan 26 11:22:20 2012
\begin{table}[htb]
\begin{center}
\caption{Mine Inspector 2010 CD 4 (Benchmark)}
\label{smine_cvap_cd_4_benchmark}
\begin{tabular}{lccccc}
  \hline
Racial Group & Turnout & D Vote & R Vote & S Vote & Total CVAP \\ 
  \hline
White & 0.34 & 0.56 & 0.44 & 0.00 & 0.40 \\ 
  Hispanic & 0.19 & 0.89 & 0.11 & 0.00 & 0.41 \\ 
  Native American & 0.13 & 0.65 & 0.32 & 0.03 & 0.03 \\ 
  Black & 0.23 & 0.82 & 0.17 & 0.01 & 0.10 \\ 
  Other & 0.30 & 0.58 & 0.41 & 0.01 & 0.06 \\ 
  Total Votes & 0.26 & 0.68 & 0.31 & 0.00 &  \\ 
   \hline
\end{tabular}
\end{center}
\end{table}

% latex table generated in R 2.13.1 by xtable 1.6-0 package
% Thu Jan 26 11:22:20 2012
\begin{table}[htb]
\begin{center}
\caption{Mine Inspector 2010 CD 7 (Benchmark)}
\label{smine_cvap_cd_7_benchmark}
\begin{tabular}{lccccc}
  \hline
Racial Group & Turnout & D Vote & R Vote & S Vote & Total CVAP \\ 
  \hline
White & 0.32 & 0.33 & 0.67 & 0.00 & 0.47 \\ 
  Hispanic & 0.31 & 0.89 & 0.11 & 0.00 & 0.39 \\ 
  Native American & 0.16 & 0.89 & 0.10 & 0.01 & 0.05 \\ 
  Black & 0.16 & 0.54 & 0.44 & 0.01 & 0.04 \\ 
  Other & 0.37 & 0.67 & 0.32 & 0.01 & 0.05 \\ 
  Total Votes & 0.30 & 0.59 & 0.41 & 0.00 &  \\ 
   \hline
\end{tabular}
\end{center}
\end{table}


% latex table generated in R 2.13.1 by xtable 1.6-0 package
% Thu Jan 26 11:22:20 2012
\begin{table}[htb]
\begin{center}
\caption{US President 2004 CD 4 (Benchmark)}
\label{pres04_cvap_cd_4_benchmark}
\begin{tabular}{lccccc}
  \hline
Racial Group & Turnout & D Vote & R Vote & S Vote & Total CVAP \\ 
  \hline
White & 0.43 & 0.55 & 0.45 & 0.00 & 0.40 \\ 
  Hispanic & 0.31 & 0.73 & 0.26 & 0.01 & 0.41 \\ 
  Native American & 0.24 & 0.48 & 0.47 & 0.05 & 0.03 \\ 
  Black & 0.21 & 0.59 & 0.38 & 0.02 & 0.10 \\ 
  Other & 0.15 & 0.42 & 0.51 & 0.07 & 0.06 \\ 
  Total Votes & 0.34 & 0.61 & 0.38 & 0.01 &  \\ 
   \hline
\end{tabular}
\end{center}
\end{table}

% latex table generated in R 2.13.1 by xtable 1.6-0 package
% Thu Jan 26 11:22:21 2012
\begin{table}[htb]
\begin{center}
\caption{US President 2004 CD 7 (Benchmark)}
\label{pres04_cvap_cd_7_benchmark}
\begin{tabular}{lccccc}
  \hline
Racial Group & Turnout & D Vote & R Vote & S Vote & Total CVAP \\ 
  \hline
White & 0.37 & 0.45 & 0.54 & 0.00 & 0.47 \\ 
  Hispanic & 0.41 & 0.70 & 0.30 & 0.00 & 0.39 \\ 
  Native American & 0.26 & 0.62 & 0.35 & 0.03 & 0.05 \\ 
  Black & 0.18 & 0.56 & 0.39 & 0.05 & 0.04 \\ 
  Other & 0.35 & 0.44 & 0.53 & 0.03 & 0.05 \\ 
  Total Votes & 0.37 & 0.57 & 0.43 & 0.01 &  \\ 
   \hline
\end{tabular}
\end{center}
\end{table}


% latex table generated in R 2.13.1 by xtable 1.6-0 package
% Thu Jan 26 11:22:21 2012
\begin{table}[htb]
\begin{center}
\caption{US President 2008 CD 4 (Benchmark)}
\label{pres08_cvap_cd_4_benchmark}
\begin{tabular}{lccccc}
  \hline
Racial Group & Turnout & D Vote & R Vote & S Vote & Total CVAP \\ 
  \hline
White & 0.51 & 0.57 & 0.42 & 0.01 & 0.40 \\ 
  Hispanic & 0.23 & 0.82 & 0.17 & 0.02 & 0.41 \\ 
  Native American & 0.28 & 0.57 & 0.35 & 0.09 & 0.03 \\ 
  Black & 0.48 & 0.82 & 0.15 & 0.03 & 0.10 \\ 
  Other & 0.43 & 0.43 & 0.51 & 0.06 & 0.06 \\ 
  Total Votes & 0.38 & 0.65 & 0.33 & 0.02 &  \\ 
   \hline
\end{tabular}
\end{center}
\end{table}

% latex table generated in R 2.13.1 by xtable 1.6-0 package
% Thu Jan 26 11:22:21 2012
\begin{table}[htb]
\begin{center}
\caption{US President 2008 CD 7 (Benchmark)}
\label{pres08_cvap_cd_7_benchmark}
\begin{tabular}{lccccc}
  \hline
Racial Group & Turnout & D Vote & R Vote & S Vote & Total CVAP \\ 
  \hline
White & 0.43 & 0.39 & 0.61 & 0.01 & 0.47 \\ 
  Hispanic & 0.43 & 0.78 & 0.21 & 0.01 & 0.39 \\ 
  Native American & 0.28 & 0.71 & 0.26 & 0.03 & 0.05 \\ 
  Black & 0.32 & 0.57 & 0.36 & 0.07 & 0.04 \\ 
  Other & 0.49 & 0.61 & 0.34 & 0.05 & 0.05 \\ 
  Total Votes & 0.42 & 0.57 & 0.41 & 0.01 &  \\ 
   \hline
\end{tabular}
\end{center}
\end{table}


% latex table generated in R 2.13.1 by xtable 1.6-0 package
% Thu Jan 26 11:22:21 2012
\begin{table}[htb]
\begin{center}
\caption{Secretary of State 2006 CD 4 (Benchmark)}
\label{sos06_cvap_cd_4_benchmark}
\begin{tabular}{lccccc}
  \hline
Racial Group & Turnout & D Vote & R Vote & S Vote & Total CVAP \\ 
  \hline
White & 0.30 & 0.49 & 0.48 & 0.02 & 0.40 \\ 
  Hispanic & 0.17 & 0.72 & 0.25 & 0.03 & 0.41 \\ 
  Native American & 0.15 & 0.47 & 0.37 & 0.17 & 0.03 \\ 
  Black & 0.17 & 0.60 & 0.30 & 0.10 & 0.10 \\ 
  Other & 0.17 & 0.40 & 0.44 & 0.16 & 0.06 \\ 
  Total Votes & 0.22 & 0.57 & 0.39 & 0.04 &  \\ 
   \hline
\end{tabular}
\end{center}
\end{table}

% latex table generated in R 2.13.1 by xtable 1.6-0 package
% Thu Jan 26 11:22:21 2012
\begin{table}[htb]
\begin{center}
\caption{Secretary of State 2006 CD 7 (Benchmark)}
\label{sos06_cvap_cd_7_benchmark}
\begin{tabular}{lccccc}
  \hline
Racial Group & Turnout & D Vote & R Vote & S Vote & Total CVAP \\ 
  \hline
White & 0.27 & 0.41 & 0.56 & 0.02 & 0.47 \\ 
  Hispanic & 0.26 & 0.77 & 0.21 & 0.03 & 0.39 \\ 
  Native American & 0.20 & 0.66 & 0.27 & 0.07 & 0.05 \\ 
  Black & 0.18 & 0.51 & 0.34 & 0.15 & 0.04 \\ 
  Other & 0.22 & 0.33 & 0.49 & 0.17 & 0.05 \\ 
  Total Votes & 0.26 & 0.56 & 0.40 & 0.04 &  \\ 
   \hline
\end{tabular}
\end{center}
\end{table}


% latex table generated in R 2.13.1 by xtable 1.6-0 package
% Thu Jan 26 11:22:21 2012
\begin{table}[htb]
\begin{center}
\caption{Secretary of State 2010 CD 4 (Benchmark)}
\label{sos10_cvap_cd_4_benchmark}
\begin{tabular}{lccccc}
  \hline
Racial Group & Turnout & D Vote & R Vote & S Vote & Total CVAP \\ 
  \hline
White & 0.35 & 0.54 & 0.46 & 0.00 & 0.40 \\ 
  Hispanic & 0.19 & 0.88 & 0.12 & 0.00 & 0.41 \\ 
  Native American & 0.15 & 0.69 & 0.29 & 0.02 & 0.03 \\ 
  Black & 0.23 & 0.81 & 0.18 & 0.01 & 0.10 \\ 
  Other & 0.33 & 0.54 & 0.45 & 0.01 & 0.06 \\ 
  Total Votes & 0.26 & 0.67 & 0.33 & 0.00 &  \\ 
   \hline
\end{tabular}
\end{center}
\end{table}

% latex table generated in R 2.13.1 by xtable 1.6-0 package
% Thu Jan 26 11:22:21 2012
\begin{table}[htb]
\begin{center}
\caption{Secretary of State 2010 CD 7 (Benchmark)}
\label{sos10_cvap_cd_7_benchmark}
\begin{tabular}{lccccc}
  \hline
Racial Group & Turnout & D Vote & R Vote & S Vote & Total CVAP \\ 
  \hline
White & 0.33 & 0.34 & 0.66 & 0.00 & 0.47 \\ 
  Hispanic & 0.30 & 0.85 & 0.15 & 0.00 & 0.39 \\ 
  Native American & 0.17 & 0.89 & 0.11 & 0.01 & 0.05 \\ 
  Black & 0.18 & 0.48 & 0.51 & 0.01 & 0.04 \\ 
  Other & 0.36 & 0.76 & 0.23 & 0.01 & 0.05 \\ 
  Total Votes & 0.31 & 0.58 & 0.42 & 0.00 &  \\ 
   \hline
\end{tabular}
\end{center}
\end{table}



%\section{Ecological Inference Estimates: VAP CD}
%\subsection{Proposed Map}
%\clearpage

%% latex table generated in R 2.13.1 by xtable 1.6-0 package
% Thu Jan 26 11:22:21 2012
\begin{table}[htb]
\begin{center}
\caption{Mine Inspector 2010 CD 3 (Proposed)}
\label{smine_vap_cd_3}
\begin{tabular}{lccccc}
  \hline
Racial Group & Turnout & D Vote & R Vote & S Vote & Total VAP \\ 
  \hline
White & 0.40 & 0.38 & 0.62 & 0.00 & 0.35 \\ 
  Hispanic & 0.18 & 0.90 & 0.10 & 0.00 & 0.55 \\ 
  Native American & 0.23 & 0.90 & 0.09 & 0.01 & 0.03 \\ 
  Black & 0.10 & 0.60 & 0.37 & 0.03 & 0.04 \\ 
  Other & 0.16 & 0.66 & 0.32 & 0.02 & 0.03 \\ 
  Total Pop & 0.25 & 0.61 & 0.39 & 0.00 &  \\ 
   \hline
\end{tabular}
\end{center}
\end{table}

%% latex table generated in R 2.13.1 by xtable 1.6-0 package
% Thu Jan 26 11:22:21 2012
\begin{table}[htb]
\begin{center}
\caption{Mine Inspector 2010 CD 7 (Proposed)}
\label{smine_vap_cd_7}
\begin{tabular}{lccccc}
  \hline
Racial Group & Turnout & D Vote & R Vote & S Vote & Total VAP \\ 
  \hline
White & 0.38 & 0.55 & 0.45 & 0.00 & 0.27 \\ 
  Hispanic & 0.10 & 0.89 & 0.11 & 0.00 & 0.58 \\ 
  Native American & 0.28 & 0.86 & 0.12 & 0.02 & 0.02 \\ 
  Black & 0.17 & 0.81 & 0.18 & 0.01 & 0.09 \\ 
  Other & 0.24 & 0.49 & 0.50 & 0.01 & 0.04 \\ 
  Total Pop & 0.19 & 0.68 & 0.32 & 0.00 &  \\ 
   \hline
\end{tabular}
\end{center}
\end{table}


%% latex table generated in R 2.13.1 by xtable 1.6-0 package
% Thu Jan 26 11:22:21 2012
\begin{table}[htb]
\begin{center}
\caption{US President 2004 CD 3 (Proposed)}
\label{pres04_vap_cd_3}
\begin{tabular}{lccccc}
  \hline
Racial Group & Turnout & D Vote & R Vote & S Vote & Total VAP \\ 
  \hline
White & 0.35 & 0.47 & 0.52 & 0.00 & 0.34 \\ 
  Hispanic & 0.27 & 0.67 & 0.33 & 0.00 & 0.55 \\ 
  Native American & 0.44 & 0.63 & 0.35 & 0.02 & 0.03 \\ 
  Black & 0.19 & 0.41 & 0.55 & 0.04 & 0.04 \\ 
  Other & 0.22 & 0.38 & 0.57 & 0.05 & 0.03 \\ 
  Total Pop & 0.30 & 0.57 & 0.42 & 0.01 &  \\ 
   \hline
\end{tabular}
\end{center}
\end{table}

%% latex table generated in R 2.13.1 by xtable 1.6-0 package
% Thu Jan 26 11:22:21 2012
\begin{table}[htb]
\begin{center}
\caption{US President 2004 CD 7 (Proposed)}
\label{pres04_vap_cd_7}
\begin{tabular}{lccccc}
  \hline
Racial Group & Turnout & D Vote & R Vote & S Vote & Total VAP \\ 
  \hline
White & 0.48 & 0.51 & 0.49 & 0.00 & 0.27 \\ 
  Hispanic & 0.17 & 0.73 & 0.26 & 0.01 & 0.58 \\ 
  Native American & 0.25 & 0.47 & 0.47 & 0.06 & 0.02 \\ 
  Black & 0.13 & 0.59 & 0.37 & 0.04 & 0.09 \\ 
  Other & 0.12 & 0.40 & 0.52 & 0.08 & 0.04 \\ 
  Total Pop & 0.25 & 0.60 & 0.39 & 0.01 &  \\ 
   \hline
\end{tabular}
\end{center}
\end{table}


%% latex table generated in R 2.13.1 by xtable 1.6-0 package
% Thu Jan 26 11:22:21 2012
\begin{table}[htb]
\begin{center}
\caption{US President 2008 CD 3 (Proposed)}
\label{pres08_vap_cd_3}
\begin{tabular}{lccccc}
  \hline
Racial Group & Turnout & D Vote & R Vote & S Vote & Total VAP \\ 
  \hline
White & 0.54 & 0.41 & 0.58 & 0.01 & 0.35 \\ 
  Hispanic & 0.25 & 0.78 & 0.21 & 0.01 & 0.55 \\ 
  Native American & 0.37 & 0.75 & 0.22 & 0.03 & 0.03 \\ 
  Black & 0.22 & 0.59 & 0.34 & 0.07 & 0.04 \\ 
  Other & 0.43 & 0.74 & 0.19 & 0.07 & 0.03 \\ 
  Total Pop & 0.36 & 0.58 & 0.41 & 0.01 &  \\ 
   \hline
\end{tabular}
\end{center}
\end{table}

%% latex table generated in R 2.13.1 by xtable 1.6-0 package
% Thu Jan 26 11:22:21 2012
\begin{table}[htb]
\begin{center}
\caption{US President 2008 CD 7 (Proposed)}
\label{pres08_vap_cd_7}
\begin{tabular}{lccccc}
  \hline
Racial Group & Turnout & D Vote & R Vote & S Vote & Total VAP \\ 
  \hline
White & 0.58 & 0.55 & 0.45 & 0.01 & 0.27 \\ 
  Hispanic & 0.12 & 0.80 & 0.18 & 0.02 & 0.58 \\ 
  Native American & 0.41 & 0.64 & 0.28 & 0.08 & 0.02 \\ 
  Black & 0.40 & 0.84 & 0.13 & 0.03 & 0.09 \\ 
  Other & 0.35 & 0.47 & 0.47 & 0.07 & 0.04 \\ 
  Total Pop & 0.28 & 0.64 & 0.34 & 0.02 &  \\ 
   \hline
\end{tabular}
\end{center}
\end{table}


%% latex table generated in R 2.13.1 by xtable 1.6-0 package
% Thu Jan 26 11:22:21 2012
\begin{table}[htb]
\begin{center}
\caption{Secretary of State 2006 CD 3 (Proposed)}
\label{sos06_vap_cd_3}
\begin{tabular}{lccccc}
  \hline
Racial Group & Turnout & D Vote & R Vote & S Vote & Total VAP \\ 
  \hline
White & 0.28 & 0.39 & 0.58 & 0.03 & 0.34 \\ 
  Hispanic & 0.16 & 0.78 & 0.20 & 0.02 & 0.55 \\ 
  Native American & 0.32 & 0.69 & 0.25 & 0.06 & 0.03 \\ 
  Black & 0.15 & 0.38 & 0.46 & 0.16 & 0.04 \\ 
  Other & 0.21 & 0.43 & 0.40 & 0.17 & 0.03 \\ 
  Total Pop & 0.21 & 0.57 & 0.39 & 0.03 &  \\ 
   \hline
\end{tabular}
\end{center}
\end{table}

%% latex table generated in R 2.13.1 by xtable 1.6-0 package
% Thu Jan 26 11:22:21 2012
\begin{table}[htb]
\begin{center}
\caption{Secretary of State 2006 CD 7 (Proposed)}
\label{sos06_vap_cd_7}
\begin{tabular}{lccccc}
  \hline
Racial Group & Turnout & D Vote & R Vote & S Vote & Total VAP \\ 
  \hline
White & 0.35 & 0.46 & 0.52 & 0.02 & 0.27 \\ 
  Hispanic & 0.09 & 0.75 & 0.21 & 0.04 & 0.58 \\ 
  Native American & 0.22 & 0.49 & 0.34 & 0.18 & 0.02 \\ 
  Black & 0.13 & 0.50 & 0.39 & 0.11 & 0.09 \\ 
  Other & 0.14 & 0.43 & 0.38 & 0.19 & 0.04 \\ 
  Total Pop & 0.17 & 0.56 & 0.40 & 0.04 &  \\ 
   \hline
\end{tabular}
\end{center}
\end{table}


%% latex table generated in R 2.13.1 by xtable 1.6-0 package
% Thu Jan 26 11:22:21 2012
\begin{table}[htb]
\begin{center}
\caption{Secretary of State 2010 CD 3 (Proposed)}
\label{sos10_vap_cd_3}
\begin{tabular}{lccccc}
  \hline
Racial Group & Turnout & D Vote & R Vote & S Vote & Total VAP \\ 
  \hline
White & 0.41 & 0.38 & 0.62 & 0.00 & 0.35 \\ 
  Hispanic & 0.18 & 0.87 & 0.13 & 0.00 & 0.55 \\ 
  Native American & 0.23 & 0.91 & 0.08 & 0.01 & 0.03 \\ 
  Black & 0.11 & 0.59 & 0.38 & 0.02 & 0.04 \\ 
  Other & 0.18 & 0.71 & 0.28 & 0.02 & 0.03 \\ 
  Total Pop & 0.25 & 0.59 & 0.41 & 0.00 &  \\ 
   \hline
\end{tabular}
\end{center}
\end{table}

%% latex table generated in R 2.13.1 by xtable 1.6-0 package
% Thu Jan 26 11:22:21 2012
\begin{table}[htb]
\begin{center}
\caption{Secretary of State 2010 CD 7 (Proposed)}
\label{sos10_vap_cd_7}
\begin{tabular}{lccccc}
  \hline
Racial Group & Turnout & D Vote & R Vote & S Vote & Total VAP \\ 
  \hline
White & 0.40 & 0.52 & 0.47 & 0.00 & 0.27 \\ 
  Hispanic & 0.09 & 0.88 & 0.12 & 0.00 & 0.58 \\ 
  Native American & 0.29 & 0.81 & 0.17 & 0.02 & 0.02 \\ 
  Black & 0.19 & 0.81 & 0.18 & 0.01 & 0.09 \\ 
  Other & 0.24 & 0.52 & 0.47 & 0.01 & 0.04 \\ 
  Total Pop & 0.19 & 0.66 & 0.34 & 0.00 &  \\ 
   \hline
\end{tabular}
\end{center}
\end{table}


%\subsection{Benchmark Map}
%\clearpage
%% latex table generated in R 2.13.1 by xtable 1.6-0 package
% Thu Jan 26 11:22:21 2012
\begin{table}[htb]
\begin{center}
\caption{Mine Inspector 2010 CD 4 (Benchmark)}
\label{smine_vap_cd_4_benchmark}
\begin{tabular}{lccccc}
  \hline
Racial Group & Turnout & D Vote & R Vote & S Vote & Total VAP \\ 
  \hline
White & 0.39 & 0.58 & 0.42 & 0.00 & 0.27 \\ 
  Hispanic & 0.09 & 0.89 & 0.11 & 0.00 & 0.58 \\ 
  Native American & 0.28 & 0.83 & 0.16 & 0.02 & 0.02 \\ 
  Black & 0.17 & 0.83 & 0.16 & 0.01 & 0.09 \\ 
  Other & 0.24 & 0.44 & 0.55 & 0.02 & 0.04 \\ 
  Total Votes & 0.19 & 0.69 & 0.31 & 0.00 &  \\ 
   \hline
\end{tabular}
\end{center}
\end{table}

%% latex table generated in R 2.13.1 by xtable 1.6-0 package
% Thu Jan 26 11:22:21 2012
\begin{table}[htb]
\begin{center}
\caption{Mine Inspector 2010 CD 7 (Benchmark)}
\label{smine_vap_cd_7_benchmark}
\begin{tabular}{lccccc}
  \hline
Racial Group & Turnout & D Vote & R Vote & S Vote & Total VAP \\ 
  \hline
White & 0.36 & 0.36 & 0.64 & 0.00 & 0.39 \\ 
  Hispanic & 0.19 & 0.90 & 0.10 & 0.00 & 0.50 \\ 
  Native American & 0.17 & 0.87 & 0.12 & 0.01 & 0.04 \\ 
  Black & 0.14 & 0.52 & 0.47 & 0.01 & 0.04 \\ 
  Other & 0.18 & 0.67 & 0.32 & 0.01 & 0.03 \\ 
  Total Votes & 0.25 & 0.59 & 0.41 & 0.00 &  \\ 
   \hline
\end{tabular}
\end{center}
\end{table}


%% latex table generated in R 2.13.1 by xtable 1.6-0 package
% Thu Jan 26 11:22:21 2012
\begin{table}[htb]
\begin{center}
\caption{US President 2004 CD 4 (Benchmark)}
\label{pres04_vap_cd_4_benchmark}
\begin{tabular}{lccccc}
  \hline
Racial Group & Turnout & D Vote & R Vote & S Vote & Total VAP \\ 
  \hline
White & 0.47 & 0.54 & 0.46 & 0.00 & 0.27 \\ 
  Hispanic & 0.17 & 0.75 & 0.25 & 0.01 & 0.57 \\ 
  Native American & 0.27 & 0.46 & 0.48 & 0.06 & 0.02 \\ 
  Black & 0.15 & 0.55 & 0.42 & 0.03 & 0.09 \\ 
  Other & 0.13 & 0.49 & 0.45 & 0.07 & 0.04 \\ 
  Total Votes & 0.25 & 0.62 & 0.38 & 0.01 &  \\ 
   \hline
\end{tabular}
\end{center}
\end{table}

%% latex table generated in R 2.13.1 by xtable 1.6-0 package
% Thu Jan 26 11:22:21 2012
\begin{table}[htb]
\begin{center}
\caption{US President 2004 CD 7 (Benchmark)}
\label{pres04_vap_cd_7_benchmark}
\begin{tabular}{lccccc}
  \hline
Racial Group & Turnout & D Vote & R Vote & S Vote & Total VAP \\ 
  \hline
White & 0.38 & 0.48 & 0.51 & 0.00 & 0.39 \\ 
  Hispanic & 0.27 & 0.67 & 0.33 & 0.00 & 0.50 \\ 
  Native American & 0.26 & 0.61 & 0.36 & 0.03 & 0.04 \\ 
  Black & 0.16 & 0.45 & 0.50 & 0.05 & 0.04 \\ 
  Other & 0.24 & 0.46 & 0.50 & 0.04 & 0.03 \\ 
  Total Votes & 0.31 & 0.57 & 0.43 & 0.01 &  \\ 
   \hline
\end{tabular}
\end{center}
\end{table}


%% latex table generated in R 2.13.1 by xtable 1.6-0 package
% Thu Jan 26 11:22:21 2012
\begin{table}[htb]
\begin{center}
\caption{US President 2008 CD 4 (Benchmark)}
\label{pres08_vap_cd_4_benchmark}
\begin{tabular}{lccccc}
  \hline
Racial Group & Turnout & D Vote & R Vote & S Vote & Total VAP \\ 
  \hline
White & 0.59 & 0.56 & 0.43 & 0.01 & 0.27 \\ 
  Hispanic & 0.11 & 0.83 & 0.15 & 0.02 & 0.58 \\ 
  Native American & 0.46 & 0.64 & 0.27 & 0.09 & 0.02 \\ 
  Black & 0.39 & 0.84 & 0.13 & 0.03 & 0.09 \\ 
  Other & 0.36 & 0.46 & 0.48 & 0.06 & 0.04 \\ 
  Total Votes & 0.28 & 0.65 & 0.33 & 0.02 &  \\ 
   \hline
\end{tabular}
\end{center}
\end{table}

%% latex table generated in R 2.13.1 by xtable 1.6-0 package
% Thu Jan 26 11:22:21 2012
\begin{table}[htb]
\begin{center}
\caption{US President 2008 CD 7 (Benchmark)}
\label{pres08_vap_cd_7_benchmark}
\begin{tabular}{lccccc}
  \hline
Racial Group & Turnout & D Vote & R Vote & S Vote & Total VAP \\ 
  \hline
White & 0.48 & 0.41 & 0.58 & 0.01 & 0.39 \\ 
  Hispanic & 0.26 & 0.77 & 0.22 & 0.01 & 0.50 \\ 
  Native American & 0.30 & 0.70 & 0.27 & 0.03 & 0.04 \\ 
  Black & 0.29 & 0.66 & 0.28 & 0.06 & 0.04 \\ 
  Other & 0.47 & 0.63 & 0.31 & 0.07 & 0.03 \\ 
  Total Votes & 0.35 & 0.57 & 0.42 & 0.01 &  \\ 
   \hline
\end{tabular}
\end{center}
\end{table}


%% latex table generated in R 2.13.1 by xtable 1.6-0 package
% Thu Jan 26 11:22:21 2012
\begin{table}[htb]
\begin{center}
\caption{Secretary of State 2006 CD 4 (Benchmark)}
\label{sos06_vap_cd_4_benchmark}
\begin{tabular}{lccccc}
  \hline
Racial Group & Turnout & D Vote & R Vote & S Vote & Total VAP \\ 
  \hline
White & 0.33 & 0.49 & 0.49 & 0.02 & 0.27 \\ 
  Hispanic & 0.09 & 0.75 & 0.22 & 0.03 & 0.57 \\ 
  Native American & 0.21 & 0.48 & 0.35 & 0.17 & 0.02 \\ 
  Black & 0.12 & 0.49 & 0.39 & 0.12 & 0.09 \\ 
  Other & 0.15 & 0.41 & 0.42 & 0.17 & 0.04 \\ 
  Total Votes & 0.16 & 0.57 & 0.39 & 0.04 &  \\ 
   \hline
\end{tabular}
\end{center}
\end{table}

%% latex table generated in R 2.13.1 by xtable 1.6-0 package
% Thu Jan 26 11:22:21 2012
\begin{table}[htb]
\begin{center}
\caption{Secretary of State 2006 CD 7 (Benchmark)}
\label{sos06_vap_cd_7_benchmark}
\begin{tabular}{lccccc}
  \hline
Racial Group & Turnout & D Vote & R Vote & S Vote & Total VAP \\ 
  \hline
White & 0.28 & 0.42 & 0.56 & 0.02 & 0.39 \\ 
  Hispanic & 0.16 & 0.76 & 0.21 & 0.03 & 0.50 \\ 
  Native American & 0.21 & 0.63 & 0.30 & 0.07 & 0.04 \\ 
  Black & 0.16 & 0.45 & 0.39 & 0.16 & 0.04 \\ 
  Other & 0.22 & 0.40 & 0.42 & 0.18 & 0.03 \\ 
  Total Votes & 0.21 & 0.56 & 0.40 & 0.04 &  \\ 
   \hline
\end{tabular}
\end{center}
\end{table}


%% latex table generated in R 2.13.1 by xtable 1.6-0 package
% Thu Jan 26 11:22:21 2012
\begin{table}[htb]
\begin{center}
\caption{Secretary of State 2010 CD 4 (Benchmark)}
\label{sos10_vap_cd_4_benchmark}
\begin{tabular}{lccccc}
  \hline
Racial Group & Turnout & D Vote & R Vote & S Vote & Total VAP \\ 
  \hline
White & 0.40 & 0.56 & 0.44 & 0.00 & 0.27 \\ 
  Hispanic & 0.09 & 0.88 & 0.11 & 0.00 & 0.58 \\ 
  Native American & 0.28 & 0.78 & 0.20 & 0.02 & 0.02 \\ 
  Black & 0.19 & 0.81 & 0.18 & 0.01 & 0.09 \\ 
  Other & 0.26 & 0.46 & 0.53 & 0.01 & 0.04 \\ 
  Total Votes & 0.19 & 0.67 & 0.33 & 0.00 &  \\ 
   \hline
\end{tabular}
\end{center}
\end{table}

%% latex table generated in R 2.13.1 by xtable 1.6-0 package
% Thu Jan 26 11:22:21 2012
\begin{table}[htb]
\begin{center}
\caption{Secretary of State 2010 CD 7 (Benchmark)}
\label{sos10_vap_cd_7_benchmark}
\begin{tabular}{lccccc}
  \hline
Racial Group & Turnout & D Vote & R Vote & S Vote & Total VAP \\ 
  \hline
White & 0.37 & 0.37 & 0.63 & 0.00 & 0.39 \\ 
  Hispanic & 0.19 & 0.85 & 0.15 & 0.00 & 0.50 \\ 
  Native American & 0.17 & 0.90 & 0.09 & 0.01 & 0.04 \\ 
  Black & 0.15 & 0.63 & 0.36 & 0.01 & 0.04 \\ 
  Other & 0.17 & 0.67 & 0.32 & 0.01 & 0.03 \\ 
  Total Votes & 0.25 & 0.58 & 0.42 & 0.00 &  \\ 
   \hline
\end{tabular}
\end{center}
\end{table}


\section{CVAP LD}
\subsection{Proposed Map}
\clearpage

% latex table generated in R 2.13.1 by xtable 1.6-0 package
% Thu Jan 26 11:22:22 2012
\begin{table}[htb]
\begin{center}
\caption{Mine Inspector 2010 LD 2 (Proposed)}
\label{smine_cvap_ld_2}
\begin{tabular}{lccccc}
  \hline
Racial Group & Turnout & D Vote & R Vote & S Vote & Total CVAP \\ 
  \hline
White & 0.51 & 0.43 & 0.57 & 0.00 & 0.49 \\ 
  Hispanic & 0.28 & 0.89 & 0.10 & 0.00 & 0.41 \\ 
  Native American & 0.35 & 0.76 & 0.23 & 0.01 & 0.02 \\ 
  Black & 0.18 & 0.52 & 0.47 & 0.01 & 0.04 \\ 
  Other & 0.24 & 0.56 & 0.43 & 0.01 & 0.04 \\ 
  Total Pop & 0.39 & 0.58 & 0.42 & 0.00 &  \\ 
   \hline
\end{tabular}
\end{center}
\end{table}

% latex table generated in R 2.13.1 by xtable 1.6-0 package
% Thu Jan 26 11:22:22 2012
\begin{table}[htb]
\begin{center}
\caption{Mine Inspector 2010 LD 3 (Proposed)}
\label{smine_cvap_ld_3}
\begin{tabular}{lccccc}
  \hline
Racial Group & Turnout & D Vote & R Vote & S Vote & Total CVAP \\ 
  \hline
White & 0.33 & 0.53 & 0.47 & 0.00 & 0.45 \\ 
  Hispanic & 0.33 & 0.87 & 0.13 & 0.00 & 0.42 \\ 
  Native American & 0.21 & 0.74 & 0.25 & 0.01 & 0.04 \\ 
  Black & 0.30 & 0.64 & 0.35 & 0.01 & 0.03 \\ 
  Other & 0.45 & 0.65 & 0.34 & 0.00 & 0.05 \\ 
  Total Pop & 0.33 & 0.69 & 0.31 & 0.00 &  \\ 
   \hline
\end{tabular}
\end{center}
\end{table}

% latex table generated in R 2.13.1 by xtable 1.6-0 package
% Thu Jan 26 11:22:22 2012
\begin{table}[htb]
\begin{center}
\caption{Mine Inspector 2010 LD 4 (Proposed)}
\label{smine_cvap_ld_4}
\begin{tabular}{lccccc}
  \hline
Racial Group & Turnout & D Vote & R Vote & S Vote & Total CVAP \\ 
  \hline
White & 0.32 & 0.19 & 0.81 & 0.00 & 0.42 \\ 
  Hispanic & 0.28 & 0.88 & 0.12 & 0.00 & 0.42 \\ 
  Native American & 0.22 & 0.92 & 0.08 & 0.01 & 0.07 \\ 
  Black & 0.38 & 0.64 & 0.35 & 0.01 & 0.04 \\ 
  Other & 0.42 & 0.75 & 0.25 & 0.01 & 0.05 \\ 
  Total Pop & 0.30 & 0.55 & 0.45 & 0.00 &  \\ 
   \hline
\end{tabular}
\end{center}
\end{table}

% latex table generated in R 2.13.1 by xtable 1.6-0 package
% Thu Jan 26 11:22:22 2012
\begin{table}[htb]
\begin{center}
\caption{Mine Inspector 2010 LD 7 (Proposed)}
\label{smine_cvap_ld_7}
\begin{tabular}{lccccc}
  \hline
Racial Group & Turnout & D Vote & R Vote & S Vote & Total CVAP \\ 
  \hline
White & 0.44 & 0.23 & 0.77 & 0.00 & 0.28 \\ 
  Hispanic & 0.25 & 0.55 & 0.45 & 0.01 & 0.05 \\ 
  Native American & 0.34 & 0.85 & 0.15 & 0.00 & 0.63 \\ 
  Black & 0.41 & 0.56 & 0.42 & 0.02 & 0.01 \\ 
  Other & 0.28 & 0.80 & 0.19 & 0.01 & 0.04 \\ 
  Total Pop & 0.36 & 0.63 & 0.37 & 0.00 &  \\ 
   \hline
\end{tabular}
\end{center}
\end{table}

% latex table generated in R 2.13.1 by xtable 1.6-0 package
% Thu Jan 26 11:22:22 2012
\begin{table}[htb]
\begin{center}
\caption{Mine Inspector 2010 LD 8 (Proposed)}
\label{smine_cvap_ld_8}
\begin{tabular}{lccccc}
  \hline
Racial Group & Turnout & D Vote & R Vote & S Vote & Total CVAP \\ 
  \hline
White & 0.42 & 0.38 & 0.62 & 0.00 & 0.53 \\ 
  Hispanic & 0.18 & 0.83 & 0.17 & 0.00 & 0.30 \\ 
  Native American & 0.15 & 0.75 & 0.24 & 0.01 & 0.07 \\ 
  Black & 0.22 & 0.42 & 0.57 & 0.01 & 0.04 \\ 
  Other & 0.21 & 0.51 & 0.49 & 0.01 & 0.06 \\ 
  Total Pop & 0.31 & 0.48 & 0.52 & 0.00 &  \\ 
   \hline
\end{tabular}
\end{center}
\end{table}

% latex table generated in R 2.13.1 by xtable 1.6-0 package
% Thu Jan 26 11:22:22 2012
\begin{table}[htb]
\begin{center}
\caption{Mine Inspector 2010 LD 19 (Proposed)}
\label{smine_cvap_ld_19}
\begin{tabular}{lccccc}
  \hline
Racial Group & Turnout & D Vote & R Vote & S Vote & Total CVAP \\ 
  \hline
White & 0.22 & 0.38 & 0.62 & 0.00 & 0.38 \\ 
  Hispanic & 0.20 & 0.82 & 0.18 & 0.00 & 0.47 \\ 
  Native American & 0.20 & 0.47 & 0.50 & 0.04 & 0.02 \\ 
  Black & 0.45 & 0.66 & 0.33 & 0.00 & 0.09 \\ 
  Other & 0.48 & 0.67 & 0.33 & 0.01 & 0.05 \\ 
  Total Pop & 0.24 & 0.63 & 0.37 & 0.00 &  \\ 
   \hline
\end{tabular}
\end{center}
\end{table}

% latex table generated in R 2.13.1 by xtable 1.6-0 package
% Thu Jan 26 11:22:22 2012
\begin{table}[htb]
\begin{center}
\caption{Mine Inspector 2010 LD 24 (Proposed)}
\label{smine_cvap_ld_24}
\begin{tabular}{lccccc}
  \hline
Racial Group & Turnout & D Vote & R Vote & S Vote & Total CVAP \\ 
  \hline
White & 0.35 & 0.55 & 0.45 & 0.00 & 0.64 \\ 
  Hispanic & 0.14 & 0.79 & 0.20 & 0.01 & 0.22 \\ 
  Native American & 0.22 & 0.59 & 0.39 & 0.02 & 0.04 \\ 
  Black & 0.33 & 0.68 & 0.31 & 0.01 & 0.05 \\ 
  Other & 0.46 & 0.68 & 0.31 & 0.01 & 0.05 \\ 
  Total Pop & 0.30 & 0.60 & 0.40 & 0.00 &  \\ 
   \hline
\end{tabular}
\end{center}
\end{table}

% latex table generated in R 2.13.1 by xtable 1.6-0 package
% Thu Jan 26 11:22:22 2012
\begin{table}[htb]
\begin{center}
\caption{Mine Inspector 2010 LD 26 (Proposed)}
\label{smine_cvap_ld_26}
\begin{tabular}{lccccc}
  \hline
Racial Group & Turnout & D Vote & R Vote & S Vote & Total CVAP \\ 
  \hline
White & 0.27 & 0.55 & 0.45 & 0.00 & 0.65 \\ 
  Hispanic & 0.15 & 0.52 & 0.47 & 0.01 & 0.18 \\ 
  Native American & 0.31 & 0.65 & 0.34 & 0.01 & 0.06 \\ 
  Black & 0.29 & 0.59 & 0.40 & 0.01 & 0.05 \\ 
  Other & 0.29 & 0.58 & 0.40 & 0.01 & 0.06 \\ 
  Total Pop & 0.25 & 0.56 & 0.44 & 0.00 &  \\ 
   \hline
\end{tabular}
\end{center}
\end{table}

% latex table generated in R 2.13.1 by xtable 1.6-0 package
% Thu Jan 26 11:22:22 2012
\begin{table}[htb]
\begin{center}
\caption{Mine Inspector 2010 LD 27 (Proposed)}
\label{smine_cvap_ld_27}
\begin{tabular}{lccccc}
  \hline
Racial Group & Turnout & D Vote & R Vote & S Vote & Total CVAP \\ 
  \hline
White & 0.30 & 0.54 & 0.46 & 0.00 & 0.30 \\ 
  Hispanic & 0.21 & 0.87 & 0.12 & 0.00 & 0.42 \\ 
  Native American & 0.15 & 0.80 & 0.18 & 0.02 & 0.05 \\ 
  Black & 0.33 & 0.79 & 0.21 & 0.00 & 0.18 \\ 
  Other & 0.40 & 0.68 & 0.31 & 0.01 & 0.06 \\ 
  Total Pop & 0.26 & 0.73 & 0.27 & 0.00 &  \\ 
   \hline
\end{tabular}
\end{center}
\end{table}

% latex table generated in R 2.13.1 by xtable 1.6-0 package
% Thu Jan 26 11:22:22 2012
\begin{table}[htb]
\begin{center}
\caption{Mine Inspector 2010 LD 29 (Proposed)}
\label{smine_cvap_ld_29}
\begin{tabular}{lccccc}
  \hline
Racial Group & Turnout & D Vote & R Vote & S Vote & Total CVAP \\ 
  \hline
White & 0.26 & 0.42 & 0.58 & 0.00 & 0.41 \\ 
  Hispanic & 0.19 & 0.81 & 0.18 & 0.00 & 0.45 \\ 
  Native American & 0.60 & 0.55 & 0.43 & 0.02 & 0.01 \\ 
  Black & 0.42 & 0.65 & 0.34 & 0.01 & 0.08 \\ 
  Other & 0.44 & 0.65 & 0.34 & 0.01 & 0.05 \\ 
  Total Pop & 0.26 & 0.61 & 0.39 & 0.00 &  \\ 
   \hline
\end{tabular}
\end{center}
\end{table}

% latex table generated in R 2.13.1 by xtable 1.6-0 package
% Thu Jan 26 11:22:22 2012
\begin{table}[htb]
\begin{center}
\caption{Mine Inspector 2010 LD 30 (Proposed)}
\label{smine_cvap_ld_30}
\begin{tabular}{lccccc}
  \hline
Racial Group & Turnout & D Vote & R Vote & S Vote & Total CVAP \\ 
  \hline
White & 0.24 & 0.38 & 0.61 & 0.00 & 0.52 \\ 
  Hispanic & 0.21 & 0.86 & 0.14 & 0.00 & 0.32 \\ 
  Native American & 0.23 & 0.67 & 0.31 & 0.02 & 0.04 \\ 
  Black & 0.25 & 0.46 & 0.53 & 0.01 & 0.06 \\ 
  Other & 0.38 & 0.69 & 0.31 & 0.01 & 0.07 \\ 
  Total Pop & 0.24 & 0.56 & 0.44 & 0.00 &  \\ 
   \hline
\end{tabular}
\end{center}
\end{table}


\subsection{Benchmark Map}
\clearpage
\input{tables/smine_cvap_ld_2_benchmark}
% latex table generated in R 2.13.1 by xtable 1.6-0 package
% Thu Jan 26 11:22:24 2012
\begin{table}[htb]
\begin{center}
\caption{Mine Inspector 2010 LD 13 (Benchmark)}
\label{smine_cvap_ld_13_benchmark}
\begin{tabular}{lccccc}
  \hline
Racial Group & Turnout & D Vote & R Vote & S Vote & Total CVAP \\ 
  \hline
White & 0.20 & 0.52 & 0.48 & 0.00 & 0.35 \\ 
  Hispanic & 0.22 & 0.83 & 0.17 & 0.00 & 0.52 \\ 
  Native American & 0.52 & 0.65 & 0.34 & 0.01 & 0.02 \\ 
  Black & 0.46 & 0.60 & 0.40 & 0.01 & 0.07 \\ 
  Other & 0.55 & 0.58 & 0.41 & 0.01 & 0.05 \\ 
  Total Votes & 0.25 & 0.68 & 0.32 & 0.00 &  \\ 
   \hline
\end{tabular}
\end{center}
\end{table}

% latex table generated in R 2.13.1 by xtable 1.6-0 package
% Thu Jan 26 11:22:24 2012
\begin{table}[htb]
\begin{center}
\caption{Mine Inspector 2010 LD 14 (Benchmark)}
\label{smine_cvap_ld_14_benchmark}
\begin{tabular}{lccccc}
  \hline
Racial Group & Turnout & D Vote & R Vote & S Vote & Total CVAP \\ 
  \hline
White & 0.21 & 0.53 & 0.47 & 0.00 & 0.39 \\ 
  Hispanic & 0.19 & 0.88 & 0.12 & 0.00 & 0.44 \\ 
  Native American & 0.26 & 0.58 & 0.41 & 0.02 & 0.04 \\ 
  Black & 0.28 & 0.58 & 0.41 & 0.01 & 0.06 \\ 
  Other & 0.44 & 0.60 & 0.39 & 0.01 & 0.06 \\ 
  Total Votes & 0.22 & 0.68 & 0.32 & 0.00 &  \\ 
   \hline
\end{tabular}
\end{center}
\end{table}

% latex table generated in R 2.13.1 by xtable 1.6-0 package
% Thu Jan 26 11:22:24 2012
\begin{table}[htb]
\begin{center}
\caption{Mine Inspector 2010 LD 15 (Benchmark)}
\label{smine_cvap_ld_15_benchmark}
\begin{tabular}{lccccc}
  \hline
Racial Group & Turnout & D Vote & R Vote & S Vote & Total CVAP \\ 
  \hline
White & 0.30 & 0.65 & 0.35 & 0.00 & 0.59 \\ 
  Hispanic & 0.15 & 0.67 & 0.33 & 0.01 & 0.25 \\ 
  Native American & 0.31 & 0.59 & 0.40 & 0.02 & 0.04 \\ 
  Black & 0.34 & 0.53 & 0.46 & 0.01 & 0.06 \\ 
  Other & 0.56 & 0.50 & 0.49 & 0.01 & 0.06 \\ 
  Total Votes & 0.28 & 0.63 & 0.37 & 0.00 &  \\ 
   \hline
\end{tabular}
\end{center}
\end{table}

\input{tables/smine_cvap_ld_16_benchmark}
% latex table generated in R 2.13.1 by xtable 1.6-0 package
% Thu Jan 26 11:22:24 2012
\begin{table}[htb]
\begin{center}
\caption{Mine Inspector 2010 LD 23 (Benchmark)}
\label{smine_cvap_ld_23_benchmark}
\begin{tabular}{lccccc}
  \hline
Racial Group & Turnout & D Vote & R Vote & S Vote & Total CVAP \\ 
  \hline
White & 0.32 & 0.30 & 0.70 & 0.00 & 0.59 \\ 
  Hispanic & 0.22 & 0.80 & 0.20 & 0.00 & 0.25 \\ 
  Native American & 0.12 & 0.81 & 0.18 & 0.01 & 0.07 \\ 
  Black & 0.31 & 0.52 & 0.48 & 0.00 & 0.04 \\ 
  Other & 0.44 & 0.63 & 0.36 & 0.00 & 0.05 \\ 
  Total Votes & 0.29 & 0.45 & 0.55 & 0.00 &  \\ 
   \hline
\end{tabular}
\end{center}
\end{table}

% latex table generated in R 2.13.1 by xtable 1.6-0 package
% Thu Jan 26 11:22:24 2012
\begin{table}[htb]
\begin{center}
\caption{Mine Inspector 2010 LD 24 (Benchmark)}
\label{smine_cvap_ld_24_benchmark}
\begin{tabular}{lccccc}
  \hline
Racial Group & Turnout & D Vote & R Vote & S Vote & Total CVAP \\ 
  \hline
White & 0.29 & 0.14 & 0.86 & 0.00 & 0.56 \\ 
  Hispanic & 0.27 & 0.83 & 0.17 & 0.00 & 0.35 \\ 
  Native American & 0.32 & 0.63 & 0.36 & 0.01 & 0.02 \\ 
  Black & 0.53 & 0.57 & 0.42 & 0.00 & 0.02 \\ 
  Other & 0.54 & 0.67 & 0.32 & 0.00 & 0.04 \\ 
  Total Votes & 0.30 & 0.43 & 0.57 & 0.00 &  \\ 
   \hline
\end{tabular}
\end{center}
\end{table}

% latex table generated in R 2.13.1 by xtable 1.6-0 package
% Thu Jan 26 11:22:24 2012
\begin{table}[htb]
\begin{center}
\caption{Mine Inspector 2010 LD 25 (Benchmark)}
\label{smine_cvap_ld_25_benchmark}
\begin{tabular}{lccccc}
  \hline
Racial Group & Turnout & D Vote & R Vote & S Vote & Total CVAP \\ 
  \hline
White & 0.41 & 0.26 & 0.74 & 0.00 & 0.57 \\ 
  Hispanic & 0.32 & 0.89 & 0.11 & 0.00 & 0.31 \\ 
  Native American & 0.21 & 0.92 & 0.07 & 0.01 & 0.05 \\ 
  Black & 0.29 & 0.42 & 0.57 & 0.01 & 0.02 \\ 
  Other & 0.39 & 0.61 & 0.38 & 0.01 & 0.04 \\ 
  Total Votes & 0.37 & 0.47 & 0.53 & 0.00 &  \\ 
   \hline
\end{tabular}
\end{center}
\end{table}

\input{tables/smine_cvap_ld_27_benchmark}
\input{tables/smine_cvap_ld_29_benchmark}

%\section{VAP LD}

%\subsection{Proposed Map}
%\clearpage

%% latex table generated in R 2.13.1 by xtable 1.6-0 package
% Thu Jan 26 11:22:25 2012
\begin{table}[htb]
\begin{center}
\caption{Mine Inspector 2010 LD 2 (Proposed)}
\label{smine_vap_ld_2}
\begin{tabular}{lccccc}
  \hline
Racial Group & Turnout & D Vote & R Vote & S Vote & Total VAP \\ 
  \hline
White & 0.50 & 0.41 & 0.59 & 0.00 & 0.41 \\ 
  Hispanic & 0.17 & 0.90 & 0.10 & 0.00 & 0.53 \\ 
  Native American & 0.44 & 0.75 & 0.24 & 0.01 & 0.01 \\ 
  Black & 0.13 & 0.53 & 0.46 & 0.01 & 0.03 \\ 
  Other & 0.26 & 0.51 & 0.48 & 0.02 & 0.02 \\ 
  Total Pop & 0.31 & 0.57 & 0.43 & 0.00 &  \\ 
   \hline
\end{tabular}
\end{center}
\end{table}

%% latex table generated in R 2.13.1 by xtable 1.6-0 package
% Thu Jan 26 11:22:25 2012
\begin{table}[htb]
\begin{center}
\caption{Mine Inspector 2010 LD 3 (Proposed)}
\label{smine_vap_ld_3}
\begin{tabular}{lccccc}
  \hline
Racial Group & Turnout & D Vote & R Vote & S Vote & Total VAP \\ 
  \hline
White & 0.36 & 0.56 & 0.44 & 0.00 & 0.39 \\ 
  Hispanic & 0.23 & 0.86 & 0.14 & 0.00 & 0.50 \\ 
  Native American & 0.28 & 0.75 & 0.24 & 0.01 & 0.03 \\ 
  Black & 0.31 & 0.66 & 0.33 & 0.01 & 0.03 \\ 
  Other & 0.18 & 0.62 & 0.37 & 0.01 & 0.04 \\ 
  Total Pop & 0.28 & 0.69 & 0.31 & 0.00 &  \\ 
   \hline
\end{tabular}
\end{center}
\end{table}

%% latex table generated in R 2.13.1 by xtable 1.6-0 package
% Thu Jan 26 11:22:25 2012
\begin{table}[htb]
\begin{center}
\caption{Mine Inspector 2010 LD 4 (Proposed)}
\label{smine_vap_ld_4}
\begin{tabular}{lccccc}
  \hline
Racial Group & Turnout & D Vote & R Vote & S Vote & Total VAP \\ 
  \hline
White & 0.32 & 0.26 & 0.74 & 0.00 & 0.33 \\ 
  Hispanic & 0.14 & 0.87 & 0.12 & 0.00 & 0.56 \\ 
  Native American & 0.22 & 0.93 & 0.06 & 0.01 & 0.06 \\ 
  Black & 0.36 & 0.58 & 0.41 & 0.01 & 0.03 \\ 
  Other & 0.67 & 0.46 & 0.53 & 0.01 & 0.02 \\ 
  Total Pop & 0.23 & 0.54 & 0.45 & 0.00 &  \\ 
   \hline
\end{tabular}
\end{center}
\end{table}

%% latex table generated in R 2.13.1 by xtable 1.6-0 package
% Thu Jan 26 11:22:25 2012
\begin{table}[htb]
\begin{center}
\caption{Mine Inspector 2010 LD 7 (Proposed)}
\label{smine_vap_ld_7}
\begin{tabular}{lccccc}
  \hline
Racial Group & Turnout & D Vote & R Vote & S Vote & Total VAP \\ 
  \hline
White & 0.43 & 0.23 & 0.77 & 0.00 & 0.29 \\ 
  Hispanic & 0.21 & 0.61 & 0.38 & 0.01 & 0.06 \\ 
  Native American & 0.35 & 0.86 & 0.14 & 0.00 & 0.63 \\ 
  Black & 0.31 & 0.48 & 0.49 & 0.03 & 0.01 \\ 
  Other & 0.54 & 0.56 & 0.42 & 0.01 & 0.01 \\ 
  Total Pop & 0.37 & 0.63 & 0.37 & 0.00 &  \\ 
   \hline
\end{tabular}
\end{center}
\end{table}

%% latex table generated in R 2.13.1 by xtable 1.6-0 package
% Thu Jan 26 11:22:25 2012
\begin{table}[htb]
\begin{center}
\caption{Mine Inspector 2010 LD 8 (Proposed)}
\label{smine_vap_ld_8}
\begin{tabular}{lccccc}
  \hline
Racial Group & Turnout & D Vote & R Vote & S Vote & Total VAP \\ 
  \hline
White & 0.28 & 0.25 & 0.75 & 0.00 & 0.53 \\ 
  Hispanic & 0.24 & 0.86 & 0.14 & 0.00 & 0.31 \\ 
  Native American & 0.09 & 0.69 & 0.30 & 0.01 & 0.07 \\ 
  Black & 0.14 & 0.48 & 0.51 & 0.01 & 0.05 \\ 
  Other & 0.17 & 0.49 & 0.50 & 0.01 & 0.04 \\ 
  Total Pop & 0.25 & 0.46 & 0.54 & 0.00 &  \\ 
   \hline
\end{tabular}
\end{center}
\end{table}

%% latex table generated in R 2.13.1 by xtable 1.6-0 package
% Thu Jan 26 11:22:25 2012
\begin{table}[htb]
\begin{center}
\caption{Mine Inspector 2010 LD 19 (Proposed)}
\label{smine_vap_ld_19}
\begin{tabular}{lccccc}
  \hline
Racial Group & Turnout & D Vote & R Vote & S Vote & Total VAP \\ 
  \hline
White & 0.31 & 0.49 & 0.51 & 0.00 & 0.26 \\ 
  Hispanic & 0.08 & 0.84 & 0.16 & 0.00 & 0.60 \\ 
  Native American & 0.53 & 0.66 & 0.32 & 0.02 & 0.01 \\ 
  Black & 0.42 & 0.74 & 0.25 & 0.00 & 0.08 \\ 
  Other & 0.58 & 0.52 & 0.47 & 0.01 & 0.04 \\ 
  Total Pop & 0.19 & 0.63 & 0.37 & 0.00 &  \\ 
   \hline
\end{tabular}
\end{center}
\end{table}

%% latex table generated in R 2.13.1 by xtable 1.6-0 package
% Thu Jan 26 11:22:25 2012
\begin{table}[htb]
\begin{center}
\caption{Mine Inspector 2010 LD 24 (Proposed)}
\label{smine_vap_ld_24}
\begin{tabular}{lccccc}
  \hline
Racial Group & Turnout & D Vote & R Vote & S Vote & Total VAP \\ 
  \hline
White & 0.39 & 0.58 & 0.42 & 0.00 & 0.52 \\ 
  Hispanic & 0.07 & 0.71 & 0.28 & 0.01 & 0.34 \\ 
  Native American & 0.34 & 0.67 & 0.31 & 0.02 & 0.03 \\ 
  Black & 0.17 & 0.66 & 0.32 & 0.02 & 0.06 \\ 
  Other & 0.43 & 0.58 & 0.40 & 0.01 & 0.04 \\ 
  Total Pop & 0.27 & 0.60 & 0.40 & 0.00 &  \\ 
   \hline
\end{tabular}
\end{center}
\end{table}

%% latex table generated in R 2.13.1 by xtable 1.6-0 package
% Thu Jan 26 11:22:25 2012
\begin{table}[htb]
\begin{center}
\caption{Mine Inspector 2010 LD 26 (Proposed)}
\label{smine_vap_ld_26}
\begin{tabular}{lccccc}
  \hline
Racial Group & Turnout & D Vote & R Vote & S Vote & Total VAP \\ 
  \hline
White & 0.28 & 0.56 & 0.44 & 0.00 & 0.52 \\ 
  Hispanic & 0.07 & 0.47 & 0.52 & 0.01 & 0.32 \\ 
  Native American & 0.28 & 0.59 & 0.40 & 0.01 & 0.04 \\ 
  Black & 0.25 & 0.62 & 0.37 & 0.02 & 0.05 \\ 
  Other & 0.12 & 0.59 & 0.39 & 0.03 & 0.06 \\ 
  Total Pop & 0.20 & 0.56 & 0.44 & 0.00 &  \\ 
   \hline
\end{tabular}
\end{center}
\end{table}

%% latex table generated in R 2.13.1 by xtable 1.6-0 package
% Thu Jan 26 11:22:25 2012
\begin{table}[htb]
\begin{center}
\caption{Mine Inspector 2010 LD 27 (Proposed)}
\label{smine_vap_ld_27}
\begin{tabular}{lccccc}
  \hline
Racial Group & Turnout & D Vote & R Vote & S Vote & Total VAP \\ 
  \hline
White & 0.23 & 0.52 & 0.48 & 0.00 & 0.24 \\ 
  Hispanic & 0.13 & 0.88 & 0.12 & 0.00 & 0.52 \\ 
  Native American & 0.18 & 0.83 & 0.15 & 0.02 & 0.04 \\ 
  Black & 0.27 & 0.82 & 0.18 & 0.00 & 0.15 \\ 
  Other & 0.56 & 0.58 & 0.41 & 0.01 & 0.05 \\ 
  Total Pop & 0.20 & 0.73 & 0.27 & 0.00 &  \\ 
   \hline
\end{tabular}
\end{center}
\end{table}

%% latex table generated in R 2.13.1 by xtable 1.6-0 package
% Thu Jan 26 11:22:25 2012
\begin{table}[htb]
\begin{center}
\caption{Mine Inspector 2010 LD 29 (Proposed)}
\label{smine_vap_ld_29}
\begin{tabular}{lccccc}
  \hline
Racial Group & Turnout & D Vote & R Vote & S Vote & Total VAP \\ 
  \hline
White & 0.28 & 0.39 & 0.60 & 0.00 & 0.27 \\ 
  Hispanic & 0.09 & 0.89 & 0.11 & 0.00 & 0.62 \\ 
  Native American & 0.53 & 0.60 & 0.38 & 0.02 & 0.01 \\ 
  Black & 0.39 & 0.75 & 0.25 & 0.01 & 0.06 \\ 
  Other & 0.59 & 0.50 & 0.49 & 0.01 & 0.04 \\ 
  Total Pop & 0.18 & 0.61 & 0.39 & 0.00 &  \\ 
   \hline
\end{tabular}
\end{center}
\end{table}

%% latex table generated in R 2.13.1 by xtable 1.6-0 package
% Thu Jan 26 11:22:25 2012
\begin{table}[htb]
\begin{center}
\caption{Mine Inspector 2010 LD 30 (Proposed)}
\label{smine_vap_ld_30}
\begin{tabular}{lccccc}
  \hline
Racial Group & Turnout & D Vote & R Vote & S Vote & Total VAP \\ 
  \hline
White & 0.30 & 0.43 & 0.57 & 0.00 & 0.36 \\ 
  Hispanic & 0.09 & 0.83 & 0.16 & 0.00 & 0.51 \\ 
  Native American & 0.36 & 0.58 & 0.40 & 0.02 & 0.02 \\ 
  Black & 0.22 & 0.62 & 0.37 & 0.01 & 0.06 \\ 
  Other & 0.29 & 0.58 & 0.41 & 0.01 & 0.05 \\ 
  Total Pop & 0.19 & 0.56 & 0.44 & 0.00 &  \\ 
   \hline
\end{tabular}
\end{center}
\end{table}


%\subsection{Benchmark Map}
%\clearpage

%% latex table generated in R 2.13.1 by xtable 1.6-0 package
% Thu Jan 26 11:22:26 2012
\begin{table}[htb]
\begin{center}
\caption{Mine Inspector 2010 LD 2 (Benchmark)}
\label{smine_vap_ld_2_benchmark}
\begin{tabular}{lccccc}
  \hline
Racial Group & Turnout & D Vote & R Vote & S Vote & Total VAP \\ 
  \hline
White & 0.38 & 0.55 & 0.45 & 0.00 & 0.27 \\ 
  Hispanic & 0.18 & 0.52 & 0.48 & 0.01 & 0.05 \\ 
  Native American & 0.37 & 0.84 & 0.16 & 0.00 & 0.63 \\ 
  Black & 0.31 & 0.59 & 0.39 & 0.02 & 0.01 \\ 
  Other & 0.24 & 0.60 & 0.38 & 0.01 & 0.04 \\ 
  Total Votes & 0.35 & 0.74 & 0.26 & 0.00 &  \\ 
   \hline
\end{tabular}
\end{center}
\end{table}

%\input{tables/smine_vap_ld_13_benchmark}
%\input{tables/smine_vap_ld_14_benchmark}
%\input{tables/smine_vap_ld_15_benchmark}
%% latex table generated in R 2.13.1 by xtable 1.6-0 package
% Thu Jan 26 11:22:26 2012
\begin{table}[htb]
\begin{center}
\caption{Mine Inspector 2010 LD 16 (Benchmark)}
\label{smine_vap_ld_16_benchmark}
\begin{tabular}{lccccc}
  \hline
Racial Group & Turnout & D Vote & R Vote & S Vote & Total VAP \\ 
  \hline
White & 0.27 & 0.53 & 0.46 & 0.00 & 0.30 \\ 
  Hispanic & 0.22 & 0.86 & 0.14 & 0.00 & 0.45 \\ 
  Native American & 0.28 & 0.80 & 0.18 & 0.02 & 0.03 \\ 
  Black & 0.29 & 0.80 & 0.20 & 0.00 & 0.17 \\ 
  Other & 0.42 & 0.63 & 0.36 & 0.01 & 0.06 \\ 
  Total Votes & 0.26 & 0.73 & 0.27 & 0.00 &  \\ 
   \hline
\end{tabular}
\end{center}
\end{table}

%\input{tables/smine_vap_ld_23_benchmark}
%\input{tables/smine_vap_ld_24_benchmark}
%\input{tables/smine_vap_ld_25_benchmark}
%% latex table generated in R 2.13.1 by xtable 1.6-0 package
% Thu Jan 26 11:22:26 2012
\begin{table}[htb]
\begin{center}
\caption{Mine Inspector 2010 LD 27 (Benchmark)}
\label{smine_vap_ld_27_benchmark}
\begin{tabular}{lccccc}
  \hline
Racial Group & Turnout & D Vote & R Vote & S Vote & Total VAP \\ 
  \hline
White & 0.35 & 0.50 & 0.50 & 0.00 & 0.46 \\ 
  Hispanic & 0.32 & 0.87 & 0.13 & 0.00 & 0.42 \\ 
  Native American & 0.21 & 0.70 & 0.29 & 0.01 & 0.04 \\ 
  Black & 0.36 & 0.57 & 0.42 & 0.01 & 0.03 \\ 
  Other & 0.49 & 0.61 & 0.39 & 0.00 & 0.05 \\ 
  Total Votes & 0.34 & 0.66 & 0.34 & 0.00 &  \\ 
   \hline
\end{tabular}
\end{center}
\end{table}

%% latex table generated in R 2.13.1 by xtable 1.6-0 package
% Thu Jan 26 11:22:26 2012
\begin{table}[htb]
\begin{center}
\caption{Mine Inspector 2010 LD 29 (Benchmark)}
\label{smine_vap_ld_29_benchmark}
\begin{tabular}{lccccc}
  \hline
Racial Group & Turnout & D Vote & R Vote & S Vote & Total VAP \\ 
  \hline
White & 0.37 & 0.46 & 0.54 & 0.00 & 0.49 \\ 
  Hispanic & 0.26 & 0.92 & 0.08 & 0.00 & 0.37 \\ 
  Native American & 0.32 & 0.78 & 0.21 & 0.01 & 0.03 \\ 
  Black & 0.22 & 0.49 & 0.50 & 0.01 & 0.06 \\ 
  Other & 0.30 & 0.52 & 0.48 & 0.01 & 0.06 \\ 
  Total Votes & 0.32 & 0.62 & 0.38 & 0.00 &  \\ 
   \hline
\end{tabular}
\end{center}
\end{table}


\section{Performance Analysis}
\end{document}
