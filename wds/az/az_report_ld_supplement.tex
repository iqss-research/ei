\documentclass[12pt]{article}
%\usepackage{Rd,Sweave,upquote}
\usepackage[reqno]{amsmath}
\usepackage{natbib,amssymb,amsthm,graphicx,verbatim,url,verbatim,lscape}
\usepackage[all]{xy}
\usepackage{vmargin}
\setpapersize{USletter}
\topmargin=0in
\usepackage{times}
\usepackage{dcolumn, booktabs}
\usepackage[dvipsnames,usenames]{color}
\usepackage{dcolumn}
\newcolumntype{.}{D{.}{.}{-1}}\newcolumntype{d}[1]{D{.}{.}{#1}}
%\usepackage[notref]{showkeys}
%\usepackage{endfloat}

\usepackage{color,setspace}
\definecolor{spot}{rgb}{0.6,0,0}
\usepackage[pdftex, bookmarksopen=true, bookmarksnumbered=true,
  pdfstartview=FitH, breaklinks=true, urlbordercolor={0 1 0},
  citebordercolor={0 0 1}, colorlinks=true, citecolor=spot, 
  linkcolor=spot, urlcolor=spot,
  pdfauthor={Gary King, Benjamin Schneer},
  pdftitle={}]{hyperref}

%squeeze
% == spacing between sections and subsections
%\usepackage[compact]{titlesec} 
% keep floats on same page
\renewcommand{\topfraction}{0.85}
\renewcommand{\textfraction}{0.1}
\renewcommand{\floatpagefraction}{0.75} % keep < \topfraction

\usepackage{soul}

\usepackage{appendix}
\usepackage{latexsym}
\newtheorem{claim}{Claim}
\newtheorem{proposition}{Proposition}
\newtheorem{theorem}{Theorem}
\newtheorem{lemma}{Lemma}
\newtheorem*{ass}{Assumption}
\newtheorem{question}{Question}
\newtheorem{remark}{Remark}
\newtheorem{definition}{Definition}
\newcommand{\bX}{\boldsymbol{X}}

\title{Supplement to: ``Analysis of the Arizona Independent
  Redistricting Commission Legislative District Map: Candidates of
  Choice''\thanks{A web appendix to this report can be found at
    \url{j.mp/az-doj}}} \author{Gary King\thanks{Alfred J.
    Weatherhead III University Professor, Harvard University,
    Institute for Quantitative Social Science, 1737 Cambridge Street,
    Cambridge MA 02138; http://GKing.harvard.edu, king@harvard.edu,
    (617) 500-7570.}  \and Benjamin Schneer\thanks{Ph.D.\ student,
    Institute for Quantitative Social Science, 1737 Cambridge Street,
    Harvard University, Cambridge MA 02138,
    bschneer@fas.harvard.edu.}}

\begin{document}
\maketitle

\doublespacing

\section{Introduction}

This supplement to our report on the Arizona Independent Redistricting
Commission Legislative District Map discusses the candidates of choice
in the benchmark legislative districts, as well as in several other
legislative districts of interest. A number of documents submitted in
support of the Commission's legislative district map make reference to
the ``candidate of choice'' in a district. Here we show how we use
ecological inference in an effort to determine the identities of the
candidates of choice, with methodological details of the estimation
procedure given in the main report. While the evidence indicates that
in many cases the minority community in a district prefers a minority
candidate, exceptions to this pattern often exist. An added
complication is determining the candidate of choice in elections where
no minority candidates are on the ballot.

Here, we analyze five state-wide races: US President 2004, Secretary
of State 2006, US President 2008, Secretary of State 2010, and State
Mine Inspector 2010. In four of these five races, there is a candidate
who belongs to a racial minority group.  The exception is the election
for US President in 2004, where Democrat John Kerry faced Republican
George W. Bush, and neither of whom belonged to an ethnic minority
group.\footnote{For this election and the ones that follow we do not
  focus on the race of third party candidates, however we do account
  for their vote shares in our estimation procedure.} In the Arizona
Secretary of State 2006 election, Democrat Israel Torres, who is
Hispanic, faced Republican Jan Brewer, who does not belong to a
minority group.  In the US President 2008 election, Democrat Barack
Obama, an African American, faced Republican John McCain, who does not
belong to a minority group.  In the Secretary of State 2010 election,
Democrat Chris Deschene, a Native American, faced Republican Ken
Bennet, who does not belong to a minority group.  Finally, in the
Arizona State Mine Inspector 2010 election, Democrat Manuel Cruz, who
is Hispanic, faced Joe Hart, who does not belong to a minority group.

Where possible, we also include results from the Arizona State Senate
2008 election, Arizona State Senate 2010 election, and other
significant local elections. In these races, the candidates vary by
legislative district. In cases where an election was uncontested---a
scenario that occurs relatively frequently in Arizona State Senate
elections---we do not attempt to estimate the level of support among
the district's minority groups because voters were given no choice.

Rather than offer any sweeping conclusions about a typical candidate
of choice in the benchmark Legislative Districts, we hope that the
analysis presented in this supplement can serve as a reference point
for questions that might arise when examining other submitted
documents.

\section{Candidates of Choice in Benchmark Legislative Districts}

This section presents a district by district analysis of candidates of
choice in Arizona. The supporting tables for each district are found
in the Appendix to this document.

\subsection{LD 2}
Native American voters, who comprise the primary minority group in
benchmark LD 2, strongly preferred the Democratic candidates across
all five statewide races. In fact, we estimate that in each of the
statewide races the vote share for the Democratic candidate was above
75\%. In the Secretary of State 2010 election, which included a
Democratic candidate of Native American descent, we estimate that
Native American voters voted at a rate of 92\% for the Democratic
candidate.

In terms of local congressional elections in benchmark LD 2, we
estimate that Native American voters also strongly preferred the
Democratic candidate for State Senator 2008. This election occurred
between two candidates of Native American descent: Native American
Democrat Albert Hale and Native American Republican Royce Jenkins.
In 2010, the State Senate election was uncontested and a Democratic
candidate of Native American descent was elected.

\subsection{LD 12}

LD 12 is not a benchmark district; however, we include a brief
analysis of its candidates of choice because it is mentioned in other
submitted documents.  Hispanic voters in benchmark LD 12 slightly
preferred the Democratic candidates across all five statewide races.
In the US President 2004 and Secretary of State 2006 elections, we
estimate that Hispanics voted for the Democratic candidate at a rate
of 53\%.  In contrast, in the other three races we examine, we
estimate that Hispanics voted for the Democratic candidate at a rate
of at least 67\%. Based on these results, Hispanic voters in LD 12
appear to prefer Democratic candidates.

The State Senate 2008 election is also of interest. The Democratic
candidate, Angela Cotera, was Hispanic. As
Table~\ref{stsen08_cvap_ld_12_benchmark} demonstrates, Hispanic voters
preferred Cotera by a substantial margin, voting for her at a rate of
73\%. However, Cotera did not win the election. The State Senate 2010
election did not include a candidate who belonged to a minority group.
Nonetheless, Hispanic voters still preferred the Democratic candidate,
voting for him at a rate of 65\%.

We also examine the State House 2008 election. Two candidates in this
election belonged to minority groups: Hispanic Republican Steve
Montenegro and Hispanic Democrat Eve Nunez. Examining House elections
is problematic from the perspective of making inferences about voting
behavior of minority groups because each district elects two
candidates (and voters cast two votes). In this case, we allot the
population two votes in our estimation procedure; results in
Table~\ref{sthse08_cvap_ld_12} are accompanied by substantial
uncertainties and we present them to provide a rough estimate of
minority group voting behavior. As the table illustrates, we estimate
that Hispanic voters in the district preferred the Democratic Hispanic
Eve Nunez compared to Republican Hispanic Steve Montenegro; we
estimate that 34\% of all votes cast by Hispanics were for Nunez while
16\% of all votes cast by Hispanics were for Montenegro. In fact, we
estimate that Hispanic voters preferred the white Democratic candidate
to Montenegro.
 
\subsection{LD 13}
Hispanic voters in benchmark LD 13 strongly preferred Democratic
candidates across all five statewide races. In fact, we estimate that
in each of the statewide races the vote share for the Democratic
candidate was above 64\%. In terms of local congressional elections in
benchmark LD 13, both the 2008 and 2010 State Senate elections were
uncontested. In each case, a Hispanic Democrat was elected to the
State Senate.

\subsection{LD 14}
Hispanic voters in benchmark LD 14 strongly preferred the Democratic
candidates across all five statewide races. In fact, we estimate that
in each of the statewide races the vote share for the Democratic
candidate was above 76\%. In terms of local congressional elections in
benchmark LD 14, both the 2008 and 2010 State Senate elections were
uncontested. In each case, a Democratic candidate was elected to the
State Senate. In 2010 the unopposed candidate, Robert Meza, was
Hispanic.

\subsection{LD 15}
Hispanic voters in benchmark LD 15 slightly preferred Democratic
candidates. In the Secretary of State 2006 election, we estimate that
only 51\% of Hispanic voters preferred the Democratic Candidate. In
the US President 2004 election, we estimate that only 53\% of Hispanic
voters preferred the Democratic Candidate. These are not large enough
margins to give us great confidence that the Democrat is the candidate
of choice; however, in the US President 2008, Mine Inspector 2010, and
Secretary of State 2010 elections, the Democratic candidate received
over 60\% of the vote among Hispanic voters. Thus, overall, Democrats
appear to be the candidates of choice among voters in benchmark LD 15.
In terms of local congressional elections in benchmark LD 15, Hispanic
voters preferred (at a rate of 60\%) the Democratic candidate in the
2008 State Senate election; in 2010 the State Senate election was
uncontested.

\subsection{LD 16}
Hispanic voters in benchmark LD 16 strongly preferred the Democratic
candidates across all five statewide races. In fact, we estimate that
in each of the statewide races the vote share for the Democratic
candidate was above 78\%. In terms of local congressional elections in
benchmark LD 16, Hispanic voters strongly preferred (at a rate of
83\%) the Democratic candidate (an African-American Leah Landrum) in
the 2008 State Senate election; in 2010 the State Senate election was
uncontested and Leah Landrum won re-election.

\subsection{LD 17}

LD 17 is not a benchmark district; however, we include a brief
analysis of its candidates of choice because it is mentioned in other
submitted documents.  Hispanic voters in benchmark LD 17 preferred the
Democratic candidates in three of the five five statewide races that
we examined. In the US President 2004 and Secretary of State 2006
elections, we estimate that Hispanics supported the Republican
candidates, neither of whom belonged to a racial minority group.
However, in the other three races we examine, we estimate that
Hispanics voted for the Democratic candidate at a rate of at least
57\%. Based on these results, we cannot generalize about who the
candidate of choice might be for Hispanic voters in the district.

We also examine State Senator elections. LD 17 is of particular
interest because Hispanic Republican Jesse Hernandez ran for office in
both 2004 and 2008. Interestingly, we estimate that Hernandez was not
the clear candidate of choice among Hispanic voters in either 2004 or
2008. Tables~\ref{stsen04_cvap_ld_17} and
\ref{stsen08_cvap_ld_17_benchmark} suggest that Hernandez received
less than half of the vote share among Hispanic voters in both
elections.  This comes with the caveat that these estimates are
particularly uncertain, as is clear from even an examination of the
tomography plots displaying the Hispanic vote for Hispanic candidate
included in our web appendix.

The State Senate 2010 election did not include a minority candidate;
however, we estimate that Hispanic voters still strongly preferred the
Democratic candidate, as Table~\ref{stsen10_cvap_ld_17_benchmark}
illustrates.

\subsection{LD 23}

Hispanic voters in benchmark LD 23 strongly preferred the Democratic
candidates across all five statewide races. In fact, we estimate that
in each of the statewide races the vote share for the Democratic
candidate was above 63\%.

In terms of local congressional elections in benchmark LD 23, Hispanic
voters strongly preferred (at a rate of 82\%) the Democratic candidate
in the 2008 State Senate election. This election pitted two candidates
of Hispanic ethnicity against each other: the Hispanic Democrat
Rebecca Rios and the Hispanic Republican Andre Campos. Nonetheless,
Hispanic voters still strongly preferred the Democratic candidate.
The 2010 State Senate election pitted the Hispanic Democrat Rebecca
Rios against a Republican candidate who did not belong to a minority
group. Hispanic voters in the district preferred the Democratic State
Senate candidate (at a rate of 79\%); however, this support did not
prove to be enough to lead to re-election for Rios.

\subsection{LD 24}

Hispanic voters in benchmark LD 24 strongly preferred the Democratic
candidates across all five statewide races. In fact, we estimate that
in each of the statewide races the vote share for the Democratic
candidate was above 63\%.

In terms of local congressional elections in benchmark LD 24, Hispanic
voters strongly preferred (at a rate of 82\%) the Hispanic Democratic
candidate, Amanda Aguirre, in the 2010 State Senate election; however,
LD this support did not prove to enough to elect Aguirre. In 2008 the
State Senate election was uncontested and Aguirre won the seat.

\subsection{LD 25}

Hispanic voters in benchmark LD 25 strongly preferred Democratic
candidates across all five statewide races.  We estimate that in each
of the statewide races the vote share for the Democratic candidate was
above 65\%.

In terms of local congressional elections in benchmark LD 25, Hispanic
voters strongly preferred (at a rate of 89\%) the Democratic candidate
in the 2008 State Senate election; in 2010 Hispanic voters preferred
the Democratic State Senate candidate (at a rate of 89\%). In each of
these races, the Hispanic Democrat Manuel Alvarez prevailed over a
Republican candidate who did not belong to a minority group.

\subsection{LD 27}

Hispanic voters in benchmark LD 27 strongly preferred the Democratic
candidates across all five statewide races. In fact, we estimate that
in each of the statewide races the vote share for the Democratic
candidate was above 77\%.

In terms of local congressional elections in benchmark LD 27, 89\% of
Hispanic voters cast their ballots for the Democratic State Senate
candidate Jorge Luis Garcia, who is Hispanic. Garcia ran against a
Republican candidate who did not belong to a minority group. In 2010,
the Hispanic Democrat Oliv Cajero Bedford won the seat, which was
uncontested.

\subsection{LD 29}

Hispanic voters in benchmark LD 29 strongly preferred the Democratic
candidates across all five statewide races. In fact, we estimate that
in each of the statewide races the vote share for the Democratic
candidate was above 77\%.

In terms of local congressional elections in benchmark LD 29, both the
2008 and 2010 State Senate elections were uncontested. In each case,
the Hispanic Democrat Linda Lopez won the seat.

\clearpage
\appendix

\section{Appendix: CVAP-Based Ecological Inference Estimates for Benchmark Map}

\subsection{LD 2}

\input{figs/smine_cvap_ld_2_benchmark}
\input{figs/pres04_cvap_ld_2_benchmark}
% latex table generated in R 2.13.1 by xtable 1.6-0 package
% Thu Jan 26 11:22:24 2012
\begin{table}[htb]
\begin{center}
\caption{US President 2008 LD 2 (Benchmark)}
\label{pres08_cvap_ld_2_benchmark}
\begin{tabular}{lccccc}
  \hline
Racial Group & Turnout & D Vote & R Vote & S Vote & Total CVAP \\ 
  \hline
White & 0.53 & 0.60 & 0.39 & 0.01 & 0.26 \\ 
  Hispanic & 0.45 & 0.53 & 0.43 & 0.04 & 0.05 \\ 
  Native American & 0.45 & 0.81 & 0.18 & 0.01 & 0.64 \\ 
  Black & 0.59 & 0.46 & 0.46 & 0.09 & 0.01 \\ 
  Other & 0.56 & 0.67 & 0.26 & 0.06 & 0.05 \\ 
  Total Votes & 0.47 & 0.73 & 0.26 & 0.01 &  \\ 
   \hline
\end{tabular}
\end{center}
\end{table}

\input{figs/sos06_cvap_ld_2_benchmark}
% latex table generated in R 2.13.1 by xtable 1.6-0 package
% Thu Jan 26 11:22:24 2012
\begin{table}[htb]
\begin{center}
\caption{Secretary of State 2010 LD 2 (Benchmark)}
\label{sos10_cvap_ld_2_benchmark}
\begin{tabular}{lccccc}
  \hline
Racial Group & Turnout & D Vote & R Vote & S Vote & Total CVAP \\ 
  \hline
White & 0.39 & 0.56 & 0.44 & 0.00 & 0.27 \\ 
  Hispanic & 0.22 & 0.45 & 0.54 & 0.01 & 0.05 \\ 
  Native American & 0.37 & 0.92 & 0.07 & 0.00 & 0.63 \\ 
  Black & 0.30 & 0.50 & 0.48 & 0.02 & 0.01 \\ 
  Other & 0.32 & 0.55 & 0.44 & 0.01 & 0.04 \\ 
  Total Votes & 0.37 & 0.79 & 0.21 & 0.00 &  \\ 
   \hline
\end{tabular}
\end{center}
\end{table}


% latex table generated in R 2.13.1 by xtable 1.6-0 package
% Thu Jan 26 11:22:20 2012
\begin{table}[htb]
\begin{center}
\caption{State Senator 2008 LD 2 (Benchmark)}
\label{stsen08_cvap_ld_2_benchmark}
\begin{tabular}{lccccc}
  \hline
Racial Group & Turnout & D Vote & R Vote & S Vote & Total CVAP \\ 
  \hline
	
    White & 0.50  & 0.59  & 0.41  & 0.00 & 0.27 \\
    Hispanic & 0.39  & 0.48  & 0.51  & 0.00 & 0.05 \\
    Native American & 0.44  & 0.84  & 0.16  & 0.00 & 0.63\\
    Black & 0.59  & 0.57  & 0.42  & 0.01 & 0.01 \\
    Other & 0.49  & 0.57  & 0.42  & 0.01 & 0.05\\

    Total Pop & 0.46  & 0.74  & 0.26  & 0.00  & NA \\
   \hline
\end{tabular}
\end{center}
\end{table}

\clearpage

\subsection{LD 12}

% latex table generated in R 2.13.1 by xtable 1.6-0 package
% Thu Jan 26 11:22:24 2012
\begin{table}[htb]
\begin{center}
\caption{Mine Inspector 2010 LD 12 (Benchmark)}
\label{smine_cvap_ld_12_benchmark}
\begin{tabular}{lccccc}
  \hline
Racial Group & Turnout & D Vote & R Vote & S Vote & Total CVAP \\ 
  \hline
White & 0.29  & 0.25  & 0.75  & 0.00  & 0.60 \\
    Hispanic & 0.29  & 0.75  & 0.25  & 0.00  & 0.26 \\
    Native American & 0.50  & 0.43  & 0.54  & 0.02  & 0.01 \\
    Black & 0.37  & 0.55  & 0.44  & 0.01  & 0.07 \\
    Other & 0.63  & 0.53  & 0.47  & 0.01  & 0.06 \\
    Total Votes & 0.32  & 0.43  & 0.57  & 0.00  &  NA \\

   \hline
\end{tabular}
\end{center}
\end{table}

% latex table generated in R 2.13.1 by xtable 1.6-0 package
% Thu Jan 26 11:22:24 2012
\begin{table}[htb]
\begin{center}
\caption{US President 2004 LD 12 (Benchmark)}
\label{pres04_cvap_ld_12_benchmark}
\begin{tabular}{lccccc}
  \hline
Racial Group & Turnout & D Vote & R Vote & S Vote & Total CVAP \\ 
  \hline
White & 0.36  & 0.36  & 0.64  & 0.00  & 0.61 \\
    Hispanic & 0.32  & 0.53  & 0.47  & 0.01  & 0.26 \\
    Native American & 0.57  & 0.55  & 0.41  & 0.04  & 0.01 \\
    Black & 0.51  & 0.48  & 0.50  & 0.01  & 0.07 \\
    Other & 0.37  & 0.29  & 0.69  & 0.02  & 0.06 \\
    Total Votes & 0.36  & 0.41  & 0.59  & 0.01  &  NA \\
   \hline
\end{tabular}
\end{center}
\end{table}

% latex table generated in R 2.13.1 by xtable 1.6-0 package
% Thu Jan 26 11:22:24 2012
\begin{table}[htb]
\begin{center}
\caption{US President 2008 LD 12 (Benchmark)}
\label{pres08_cvap_ld_12_benchmark}
\begin{tabular}{lccccc}
  \hline
Racial Group & Turnout & D Vote & R Vote & S Vote & Total CVAP \\ 
  \hline
White & 0.42  & 0.30  & 0.69  & 0.01  & 0.60 \\
    Hispanic & 0.44  & 0.67  & 0.31  & 0.01  & 0.26 \\
    Native American & 0.57  & 0.39  & 0.49  & 0.11  & 0.01 \\
    Black & 0.82  & 0.58  & 0.39  & 0.02  & 0.07 \\
    Other & 0.86  & 0.57  & 0.39  & 0.04  & 0.06 \\
    Total Votes & 0.48  & 0.45  & 0.53  & 0.01  &  NA \\
   \hline
\end{tabular}
\end{center}
\end{table}

% latex table generated in R 2.13.1 by xtable 1.6-0 package
% Thu Jan 26 11:22:24 2012
\begin{table}[htb]
\begin{center}
\caption{Secretary of State 2006 LD 12 (Benchmark)}
\label{sos06_cvap_ld_12_benchmark}
\begin{tabular}{lccccc}
  \hline
Racial Group & Turnout & D Vote & R Vote & S Vote & Total CVAP \\ 
  \hline
    White & 0.26  & 0.29  & 0.70  & 0.01  & 0.61 \\
    Hispanic & 0.16  & 0.53  & 0.42  & 0.05  & 0.26 \\
    Native American & 0.51  & 0.41  & 0.47  & 0.12  & 0.01 \\
    Black & 0.47  & 0.44  & 0.50  & 0.07  & 0.07 \\
    Other & 0.47  & 0.42  & 0.50  & 0.08  & 0.06 \\
    Total Votes & 0.27  & 0.36  & 0.60  & 0.03  &  NA \\
   \hline
\end{tabular}
\end{center}
\end{table}

% latex table generated in R 2.13.1 by xtable 1.6-0 package
% Thu Jan 26 11:22:24 2012
\begin{table}[htb]
\begin{center}
\caption{Secretary of State 2010 LD 12 (Benchmark)}
\label{sos10_cvap_ld_12_benchmark}
\begin{tabular}{lccccc}
  \hline
Racial Group & Turnout & D Vote & R Vote & S Vote & Total CVAP \\ 
  \hline
White & 0.30  & 0.24  & 0.76  & 0.00  & 0.60 \\
    Hispanic & 0.29  & 0.69  & 0.31  & 0.00  & 0.26 \\
    Native American & 0.45  & 0.48  & 0.50  & 0.02  & 0.01 \\
    Black & 0.40  & 0.57  & 0.43  & 0.01  & 0.07 \\
    Other & 0.73  & 0.52  & 0.47  & 0.00  & 0.06 \\
    Total Votes & 0.33  & 0.41  & 0.59  & 0.00  &  NA \\
   \hline
\end{tabular}
\end{center}
\end{table}


\clearpage
\begin{landscape}
% latex table generated in R 2.13.1 by xtable 1.6-0 package
% Thu Jan 26 11:22:20 2012
\begin{table}[!h]
\begin{center}
\caption{State House 2008 LD 12 (Benchmark)}
\label{sthse08_cvap_ld_12}
\begin{tabular}{lccccccc}
  \hline
Racial Group & Turnout & R Vote 1 (W) & R Vote 2 (H) & D Vote 1 (W) & D Vote 2 (H) & S Vote & Total CVAP \\ 
  \hline
 White & 0.36  & 0.33  & 0.31  & 0.17  & 0.17  & 0.01  & 0.60 \\
    Hispanic & 0.31  & 0.16  & 0.19  & 0.27  & 0.34  & 0.04  & 0.26 \\
    Native American & 0.61  & 0.22  & 0.19  & 0.23  & 0.25  & 0.11  & 0.01 \\
    Black & 0.73  & 0.20  & 0.17  & 0.29  & 0.28  & 0.06  & 0.07 \\
    Other & 0.89  & 0.20  & 0.22  & 0.25  & 0.25  & 0.08  & 0.06 \\
    Total Pop & 0.40  & 0.26  & 0.26  & 0.22  & 0.23  & 0.04  & NA \\
   \hline
\end{tabular}
\end{center}
\end{table}
\end{landscape}
\clearpage

% latex table generated in R 2.13.1 by xtable 1.6-0 package
% Thu Jan 26 11:22:20 2012
\begin{table}[htb]
\begin{center}
\caption{State Senator 2008 LD 12 (Benchmark)}
\label{stsen08_cvap_ld_12_benchmark}
\begin{tabular}{lccccc}
  \hline
Racial Group & Turnout & D Vote & R Vote & S Vote & Total CVAP \\ 
  \hline
	White & 0.38  & 0.32  & 0.68  & 0.00  & 0.60 \\
    Hispanic & 0.43  & 0.73  & 0.26  & 0.00  & 0.26 \\
    Native American & 0.52  & 0.50  & 0.48  & 0.02  & 0.01 \\
    Black & 0.75  & 0.54  & 0.45  & 0.00  & 0.07 \\
    Other & 0.86  & 0.60  & 0.39  & 0.00  & 0.06 \\
    Total Votes & 0.45  & 0.48  & 0.51  & 0.00  &  NA \\

   \hline
\end{tabular}
\end{center}
\end{table}
% latex table generated in R 2.13.1 by xtable 1.6-0 package
% Thu Jan 26 11:22:20 2012
\begin{table}[htb]
\begin{center}
\caption{State Senator 2010 LD 12 (Benchmark)}
\label{stsen10_cvap_ld_12_benchmark}
\begin{tabular}{lccccc}
  \hline
Racial Group & Turnout & D Vote & R Vote & S Vote & Total CVAP \\ 
  \hline
    White & 0.31  & 0.26  & 0.72  & 0.02  & 0.60 \\
    Hispanic & 0.24  & 0.65  & 0.28  & 0.07  & 0.26 \\
    Native American & 0.50  & 0.49  & 0.36  & 0.15  & 0.01 \\
    Black & 0.46  & 0.47  & 0.42  & 0.11  & 0.07 \\
    Other & 0.74  & 0.49  & 0.40  & 0.10  & 0.06 \\
    Total Pop & 0.33  & 0.39  & 0.56  & 0.05  & NA \\
   \hline
\end{tabular}
\end{center}
\end{table}
\clearpage


\subsection{LD 13}

% latex table generated in R 2.13.1 by xtable 1.6-0 package
% Thu Jan 26 11:22:24 2012
\begin{table}[htb]
\begin{center}
\caption{Mine Inspector 2010 LD 13 (Benchmark)}
\label{smine_cvap_ld_13_benchmark}
\begin{tabular}{lccccc}
  \hline
Racial Group & Turnout & D Vote & R Vote & S Vote & Total CVAP \\ 
  \hline
White & 0.20 & 0.52 & 0.48 & 0.00 & 0.35 \\ 
  Hispanic & 0.22 & 0.83 & 0.17 & 0.00 & 0.52 \\ 
  Native American & 0.52 & 0.65 & 0.34 & 0.01 & 0.02 \\ 
  Black & 0.46 & 0.60 & 0.40 & 0.01 & 0.07 \\ 
  Other & 0.55 & 0.58 & 0.41 & 0.01 & 0.05 \\ 
  Total Votes & 0.25 & 0.68 & 0.32 & 0.00 &  \\ 
   \hline
\end{tabular}
\end{center}
\end{table}

% latex table generated in R 2.13.1 by xtable 1.6-0 package
% Thu Jan 26 11:22:24 2012
\begin{table}[htb]
\begin{center}
\caption{US President 2004 LD 13 (Benchmark)}
\label{pres04_cvap_ld_13_benchmark}
\begin{tabular}{lccccc}
  \hline
Racial Group & Turnout & D Vote & R Vote & S Vote & Total CVAP \\ 
  \hline
White & 0.31 & 0.40 & 0.59 & 0.01 & 0.35 \\ 
  Hispanic & 0.33 & 0.69 & 0.31 & 0.00 & 0.52 \\ 
  Native American & 0.50 & 0.54 & 0.42 & 0.04 & 0.02 \\ 
  Black & 0.47 & 0.46 & 0.51 & 0.02 & 0.07 \\ 
  Other & 0.36 & 0.51 & 0.46 & 0.03 & 0.05 \\ 
  Total Votes & 0.33 & 0.56 & 0.43 & 0.01 &  \\ 
   \hline
\end{tabular}
\end{center}
\end{table}

\input{figs/pres08_cvap_ld_13_benchmark}
% latex table generated in R 2.13.1 by xtable 1.6-0 package
% Thu Jan 26 11:22:24 2012
\begin{table}[htb]
\begin{center}
\caption{Secretary of State 2006 LD 13 (Benchmark)}
\label{sos06_cvap_ld_13_benchmark}
\begin{tabular}{lccccc}
  \hline
Racial Group & Turnout & D Vote & R Vote & S Vote & Total CVAP \\ 
  \hline
White & 0.19 & 0.43 & 0.54 & 0.03 & 0.35 \\ 
  Hispanic & 0.20 & 0.64 & 0.33 & 0.02 & 0.52 \\ 
  Native American & 0.31 & 0.44 & 0.44 & 0.12 & 0.02 \\ 
  Black & 0.32 & 0.46 & 0.45 & 0.09 & 0.07 \\ 
  Other & 0.26 & 0.41 & 0.44 & 0.14 & 0.05 \\ 
  Total Votes & 0.21 & 0.54 & 0.42 & 0.04 &  \\ 
   \hline
\end{tabular}
\end{center}
\end{table}

% latex table generated in R 2.13.1 by xtable 1.6-0 package
% Thu Jan 26 11:22:24 2012
\begin{table}[htb]
\begin{center}
\caption{Secretary of State 2010 LD 13 (Benchmark)}
\label{sos10_cvap_ld_13_benchmark}
\begin{tabular}{lccccc}
  \hline
Racial Group & Turnout & D Vote & R Vote & S Vote & Total CVAP \\ 
  \hline
White & 0.21 & 0.50 & 0.50 & 0.00 & 0.35 \\ 
  Hispanic & 0.22 & 0.78 & 0.22 & 0.00 & 0.52 \\ 
  Native American & 0.52 & 0.62 & 0.36 & 0.02 & 0.02 \\ 
  Black & 0.48 & 0.62 & 0.37 & 0.01 & 0.07 \\ 
  Other & 0.56 & 0.62 & 0.37 & 0.01 & 0.05 \\ 
  Total Votes & 0.25 & 0.66 & 0.34 & 0.00 &  \\ 
   \hline
\end{tabular}
\end{center}
\end{table}


\clearpage

\subsection{LD 14}

% latex table generated in R 2.13.1 by xtable 1.6-0 package
% Thu Jan 26 11:22:24 2012
\begin{table}[htb]
\begin{center}
\caption{Mine Inspector 2010 LD 14 (Benchmark)}
\label{smine_cvap_ld_14_benchmark}
\begin{tabular}{lccccc}
  \hline
Racial Group & Turnout & D Vote & R Vote & S Vote & Total CVAP \\ 
  \hline
White & 0.21 & 0.53 & 0.47 & 0.00 & 0.39 \\ 
  Hispanic & 0.19 & 0.88 & 0.12 & 0.00 & 0.44 \\ 
  Native American & 0.26 & 0.58 & 0.41 & 0.02 & 0.04 \\ 
  Black & 0.28 & 0.58 & 0.41 & 0.01 & 0.06 \\ 
  Other & 0.44 & 0.60 & 0.39 & 0.01 & 0.06 \\ 
  Total Votes & 0.22 & 0.68 & 0.32 & 0.00 &  \\ 
   \hline
\end{tabular}
\end{center}
\end{table}

\input{figs/pres04_cvap_ld_14_benchmark}
% latex table generated in R 2.13.1 by xtable 1.6-0 package
% Thu Jan 26 11:22:24 2012
\begin{table}[htb]
\begin{center}
\caption{US President 2008 LD 14 (Benchmark)}
\label{pres08_cvap_ld_14_benchmark}
\begin{tabular}{lccccc}
  \hline
Racial Group & Turnout & D Vote & R Vote & S Vote & Total CVAP \\ 
  \hline
White & 0.38 & 0.52 & 0.47 & 0.01 & 0.39 \\ 
  Hispanic & 0.25 & 0.84 & 0.15 & 0.01 & 0.44 \\ 
  Native American & 0.33 & 0.54 & 0.41 & 0.05 & 0.04 \\ 
  Black & 0.35 & 0.56 & 0.38 & 0.05 & 0.06 \\ 
  Other & 0.50 & 0.47 & 0.48 & 0.05 & 0.06 \\ 
  Total Votes & 0.32 & 0.63 & 0.35 & 0.02 &  \\ 
   \hline
\end{tabular}
\end{center}
\end{table}

% latex table generated in R 2.13.1 by xtable 1.6-0 package
% Thu Jan 26 11:22:24 2012
\begin{table}[htb]
\begin{center}
\caption{Secretary of State 2006 LD 14 (Benchmark)}
\label{sos06_cvap_ld_14_benchmark}
\begin{tabular}{lccccc}
  \hline
Racial Group & Turnout & D Vote & R Vote & S Vote & Total CVAP \\ 
  \hline
White & 0.27 & 0.48 & 0.50 & 0.02 & 0.38 \\ 
  Hispanic & 0.13 & 0.76 & 0.21 & 0.04 & 0.44 \\ 
  Native American & 0.28 & 0.52 & 0.41 & 0.07 & 0.04 \\ 
  Black & 0.20 & 0.47 & 0.40 & 0.14 & 0.06 \\ 
  Other & 0.28 & 0.39 & 0.49 & 0.12 & 0.06 \\ 
  Total Votes & 0.21 & 0.55 & 0.40 & 0.04 &  \\ 
   \hline
\end{tabular}
\end{center}
\end{table}

% latex table generated in R 2.13.1 by xtable 1.6-0 package
% Thu Jan 26 11:22:25 2012
\begin{table}[htb]
\begin{center}
\caption{Secretary of State 2010 LD 14 (Benchmark)}
\label{sos10_cvap_ld_14_benchmark}
\begin{tabular}{lccccc}
  \hline
Racial Group & Turnout & D Vote & R Vote & S Vote & Total CVAP \\ 
  \hline
White & 0.23 & 0.52 & 0.48 & 0.00 & 0.39 \\ 
  Hispanic & 0.19 & 0.87 & 0.12 & 0.00 & 0.44 \\ 
  Native American & 0.24 & 0.57 & 0.42 & 0.01 & 0.04 \\ 
  Black & 0.27 & 0.59 & 0.40 & 0.01 & 0.06 \\ 
  Other & 0.43 & 0.53 & 0.46 & 0.01 & 0.06 \\ 
  Total Votes & 0.23 & 0.66 & 0.34 & 0.00 &  \\ 
   \hline
\end{tabular}
\end{center}
\end{table}


\clearpage

\subsection{LD 15}

% latex table generated in R 2.13.1 by xtable 1.6-0 package
% Thu Jan 26 11:22:24 2012
\begin{table}[htb]
\begin{center}
\caption{Mine Inspector 2010 LD 15 (Benchmark)}
\label{smine_cvap_ld_15_benchmark}
\begin{tabular}{lccccc}
  \hline
Racial Group & Turnout & D Vote & R Vote & S Vote & Total CVAP \\ 
  \hline
White & 0.30 & 0.65 & 0.35 & 0.00 & 0.59 \\ 
  Hispanic & 0.15 & 0.67 & 0.33 & 0.01 & 0.25 \\ 
  Native American & 0.31 & 0.59 & 0.40 & 0.02 & 0.04 \\ 
  Black & 0.34 & 0.53 & 0.46 & 0.01 & 0.06 \\ 
  Other & 0.56 & 0.50 & 0.49 & 0.01 & 0.06 \\ 
  Total Votes & 0.28 & 0.63 & 0.37 & 0.00 &  \\ 
   \hline
\end{tabular}
\end{center}
\end{table}

\input{figs/pres04_cvap_ld_15_benchmark}
\input{figs/pres08_cvap_ld_15_benchmark}
\input{figs/sos06_cvap_ld_15_benchmark}
\input{figs/sos10_cvap_ld_15_benchmark}

% latex table generated in R 2.13.1 by xtable 1.6-0 package
% Thu Jan 26 11:22:20 2012
\begin{table}[htb]
\begin{center}
\caption{State Senator 2008 LD 15 (Benchmark)}
\label{stsen08_cvap_ld_15_benchmark}
\begin{tabular}{lccccc}
  \hline
Racial Group & Turnout & D Vote & R Vote & S Vote & Total CVAP \\ 
  \hline
    White & 0.31  & 0.67  & 0.33  & 0.00  & 0.59 \\
    Hispanic & 0.16  & 0.60  & 0.40  & 0.01  & 0.25 \\
    Native American & 0.31  & 0.59  & 0.40  & 0.01  & 0.04 \\
    Black & 0.45  & 0.51  & 0.49  & 0.01  & 0.06 \\
    Other & 0.46  & 0.50  & 0.49  & 0.01  & 0.06 \\
    Total Votes & 0.29  & 0.63  & 0.37  & 0.00  &  NA \\
   \hline
\end{tabular}
\end{center}
\end{table}
\input{figs/stsen10_ld_15_benchmark}

\clearpage

\subsection{LD 16}

\input{figs/smine_cvap_ld_16_benchmark}
% latex table generated in R 2.13.1 by xtable 1.6-0 package
% Thu Jan 26 11:22:24 2012
\begin{table}[htb]
\begin{center}
\caption{US President 2004 LD 16 (Benchmark)}
\label{pres04_cvap_ld_16_benchmark}
\begin{tabular}{lccccc}
  \hline
Racial Group & Turnout & D Vote & R Vote & S Vote & Total CVAP \\ 
  \hline
White & 0.16 & 0.50 & 0.49 & 0.01 & 0.30 \\ 
  Hispanic & 0.34 & 0.78 & 0.22 & 0.00 & 0.45 \\ 
  Native American & 0.31 & 0.50 & 0.46 & 0.03 & 0.03 \\ 
  Black & 0.33 & 0.56 & 0.43 & 0.01 & 0.17 \\ 
  Other & 0.21 & 0.46 & 0.51 & 0.03 & 0.06 \\ 
  Total Votes & 0.28 & 0.67 & 0.33 & 0.01 &  \\ 
   \hline
\end{tabular}
\end{center}
\end{table}

% latex table generated in R 2.13.1 by xtable 1.6-0 package
% Thu Jan 26 11:22:24 2012
\begin{table}[htb]
\begin{center}
\caption{US President 2008 LD 16 (Benchmark)}
\label{pres08_cvap_ld_16_benchmark}
\begin{tabular}{lccccc}
  \hline
Racial Group & Turnout & D Vote & R Vote & S Vote & Total CVAP \\ 
  \hline
White & 0.45 & 0.53 & 0.46 & 0.01 & 0.30 \\ 
  Hispanic & 0.28 & 0.82 & 0.17 & 0.01 & 0.45 \\ 
  Native American & 0.42 & 0.60 & 0.33 & 0.07 & 0.03 \\ 
  Black & 0.54 & 0.84 & 0.15 & 0.01 & 0.17 \\ 
  Other & 0.52 & 0.57 & 0.39 & 0.04 & 0.06 \\ 
  Total Votes & 0.39 & 0.70 & 0.29 & 0.01 &  \\ 
   \hline
\end{tabular}
\end{center}
\end{table}

\input{figs/sos06_cvap_ld_16_benchmark}
\input{figs/sos10_cvap_ld_16_benchmark}

% latex table generated in R 2.13.1 by xtable 1.6-0 package
% Thu Jan 26 11:22:20 2012
\begin{table}[htb]
\begin{center}
\caption{State Senator 2008 LD 16 (Benchmark)}
\label{stsen08_cvap_ld_16_benchmark}
\begin{tabular}{lccccc}
  \hline
Racial Group & Turnout & D Vote & R Vote & S Vote & Total CVAP \\ 
  \hline
	    White & 0.40  & 0.59  & 0.41  & 0.00  & 0.30 \\
    Hispanic & 0.27  & 0.83  & 0.17  & 0.00  & 0.45 \\
    Native American & 0.40  & 0.73  & 0.26  & 0.01  & 0.03 \\
    Black & 0.50  & 0.85  & 0.15  & 0.00  & 0.17 \\
    Other & 0.46  & 0.55  & 0.44  & 0.01  & 0.06 \\
    Total Votes & 0.36  & 0.73  & 0.26  & 0.00  &  NA \\
   \hline
\end{tabular}
\end{center}
\end{table}

\clearpage

\subsection{LD 17}

% latex table generated in R 2.13.1 by xtable 1.6-0 package
% Thu Jan 26 11:22:24 2012
\begin{table}[htb]
\begin{center}
\caption{Mine Inspector 2010 LD 17 (Benchmark)}
\label{smine_cvap_ld_17_benchmark}
\begin{tabular}{lccccc}
  \hline
Racial Group & Turnout & D Vote & R Vote & S Vote & Total CVAP \\ 
  \hline
    White & 0.31  & 0.51  & 0.49  & 0.00  & 0.76 \\
    Hispanic & 0.22  & 0.65  & 0.34  & 0.01  & 0.13 \\
    Native American & 0.41  & 0.62  & 0.36  & 0.02  & 0.03 \\
    Black & 0.36  & 0.58  & 0.40  & 0.02  & 0.03 \\
    Other & 0.33  & 0.58  & 0.40  & 0.02  & 0.06 \\
    Total Votes & 0.30  & 0.54  & 0.46  & 0.00  &  NA \\
   \hline
\end{tabular}
\end{center}
\end{table}

% latex table generated in R 2.13.1 by xtable 1.6-0 package
% Thu Jan 26 11:22:24 2012
\begin{table}[htb]
\begin{center}
\caption{US President 2004 LD 17 (Benchmark)}
\label{pres04_cvap_ld_17_benchmark}
\begin{tabular}{lccccc}
  \hline
Racial Group & Turnout & D Vote & R Vote & S Vote & Total CVAP \\ 
  \hline
White & 0.52  & 0.56  & 0.43  & 0.00  & 0.76 \\
    Hispanic & 0.33  & 0.46  & 0.52  & 0.02  & 0.13 \\
    Native American & 0.63  & 0.51  & 0.45  & 0.04  & 0.03 \\
    Black & 0.52  & 0.47  & 0.49  & 0.04  & 0.03 \\
    Other & 0.39  & 0.51  & 0.45  & 0.04  & 0.06 \\
    Total Votes & 0.49  & 0.55  & 0.44  & 0.01  &  NA \\
   \hline
\end{tabular}
\end{center}
\end{table}

% latex table generated in R 2.13.1 by xtable 1.6-0 package
% Thu Jan 26 11:22:24 2012
\begin{table}[htb]
\begin{center}
\caption{US President 2008 LD 17 (Benchmark)}
\label{pres08_cvap_ld_17_benchmark}
\begin{tabular}{lccccc}
  \hline
Racial Group & Turnout & D Vote & R Vote & S Vote & Total CVAP \\ 
  \hline
    White & 0.51  & 0.56  & 0.43  & 0.01  & 0.76 \\
    Hispanic & 0.31  & 0.57  & 0.37  & 0.06  & 0.13 \\
    Native American & 0.62  & 0.60  & 0.34  & 0.06  & 0.03 \\
    Black & 0.59  & 0.61  & 0.31  & 0.08  & 0.03 \\
    Other & 0.56  & 0.53  & 0.37  & 0.10  & 0.06 \\
    Total Votes & 0.49  & 0.56  & 0.41  & 0.02  &  NA \\
   \hline
\end{tabular}
\end{center}
\end{table}

% latex table generated in R 2.13.1 by xtable 1.6-0 package
% Thu Jan 26 11:22:24 2012
\begin{table}[htb]
\begin{center}
\caption{Secretary of State 2006 LD 17 (Benchmark)}
\label{sos06_cvap_ld_17_benchmark}
\begin{tabular}{lccccc}
  \hline
Racial Group & Turnout & D Vote & R Vote & S Vote & Total CVAP \\ 
  \hline
    White & 0.33  & 0.48  & 0.51  & 0.02  & 0.76 \\
    Hispanic & 0.25  & 0.41  & 0.46  & 0.13  & 0.13 \\
    Native American & 0.40  & 0.36  & 0.56  & 0.09  & 0.03 \\
    Black & 0.38  & 0.39  & 0.44  & 0.17  & 0.03 \\
    Other & 0.29  & 0.41  & 0.41  & 0.17  & 0.06 \\
    Total Votes & 0.32  & 0.46  & 0.50  & 0.04  &  NA \\
   \hline
\end{tabular}
\end{center}
\end{table}

% latex table generated in R 2.13.1 by xtable 1.6-0 package
% Thu Jan 26 11:22:24 2012
\begin{table}[htb]
\begin{center}
\caption{Secretary of State 2010 LD 17 (Benchmark)}
\label{sos10_cvap_ld_17_benchmark}
\begin{tabular}{lccccc}
  \hline
Racial Group & Turnout & D Vote & R Vote & S Vote & Total CVAP \\ 
  \hline
    White & 0.33  & 0.50  & 0.50  & 0.00  & 0.76 \\
    Hispanic & 0.20  & 0.64  & 0.35  & 0.01  & 0.13 \\
    Native American & 0.40  & 0.63  & 0.36  & 0.02  & 0.03 \\
    Black & 0.40  & 0.60  & 0.39  & 0.01  & 0.03 \\
    Other & 0.35  & 0.60  & 0.39  & 0.01  & 0.06 \\
    Total Votes & 0.32  & 0.52  & 0.48  & 0.00  &  NA \\
   \hline
\end{tabular}
\end{center}
\end{table}


% latex table generated in R 2.13.1 by xtable 1.6-0 package
% Thu Jan 26 11:22:20 2012
\begin{table}[htb]
\begin{center}
\caption{State Senator 2004 LD 17 (Benchmark)}
\label{stsen04_cvap_ld_17}
\begin{tabular}{lccccc}
  \hline
Racial Group & Turnout & D Vote & R Vote (H) & S Vote & Total CVAP \\ 
  \hline
    White & 0.46  & 0.64  & 0.36  & 0.00  & 0.76 \\
    Hispanic & 0.37  & 0.52  & 0.48  & 0.00  & 0.13 \\
    Native American & 0.57  & 0.48  & 0.52  & 0.00  & 0.03 \\
    Black & 0.47  & 0.50  & 0.50  & 0.00  & 0.03 \\
    Other & 0.33  & 0.48  & 0.52  & 0.00  & 0.06 \\
    Total Pop & 0.44  & 0.61  & 0.39  & 0.00  & NA \\
   \hline
\end{tabular}
\end{center}
\end{table}
% latex table generated in R 2.13.1 by xtable 1.6-0 package
% Thu Jan 26 11:22:20 2012
\begin{table}[htb]
\begin{center}
\caption{State Senator 2008 LD 17 (Benchmark)}
\label{stsen08_cvap_ld_17_benchmark}
\begin{tabular}{lccccc}
  \hline
Racial Group & Turnout & D Vote & R Vote & S Vote & Total CVAP \\ 
  \hline
   White & 0.47  & 0.62  & 0.38  & 0.00  & 0.76 \\
    Hispanic & 0.26  & 0.61  & 0.39  & 0.01  & 0.13 \\
    Native American & 0.56  & 0.61  & 0.37  & 0.01  & 0.03 \\
    Black & 0.54  & 0.57  & 0.42  & 0.01  & 0.03 \\
    Other & 0.44  & 0.60  & 0.39  & 0.01  & 0.06 \\
    Total Votes & 0.44  & 0.61  & 0.38  & 0.00  &  NA \\

   \hline
\end{tabular}
\end{center}
\end{table}
% latex table generated in R 2.13.1 by xtable 1.6-0 package
% Thu Jan 26 11:22:20 2012
\begin{table}[htb]
\begin{center}
\caption{State Senator 2010 LD 17 (Benchmark)}
\label{stsen10_cvap_ld_17_benchmark}
\begin{tabular}{lccccc}
  \hline
Racial Group & Turnout & D Vote & R Vote & S Vote & Total CVAP \\ 
  \hline
    White & 0.33  & 0.50  & 0.48  & 0.02  & 0.76 \\
    Hispanic & 0.23  & 0.53  & 0.32  & 0.15  & 0.13 \\
    Native American & 0.48  & 0.59  & 0.34  & 0.07  & 0.03 \\
    Black & 0.44  & 0.51  & 0.34  & 0.15  & 0.03 \\
    Other & 0.39  & 0.49  & 0.32  & 0.19  & 0.06 \\
    Total Pop & 0.33  & 0.51  & 0.44  & 0.05  & NA \\   
   \hline
\end{tabular}
\end{center}
\end{table}

%% latex table generated in R 2.13.1 by xtable 1.6-0 package
% Thu Jan 26 11:22:20 2012
\begin{table}[htb]
\begin{center}
\caption{State Senator 2008 LD 17 (Benchmark)}
\label{stsen08_cvap_ld_17}
\begin{tabular}{lccccc}
  \hline
Racial Group & Turnout & D Vote & R Vote (H) & S Vote & Total CVAP \\ 
  \hline
	White & 0.46  & 0.62  & 0.38  & 0.00  & 0.76 \\
    Hispanic & 0.27  & 0.62  & 0.37  & 0.01  & 0.13 \\
    Native American & 0.58  & 0.62  & 0.37  & 0.01  & 0.03 \\
    Black & 0.49  & 0.61  & 0.37  & 0.01  & 0.03 \\
    Other & 0.45  & 0.53  & 0.46  & 0.01  & 0.06 \\
    Total Pop & 0.44  & 0.61  & 0.38  & 0.00  & NA \\
   \hline
\end{tabular}
\end{center}
\end{table}

\clearpage

\subsection{LD 23}

% latex table generated in R 2.13.1 by xtable 1.6-0 package
% Thu Jan 26 11:22:24 2012
\begin{table}[htb]
\begin{center}
\caption{Mine Inspector 2010 LD 23 (Benchmark)}
\label{smine_cvap_ld_23_benchmark}
\begin{tabular}{lccccc}
  \hline
Racial Group & Turnout & D Vote & R Vote & S Vote & Total CVAP \\ 
  \hline
White & 0.32 & 0.30 & 0.70 & 0.00 & 0.59 \\ 
  Hispanic & 0.22 & 0.80 & 0.20 & 0.00 & 0.25 \\ 
  Native American & 0.12 & 0.81 & 0.18 & 0.01 & 0.07 \\ 
  Black & 0.31 & 0.52 & 0.48 & 0.00 & 0.04 \\ 
  Other & 0.44 & 0.63 & 0.36 & 0.00 & 0.05 \\ 
  Total Votes & 0.29 & 0.45 & 0.55 & 0.00 &  \\ 
   \hline
\end{tabular}
\end{center}
\end{table}

% latex table generated in R 2.13.1 by xtable 1.6-0 package
% Thu Jan 26 11:22:24 2012
\begin{table}[htb]
\begin{center}
\caption{US President 2004 LD 23 (Benchmark)}
\label{pres04_cvap_ld_23_benchmark}
\begin{tabular}{lccccc}
  \hline
Racial Group & Turnout & D Vote & R Vote & S Vote & Total CVAP \\ 
  \hline
White & 0.25 & 0.35 & 0.64 & 0.00 & 0.59 \\ 
  Hispanic & 0.26 & 0.63 & 0.36 & 0.01 & 0.25 \\ 
  Native American & 0.16 & 0.62 & 0.36 & 0.02 & 0.07 \\ 
  Black & 0.46 & 0.41 & 0.58 & 0.01 & 0.04 \\ 
  Other & 0.39 & 0.51 & 0.48 & 0.02 & 0.05 \\ 
  Total Votes & 0.26 & 0.45 & 0.55 & 0.01 &  \\ 
   \hline
\end{tabular}
\end{center}
\end{table}

% latex table generated in R 2.13.1 by xtable 1.6-0 package
% Thu Jan 26 11:22:24 2012
\begin{table}[htb]
\begin{center}
\caption{US President 2008 LD 23 (Benchmark)}
\label{pres08_cvap_ld_23_benchmark}
\begin{tabular}{lccccc}
  \hline
Racial Group & Turnout & D Vote & R Vote & S Vote & Total CVAP \\ 
  \hline
White & 0.43 & 0.36 & 0.64 & 0.01 & 0.59 \\ 
  Hispanic & 0.26 & 0.79 & 0.20 & 0.02 & 0.25 \\ 
  Native American & 0.25 & 0.68 & 0.30 & 0.02 & 0.07 \\ 
  Black & 0.38 & 0.45 & 0.49 & 0.05 & 0.04 \\ 
  Other & 0.56 & 0.39 & 0.57 & 0.04 & 0.05 \\ 
  Total Votes & 0.38 & 0.45 & 0.53 & 0.01 &  \\ 
   \hline
\end{tabular}
\end{center}
\end{table}

\input{figs/sos06_cvap_ld_23_benchmark}
\input{figs/sos10_cvap_ld_23_benchmark}

% latex table generated in R 2.13.1 by xtable 1.6-0 package
% Thu Jan 26 11:22:20 2012
\begin{table}[htb]
\begin{center}
\caption{State Senator 2008 LD 23 (Benchmark)}
\label{stsen08_cvap_ld_23_benchmark}
\begin{tabular}{lccccc}
  \hline
Racial Group & Turnout & D Vote & R Vote & S Vote & Total CVAP \\ 
  \hline
      White & 0.38  & 0.49  & 0.51  & 0.00  & 0.59 \\
    Hispanic & 0.27  & 0.82  & 0.18  & 0.00  & 0.25 \\
    Native American & 0.22  & 0.88  & 0.12  & 0.00  & 0.07 \\
    Black & 0.52  & 0.55  & 0.44  & 0.00  & 0.04 \\
    Other & 0.58  & 0.64  & 0.36  & 0.00  & 0.05 \\
    Total Votes & 0.36  & 0.59  & 0.41  & 0.00  &  NA \\
   \hline
\end{tabular}
\end{center}
\end{table}
% latex table generated in R 2.13.1 by xtable 1.6-0 package
% Thu Jan 26 11:22:20 2012
\begin{table}[htb]
\begin{center}
\caption{State Senator 2010 LD 23 (Benchmark)}
\label{stsen10_cvap_ld_23_benchmark}
\begin{tabular}{lccccc}
  \hline
Racial Group & Turnout & D Vote & R Vote & S Vote & Total CVAP \\ 
  \hline
    White & 0.32  & 0.34  & 0.66  & 0.00  & 0.59 \\
    Hispanic & 0.23  & 0.79  & 0.20  & 0.00  & 0.25 \\
    Native American & 0.12  & 0.83  & 0.16  & 0.01  & 0.07 \\
    Black & 0.32  & 0.52  & 0.47  & 0.00  & 0.04 \\
    Other & 0.55  & 0.68  & 0.32  & 0.00  & 0.05 \\
    Total Votes & 0.29  & 0.48  & 0.52  & 0.00  &  NA \\
   \hline
\end{tabular}
\end{center}
\end{table}

\clearpage

\subsection{LD 24}

% latex table generated in R 2.13.1 by xtable 1.6-0 package
% Thu Jan 26 11:22:24 2012
\begin{table}[htb]
\begin{center}
\caption{Mine Inspector 2010 LD 24 (Benchmark)}
\label{smine_cvap_ld_24_benchmark}
\begin{tabular}{lccccc}
  \hline
Racial Group & Turnout & D Vote & R Vote & S Vote & Total CVAP \\ 
  \hline
White & 0.29 & 0.14 & 0.86 & 0.00 & 0.56 \\ 
  Hispanic & 0.27 & 0.83 & 0.17 & 0.00 & 0.35 \\ 
  Native American & 0.32 & 0.63 & 0.36 & 0.01 & 0.02 \\ 
  Black & 0.53 & 0.57 & 0.42 & 0.00 & 0.02 \\ 
  Other & 0.54 & 0.67 & 0.32 & 0.00 & 0.04 \\ 
  Total Votes & 0.30 & 0.43 & 0.57 & 0.00 &  \\ 
   \hline
\end{tabular}
\end{center}
\end{table}

\input{figs/pres04_cvap_ld_24_benchmark}
\input{figs/pres08_cvap_ld_24_benchmark}
\input{figs/sos06_cvap_ld_24_benchmark}
% latex table generated in R 2.13.1 by xtable 1.6-0 package
% Thu Jan 26 11:22:25 2012
\begin{table}[htb]
\begin{center}
\caption{Secretary of State 2010 LD 24 (Benchmark)}
\label{sos10_cvap_ld_24_benchmark}
\begin{tabular}{lccccc}
  \hline
Racial Group & Turnout & D Vote & R Vote & S Vote & Total CVAP \\ 
  \hline
White & 0.30 & 0.15 & 0.85 & 0.00 & 0.56 \\ 
  Hispanic & 0.28 & 0.78 & 0.22 & 0.00 & 0.35 \\ 
  Native American & 0.32 & 0.68 & 0.32 & 0.01 & 0.02 \\ 
  Black & 0.59 & 0.61 & 0.38 & 0.00 & 0.02 \\ 
  Other & 0.51 & 0.62 & 0.37 & 0.00 & 0.04 \\ 
  Total Votes & 0.31 & 0.42 & 0.58 & 0.00 &  \\ 
   \hline
\end{tabular}
\end{center}
\end{table}


% latex table generated in R 2.13.1 by xtable 1.6-0 package
% Thu Jan 26 11:22:20 2012
\begin{table}[htb]
\begin{center}
\caption{State Senator 2010 LD 24 (Benchmark)}
\label{stsen10_cvap_ld_24_benchmark}
\begin{tabular}{lccccc}
  \hline
Racial Group & Turnout & D Vote & R Vote & S Vote & Total CVAP \\ 
  \hline
   White & 0.31  & 0.17  & 0.83  & 0.00  & 0.56 \\
    Hispanic & 0.29  & 0.82  & 0.18  & 0.00  & 0.35 \\
    Native American & 0.31  & 0.72  & 0.27  & 0.01  & 0.02 \\
    Black & 0.62  & 0.59  & 0.40  & 0.00  & 0.02 \\
    Other & 0.47  & 0.72  & 0.27  & 0.00  & 0.04 \\
    Total Votes & 0.32  & 0.45  & 0.55  & 0.00  &  NA \\
   \hline
\end{tabular}
\end{center}
\end{table}

\clearpage

\subsection{LD 25}

% latex table generated in R 2.13.1 by xtable 1.6-0 package
% Thu Jan 26 11:22:24 2012
\begin{table}[htb]
\begin{center}
\caption{Mine Inspector 2010 LD 25 (Benchmark)}
\label{smine_cvap_ld_25_benchmark}
\begin{tabular}{lccccc}
  \hline
Racial Group & Turnout & D Vote & R Vote & S Vote & Total CVAP \\ 
  \hline
White & 0.41 & 0.26 & 0.74 & 0.00 & 0.57 \\ 
  Hispanic & 0.32 & 0.89 & 0.11 & 0.00 & 0.31 \\ 
  Native American & 0.21 & 0.92 & 0.07 & 0.01 & 0.05 \\ 
  Black & 0.29 & 0.42 & 0.57 & 0.01 & 0.02 \\ 
  Other & 0.39 & 0.61 & 0.38 & 0.01 & 0.04 \\ 
  Total Votes & 0.37 & 0.47 & 0.53 & 0.00 &  \\ 
   \hline
\end{tabular}
\end{center}
\end{table}

% latex table generated in R 2.13.1 by xtable 1.6-0 package
% Thu Jan 26 11:22:24 2012
\begin{table}[htb]
\begin{center}
\caption{US President 2004 LD 25 (Benchmark)}
\label{pres04_cvap_ld_25_benchmark}
\begin{tabular}{lccccc}
  \hline
Racial Group & Turnout & D Vote & R Vote & S Vote & Total CVAP \\ 
  \hline
White & 0.36 & 0.34 & 0.66 & 0.00 & 0.57 \\ 
  Hispanic & 0.50 & 0.65 & 0.34 & 0.01 & 0.31 \\ 
  Native American & 0.50 & 0.70 & 0.29 & 0.01 & 0.05 \\ 
  Black & 0.38 & 0.32 & 0.64 & 0.04 & 0.02 \\ 
  Other & 0.63 & 0.24 & 0.73 & 0.03 & 0.04 \\ 
  Total Votes & 0.42 & 0.47 & 0.52 & 0.01 &  \\ 
   \hline
\end{tabular}
\end{center}
\end{table}

% latex table generated in R 2.13.1 by xtable 1.6-0 package
% Thu Jan 26 11:22:24 2012
\begin{table}[htb]
\begin{center}
\caption{US President 2008 LD 25 (Benchmark)}
\label{pres08_cvap_ld_25_benchmark}
\begin{tabular}{lccccc}
  \hline
Racial Group & Turnout & D Vote & R Vote & S Vote & Total CVAP \\ 
  \hline
White & 0.49 & 0.28 & 0.71 & 0.01 & 0.57 \\ 
  Hispanic & 0.48 & 0.76 & 0.23 & 0.01 & 0.32 \\ 
  Native American & 0.35 & 0.78 & 0.20 & 0.02 & 0.05 \\ 
  Black & 0.45 & 0.34 & 0.58 & 0.09 & 0.02 \\ 
  Other & 0.53 & 0.43 & 0.48 & 0.09 & 0.04 \\ 
  Total Votes & 0.48 & 0.46 & 0.52 & 0.01 &  \\ 
   \hline
\end{tabular}
\end{center}
\end{table}

\input{figs/sos06_cvap_ld_25_benchmark}
\input{figs/sos10_cvap_ld_25_benchmark}

% latex table generated in R 2.13.1 by xtable 1.6-0 package
% Thu Jan 26 11:22:20 2012
\begin{table}[htb]
\begin{center}
\caption{State Senator 2008 LD 25 (Benchmark)}
\label{stsen08_cvap_ld_25_benchmark}
\begin{tabular}{lccccc}
  \hline
Racial Group & Turnout & D Vote & R Vote & S Vote & Total CVAP \\ 
  \hline
    White & 0.49  & 0.33  & 0.67  & 0.00  & 0.57 \\
    Hispanic & 0.44  & 0.89  & 0.11  & 0.00  & 0.32 \\
    Native American & 0.35  & 0.92  & 0.08  & 0.00  & 0.05 \\
    Black & 0.37  & 0.37  & 0.62  & 0.01  & 0.02 \\
    Other & 0.41  & 0.59  & 0.41  & 0.01  & 0.04 \\
    Total Votes & 0.46  & 0.54  & 0.46  & 0.00  &  NA \\
   \hline
\end{tabular}
\end{center}
\end{table}
% latex table generated in R 2.13.1 by xtable 1.6-0 package
% Thu Jan 26 11:22:20 2012
\begin{table}[htb]
\begin{center}
\caption{State Senator 2010 LD 25 (Benchmark)}
\label{stsen10_cvap_ld_25_benchmark}
\begin{tabular}{lccccc}
  \hline
Racial Group & Turnout & D Vote & R Vote & S Vote & Total CVAP \\ 
  \hline
 White & 0.42  & 0.23  & 0.77  & 0.00  & 0.57 \\
    Hispanic & 0.33  & 0.89  & 0.11  & 0.00  & 0.31 \\
    Native American & 0.21  & 0.90  & 0.09  & 0.01  & 0.05 \\
    Black & 0.29  & 0.35  & 0.64  & 0.01  & 0.02 \\
    Other & 0.48  & 0.63  & 0.36  & 0.01  & 0.04 \\
    Total Votes & 0.38  & 0.45  & 0.55  & 0.00  &  NA \\
   \hline
\end{tabular}
\end{center}
\end{table}

\clearpage

\subsection{LD 27}

\input{figs/smine_cvap_ld_27_benchmark}
\input{figs/pres04_cvap_ld_27_benchmark}
\input{figs/pres08_cvap_ld_27_benchmark}
% latex table generated in R 2.13.1 by xtable 1.6-0 package
% Thu Jan 26 11:22:24 2012
\begin{table}[htb]
\begin{center}
\caption{Secretary of State 2006 LD 27 (Benchmark)}
\label{sos06_cvap_ld_27_benchmark}
\begin{tabular}{lccccc}
  \hline
Racial Group & Turnout & D Vote & R Vote & S Vote & Total CVAP \\ 
  \hline
White & 0.34 & 0.57 & 0.42 & 0.01 & 0.46 \\ 
  Hispanic & 0.26 & 0.77 & 0.20 & 0.02 & 0.42 \\ 
  Native American & 0.22 & 0.56 & 0.33 & 0.11 & 0.04 \\ 
  Black & 0.41 & 0.48 & 0.40 & 0.12 & 0.03 \\ 
  Other & 0.39 & 0.44 & 0.43 & 0.13 & 0.05 \\ 
  Total Votes & 0.31 & 0.63 & 0.34 & 0.03 &  \\ 
   \hline
\end{tabular}
\end{center}
\end{table}

% latex table generated in R 2.13.1 by xtable 1.6-0 package
% Thu Jan 26 11:22:25 2012
\begin{table}[htb]
\begin{center}
\caption{Secretary of State 2010 LD 27 (Benchmark)}
\label{sos10_cvap_ld_27_benchmark}
\begin{tabular}{lccccc}
  \hline
Racial Group & Turnout & D Vote & R Vote & S Vote & Total CVAP \\ 
  \hline
White & 0.36 & 0.50 & 0.50 & 0.00 & 0.46 \\ 
  Hispanic & 0.32 & 0.84 & 0.16 & 0.00 & 0.42 \\ 
  Native American & 0.19 & 0.72 & 0.27 & 0.01 & 0.04 \\ 
  Black & 0.36 & 0.62 & 0.37 & 0.01 & 0.03 \\ 
  Other & 0.51 & 0.59 & 0.41 & 0.00 & 0.05 \\ 
  Total Votes & 0.34 & 0.65 & 0.35 & 0.00 &  \\ 
   \hline
\end{tabular}
\end{center}
\end{table}


% latex table generated in R 2.13.1 by xtable 1.6-0 package
% Thu Jan 26 11:22:20 2012
\begin{table}[htb]
\begin{center}
\caption{State Senator 2008 LD 27 (Benchmark)}
\label{stsen08_cvap_ld_27_benchmark}
\begin{tabular}{lccccc}
  \hline
Racial Group & Turnout & D Vote & R Vote & S Vote & Total CVAP \\ 
  \hline
   White & 0.44  & 0.51  & 0.49  & 0.00  & 0.46 \\
    Hispanic & 0.40  & 0.89  & 0.11  & 0.00  & 0.42 \\
    Native American & 0.27  & 0.75  & 0.24  & 0.01  & 0.04 \\
    Black & 0.55  & 0.61  & 0.38  & 0.00  & 0.03 \\
    Other & 0.73  & 0.55  & 0.44  & 0.00  & 0.05 \\
    Total Votes & 0.43  & 0.68  & 0.32  & 0.00  &  NA \\
   \hline
\end{tabular}
\end{center}
\end{table}

\clearpage

\subsection{LD 29}

\input{figs/smine_cvap_ld_29_benchmark}
\input{figs/pres04_cvap_ld_29_benchmark}
% latex table generated in R 2.13.1 by xtable 1.6-0 package
% Thu Jan 26 11:22:24 2012
\begin{table}[htb]
\begin{center}
\caption{US President 2008 LD 29 (Benchmark)}
\label{pres08_cvap_ld_29_benchmark}
\begin{tabular}{lccccc}
  \hline
Racial Group & Turnout & D Vote & R Vote & S Vote & Total CVAP \\ 
  \hline
White & 0.52 & 0.48 & 0.52 & 0.00 & 0.49 \\ 
  Hispanic & 0.33 & 0.85 & 0.14 & 0.01 & 0.37 \\ 
  Native American & 0.39 & 0.72 & 0.23 & 0.05 & 0.03 \\ 
  Black & 0.32 & 0.47 & 0.47 & 0.05 & 0.06 \\ 
  Other & 0.51 & 0.53 & 0.43 & 0.04 & 0.06 \\ 
  Total Votes & 0.43 & 0.59 & 0.40 & 0.01 &  \\ 
   \hline
\end{tabular}
\end{center}
\end{table}

% latex table generated in R 2.13.1 by xtable 1.6-0 package
% Thu Jan 26 11:22:24 2012
\begin{table}[htb]
\begin{center}
\caption{Secretary of State 2006 LD 29 (Benchmark)}
\label{sos06_cvap_ld_29_benchmark}
\begin{tabular}{lccccc}
  \hline
Racial Group & Turnout & D Vote & R Vote & S Vote & Total CVAP \\ 
  \hline
White & 0.32 & 0.45 & 0.53 & 0.02 & 0.49 \\ 
  Hispanic & 0.25 & 0.82 & 0.15 & 0.03 & 0.37 \\ 
  Native American & 0.24 & 0.61 & 0.27 & 0.12 & 0.03 \\ 
  Black & 0.21 & 0.42 & 0.44 & 0.13 & 0.06 \\ 
  Other & 0.24 & 0.54 & 0.32 & 0.14 & 0.06 \\ 
  Total Votes & 0.28 & 0.58 & 0.38 & 0.03 &  \\ 
   \hline
\end{tabular}
\end{center}
\end{table}

% latex table generated in R 2.13.1 by xtable 1.6-0 package
% Thu Jan 26 11:22:25 2012
\begin{table}[htb]
\begin{center}
\caption{Secretary of State 2010 LD 29 (Benchmark)}
\label{sos10_cvap_ld_29_benchmark}
\begin{tabular}{lccccc}
  \hline
Racial Group & Turnout & D Vote & R Vote & S Vote & Total CVAP \\ 
  \hline
White & 0.38 & 0.47 & 0.53 & 0.00 & 0.49 \\ 
  Hispanic & 0.26 & 0.90 & 0.10 & 0.00 & 0.37 \\ 
  Native American & 0.32 & 0.79 & 0.20 & 0.01 & 0.03 \\ 
  Black & 0.25 & 0.44 & 0.55 & 0.01 & 0.06 \\ 
  Other & 0.33 & 0.56 & 0.44 & 0.00 & 0.06 \\ 
  Total Votes & 0.32 & 0.61 & 0.39 & 0.00 &  \\ 
   \hline
\end{tabular}
\end{center}
\end{table}


\clearpage
\singlespace
\bibliographystyle{apsr} 
\bibsep=0in 
%\phantomsection
%\addcontentsline{toc}{section}{References}
\bibliography{gk,gkpubs}
\end{document}
