\documentclass[12pt]{article}
%\usepackage{Rd,Sweave,upquote}
\usepackage[reqno]{amsmath}
\usepackage{natbib,amssymb,amsthm,graphicx,verbatim,url,verbatim}
\usepackage[all]{xy}
\usepackage{vmargin}
\setpapersize{USletter}
\topmargin=0in
\usepackage{times}
\usepackage{dcolumn, booktabs}
\usepackage[dvipsnames,usenames]{color}
\usepackage{dcolumn}
\newcolumntype{.}{D{.}{.}{-1}}\newcolumntype{d}[1]{D{.}{.}{#1}}
%\usepackage[notref]{showkeys}
%\usepackage{endfloat}

\usepackage{color,setspace}
\definecolor{spot}{rgb}{0.6,0,0}
\usepackage[pdftex, bookmarksopen=true, bookmarksnumbered=true,
  pdfstartview=FitH, breaklinks=true, urlbordercolor={0 1 0},
  citebordercolor={0 0 1}, colorlinks=true, citecolor=spot, 
  linkcolor=spot, urlcolor=spot,
  pdfauthor={Gary King},
  pdftitle={Differential Racial Disenfranchisement induced by the Texas Voter ID Law}]{hyperref}

%squeeze
% == spacing between sections and subsections
%\usepackage[compact]{titlesec} 
% keep floats on same page
\renewcommand{\topfraction}{0.85}
\renewcommand{\textfraction}{0.1}
\renewcommand{\floatpagefraction}{0.75} % keep < \topfraction

\usepackage{soul}

\usepackage{appendix}
\usepackage{latexsym}
\newtheorem{claim}{Claim}
\newtheorem{proposition}{Proposition}
\newtheorem{theorem}{Theorem}
\newtheorem{lemma}{Lemma}
\newtheorem*{ass}{Assumption}
\newtheorem{question}{Question}
\newtheorem{remark}{Remark}
\newtheorem{definition}{Definition}
\newcommand{\bX}{\boldsymbol{X}}


\title{Differential Racial Disenfranchisement induced by the Texas
  Voter ID Law}\author{Gary King\thanks{Alfred J.  Weatherhead III
      University Professor, Harvard University, Institute for
      Quantitative Social Science, 1737 Cambridge Street, Cambridge MA
      02138; http://GKing.harvard.edu, king@harvard.edu, (617)
      500-7570.}}

%\date{June 8, 2012}

\begin{document}
\maketitle
\tableofcontents
\clearpage
\doublespacing

\section{Introduction}

The purpose of this report is to estimate the differential percentages
of black, white, and Hispanic citizens of voting age who would be
disenfranchised (not allowed to vote) solely because they do not meet
the additional requirements of the new Texas Voter ID law.

The report from the Department of Justice's expert Stephen
Ansolabehere examines the differential percentages of black, white,
and Hispanic \emph{registered voters} who would be disenfranchised by
the new law.  We supplement this information by attempting to estimate
the broader percentages disenfranchised of \emph{all} those in each
racial group who would have been eligible to vote without the new law.
We also use different assumptions and methods than Ansolabehere, the
uncertainties of which are likely uncorrelated with those from his
report.  The result is that the combination of the two reports should
provide stronger inferences than either alone.

\section{Background and Qualifications}

I am the Albert J Weatherhead III University Professor at Harvard
University, one of 24 with the designation of University Professor.  I
have taught at Harvard since 1987 and before that at New York
University.  I hold a Ph.D. from the University of Wisconsin.  I have
been elected a member of six honorary societies (National Academy of
Sciences 2010, American Statistical Association 2009, American
Association for the Advancement of Science 2004, American Academy of
Arts and Sciences 1998, Society for Political Methodology 2008, and
American Academy of Political and Social Science 2004), and have won
more than 30 ``best of'' awards for my articles, books, conference
papers, and software, among others.  I have published more than 130
journal articles, 15 open source software packages, and 8 books
spanning most aspects of political methodology, several fields of
political science, statistics, and other scholarly disciplines.  I
have served on the editorial boards of more than 30 scholarly
journals.  The statistical methods and software I developed are used
extensively in the litigation, academia, government, consulting, and
private industry.  More information on me and copies of most of my
work is available at \url{http://gking.harvard.edu}.

I have been an expert witness or consultant (for governments,
partisans on both sides, and nonpartisan and bipartisan groups) in
many U.S.\ states for legislative redistricting and other matters.
Recently, I was an expert for the Arizona Independent Redistricting
Commission.  I filed an Amici Curiae brief in \emph{LULAC v. Perry}
with three other scholars, which was discussed in three of the Supreme
Court's opinions, including the plurality decision.

\section{Methodological Approach}\label{s:methods}

In this section, we offer an overview of our methodological approach.
We will attempt to estimate the percentage of black, white, Hispanic,
and other citizens of voting age who have valid voter identification,
according to the new Texas Voter ID law.  Throughout, I was assisted
by two graduate students working under my direction.

From U.S.\ Census Data, we compute the percent of citizens of voting
age who are black, white, Hispanic, or other in each county in
Texas.  Then from the State of Texas' driver's license database, we
compute the number of citizens of voting age (of any race) who have
valid identification, and from the combination of the two databases
the percent of citizens of voting age in each racial group who have a valid ID.  

As there exists no data base to provide a complete cross-tabulation of
these results at the individual level, we use methods of ``ecological
inference'' --- which is the process of inferring from group level
data to information about individuals.  We summarize here the methods
of ecological inference we use.

In 1953, two methods of ecological inference were introduced --- the
method of bounds \citep{DunDav53} and ecological regression
\citep{Goodman53}. A special case of the method of bounds is known as
``homogeneous precinct analysis,'' which had been used in many court
cases: this approach seeks out ethnically homogeneous precincts (100\%
black or 100\% white, or 100\% Hispanic) because for those precincts
we know for certain (i.e., up to the input data) the behavior of
interest (in this case the percent of those with a valid ID) of that
one ethnic group.  The assumption behind this method in inferring to
the population for the entire state is that the percent with an ID
observed in homogeneous precincts is identical to that in other areas.
The advantage of this method is that it yields completely certain
information about some subgroups of citizens in some areas; the
disadvantage is that the relatively few who live in racially
homogeneous precincts may have different proportions of IDs
than the vast majority of the population who live in (at least
partially) heterogeneous areas.  The method of bounds is more general
than homogeneous precinct analysis because the former can also provide
some information about heterogeneous precincts; it does this in the
form of ``bounds'' or ranges into which the fraction of a minority
group having IDs must fall.

The second method of ecological inference --- ecological regression
--- ignores the information revealed by the method of bounds and its
special case of homogeneous precinct analysis. Instead, ecological
regression uses statistical information across all precincts. For
example, if we find in areas with more African Americans that fewer
people have voter IDs, then we may be willing to infer that it is the
African Americans who have fewer IDs. The advantage of this approach
is that it uses some information from all precincts. The disadvantage
is that the information can be misleading in a different way: For
example, also consistent with the same evidence would be that the
(perhaps poor) whites who live in areas with high African Americans
concentrations are the ones with fewer IDs. In fact, as an indication
of the common problems with this method, ecological regression, unlike
the method of bounds, regularly gives impossible answers -- such as
estimating the percent of Hispanics voting for the Democrats as 120\%
or $-54$\%.

The method of bounds (or homogeneous precinct analysis) and ecological
regression dominated the academic literature and courtroom expert
testimony from 1953 until 1997 when the approach known as EI was
introduced \citep{King97}. This approach was the first to combine the
deterministic information from the method of bounds with the
statistical information from ecological regression into a single set
of estimates. Thus, it uses the statistical information from all
precincts, the certain information from homogeneous precincts, and
other deterministic information known for certain from other precincts
(given in the form of ranges of estimates).  Impossible estimates are
never produced by this methodology, and all information from all
precincts are used in the analysis. Like any indirect method of
revealing information about individuals from aggregate data, EI is
also uncertain to a degree, but it uses more information than other
previous approaches.

Since EI was introduced, a variety of other methods have been
developed in the academic literature, almost all of which follow the
same practice of including deterministic and statistical information
in the same model.  For example, \citet{RosJiaKin01} extend King's
method to larger numbers of ethnic groups and candidates; we use this
method in our work as well as King's original method. Some of the
other methods of ecological inference have been collected in the
edited volume by \citet{KinRosTan04b}.

In practice, when we have data we can use to validate the methods,
ecological regression and homogeneous precinct analysis can be
inaccurate.  Studies have shown that uncertainty remains with King's
and other subsequent methods --- which will always be the case because
some information is lost in the process of aggregation --- but the
estimates are usually superior. Differences among the new methods that
include both deterministic and statistical information are, in
comparison, relatively minor.

In the present case, we use these newer methods to give estimates of
the proportion of each ethnic group that has a valid voter ID.  We
also use ``tomography plots'' and other diagnostic methods that help
discern when adjustments in the methods need to be made, and how much
uncertainty remains.

The ecological inference methods we use here have the advantage of
avoiding error that may otherwise be introduced in matching voter
registration files with the driver's license database and using the
Catalyst data to ascertain the race of voters who do not have IDs (and
so are not in the State Driver's license data base).  In addition, we
are able to estimate the percent of all citizens of voting age who
have IDs, rather than only those who are presently registered.
Although one cannot vote without being registered, in future elections
those not registered now of course do have the choice of registering.

\section{Data}\label{s:data}
\subsection{Data Sources}
For our analysis we relied on the Texas Driver's License database (DL),
2010 Census data, 2006-2010 American Community Survey data, and
projected population data from the Texas State Data Center.  The
Driver's License database was provided to us as of April 30, 2012.
From the files provided to us, we used the list of individuals, the
administrative file and the issuances file.\footnote{Individuals:
  DPS.DL.person\_4digit\_social.20120430.txt, Admin:
  DPS.DL.admin\_status.20120430.txt, Issuances:
  DPS.DL.issuance.20120430.txt}

From the 2010 Census, we used "2010 Zip Code to County Relationship File" to match zip codes to counties.\footnote{\url{http://www.census.gov/geo/www/2010census/zcta_rel/zcta_county_rel_10.txt}}

From the 2006-2010 American Community Survey, we used the Citizenship Voting Age Population (CVAP) Special Tabulation to estimate the percentage of individuals who are citizens of voting age within each county by ethnic group.\footnote{\url{http://www.census.gov/geo/www/2010census/zcta_rel/zcta_county_rel_10.txt}}

Due to the rate of population growth in Texas, the 2010 Census data is
insufficient for analyzing population relaitve to the April 30, 2012
Driver's License database.  As a result, we used 2012 Projected
Population estimates from the Texas State Data
Center.\footnote{\url{http://http://txsdc.utsa.edu/}.  We downloaded a
  spreadsheet of this data divided by county and age categories at
  \url{http://www.dshs.state.tx.us/chs/popdat/detailX.shtm}.}

\subsection{Formatting and Cleaning the Driver's License Files}
The Driver's License individuals file contained 25,985,424 records.
Many of these records corresponded to individuals who are deceased,
have expired licenses, or do not reside in Texas.  This database also
included duplicate records that needed to be identified and removed.

We first deleted 287,225 records where the issuance file listed the
individual as not possessing a state driver's license or
identification card.  We then removed 3,144,900 records where the
license or identification card expired was listed in the issuance file
as expired for more than two years.  We then removed 1,535,504 records
where the license or identification was expired for less than two
years but more than sixty days because the voter identification law
allows for a grace period in which identification expired up to sixty
days is accepted.\footnote{We used April 30, 2012, the date of the
  driver's license database we were provided, as the effective election
  date for our analysis.  Therefore, any license expired more than
  sixty days previous to April 30th was removed.}

We then deleted 341,767 records where the individual was reported
deceased in the administrative file.  We also deleted one additional
record with the status code of "DELETED" in the issuance file.

These deletions left us with 20,676,027 records.  We then deleted
102,686 records where the permanent state of the individual was not
Texas, 6 additional records with the permanent country was not the
United States, and 138 records that were missing state, county and
country information.  We deleted 481,524 records where the date of
birth was after April 30, 1994, the cutoff date for turning eighteen
if the election were held on April 30, 2012.

We converted all 9-digit zip codes to 5-digit zip codes.  Without complete 9-digit social security numbers, we were unable to identify and remove duplicates using SSNs.  We removed 496,894 duplicate records that shared first name, last name, date of birth, the last four digits of the social security number, and permanent 5-digit zip code with another record.  We retained the record with the higher record number in the Driver's License database.  The database then contained 19,594,779 records.

[Additional information here on identifying duplicates with other methods, once finalized]

[We then restricted by citizenship.  We dropped XXX records where the individual was flagged as a non-citizen in the Driver's License database, and XXX records where citizenship was missing.] 

\subsection{Aggregating by County and Merging Population Data}
We collapsed the Driver's License database by permanent county and zip
code.  For the set of records [XXX, once finalized] that included zip
code but were missing county, we used the census zip code to county
relationship file to match zip codes to counties.  For zip codes that
are divided between multiple counties, we weighted the records by the
percentage of the population of the zip code within each county, based
on the 2010 census.\footnote{This weighting is based upon the
  assumption that while populations have changed since the 2010
  census, population has changed uniformly within each individual zip
  code that is divided between multiple counties.  This calculation
  only applies to XXX individuals in the driver's license database.}

To estimate current Citizen Voting Age Population (CVAP), we
calculated the percentage of all people who are citizens of voting age
by ethnic group for each county based on the 2006-2010 ACS Special
Tabulation.  We then multiplied these rates by the total number of
people of each ethnic group for each county to estimate CVAP by ethnic
group.\footnote{This assumes that while populations have changed since
  the ACS, the percentage of CVAP for each ethnic group within each
  county has not changed.}  We merged this population data with the
driver's license data to get the number of citizens with IDs and CVAP
totals by ethnic group in each county.  We then calculated the
percentage of CVAP with IDs and CVAP percentage by ethnic group.

\section{Results}\label{s:res}

\subsection{An Explication of Ecological Inference Results}

We used our approach to ecological inference to determine the extent
of the disparity among racial groups in holding a valid form of voter
identification. For each county in Texas, we estimated the proportion
of whites, blacks, Hispanics, and others who held a driver's
license or ID.\footnote{While we could have also used data on concealed hand
  gun permits (as this form of identification would also be considered
  valid at the polls), in practice only 3,739 people possess a permit to
  carry a concealed weapon but not a driver's license or ID.}

Table~\ref{cvap_f3_ex} displays estimates of the proportion of each
racial group that holds a valid driver's license or ID in the state of
Texas. Note that in each table the total population in the last column
and the total license or ID holders in the last row are {\it observed} while
the internal cells are {\it estimated} by our methodology. The bottom
row labeled Total Pop gives the license or ID holder population totals
computed by aggregating over all counties in the state.  Similarly,
the column labeled Total CVAP lists the share of citizen voting age
population comprised by each racial group in the state.  Again, this
is calculated by simply aggregating over all counties in the state.

\begin{table}[ht]
\begin{center}
\caption{\label{cvap_f3_ex}Texas Driver's License or ID Holders by Racial Group}
\begin{tabular}{lccc}
  \hline
Racial Group & License & No License & Total CVAP\\
  \hline
White & 0.855 & 0.145 & 0.551 \\ 
Black & 0.781 & 0.219 & 0.129 \\ 
Hispanic & 0.750 & 0.250 & 0.281 \\ 
Other & 0.857 & 0.143 & 0.039 \\ 
Total Pop & 0.816 & 0.184 &   \\ 
   \hline

\end{tabular}
\end{center}
\end{table}

The internal cells in the table contain estimates from ecological
inference. For example, we estimate that 85.5\% (0.855 in the table)
of whites in Texas held valid driver's licenses or IDs and 14.5\% of
whites in Texas did not.  Furthermore, we know that (according to the
census data in use) whites made up 55.1\% of the state's citizen voting
age population (CVAP).  Looking at the state as a whole, we observe
that 81.6\% of citizens in the voting age population held a driver's
license or ID.

The comparison of interest is between the proportion of white license
or ID holders and license or ID holders of other races. A simple
difference in the estimated percentages suggests that white citizens
were more likely to possess a valid license or ID than blacks or
Hispanics. In particular, we estimate that the share of white license
or ID holders is 7.4\% greater than among black citizens and 10.6\%
greater than among Hispanic citizens. Furthermore, when calculating
standard confidence intervals for the difference of these quantities,
we find that the differences in license or ID holding between whites
and blacks as well as between whites and Hispanics do not overlap
zero.

Based on these results, we conclude that Texas's voter identification law will have a disparate impact on black and Hispanic citizens as compared to white citizens because it will effectively disenfranchise a greater proportion of these groups as compared with white citizens.

\subsection{Sources of Uncertainty in the Estimates}

There are several sources of uncertainty in ecological inference
estimates including estimation uncertainty, fundamental uncertainty,
and model dependence, conditional on the input data.  \emph{Estimation
  uncertainty} occurs because we have a limited set of observations
(counties) to use in estimating the disenfranchisement rates for each
racial group's votes.  \emph{Fundamental uncertainty} is the idea that
even with a large amount of data there is some degree of randomness in
the process of signing up to get a driver's license or ID.  A
citizen's license expires inadvertently, or they forget to sign up, or
their name is typed into the database incorrectly, or racial group
information is mis-estimated by the Census bureau; this is the stuff
of measurement and human behavior and is modeled in a well-known way
as fundamental variability. Finally, \emph{model dependence} is the
degree to which changes in modeling assumptions affect our estimates.
Models are required in ecological inference because some information
is destroyed in the process of aggregation and kept from the analyst
without having full information on every individual in the state.

In different types of statistical problems, each of these sources of
uncertainty can play different roles and to differing degrees.
Ecological inferences of course have all three sources of uncertainty,
but it is the model dependence caused by having to estimate voter ID
rates because all information is not in the public domain that
accounts for most of the overall uncertainty.  Model dependence
typically dwarfs the other two components. Thus, we tune our methods
of assessing uncertainty to focusing on model dependence.

To convey the degree of this uncertainty, we use ``tomography plots''.
These plots give all information in the data without making any
statistical modeling assumptions, as well as summarizing the available
statistical information in the data. This enables us to evaluate
directly how much information is available in the data and how much is
imposed by the statistical model.

One entire tomography plot is used for each numerical estimate in the
corresponding table. For one example, the tomography plot in Figure
\ref{tomog} analyzes the information and uncertainty in the data and
our analyses with respect to two unknowns: the percent of Whites
(horizontally) and non-Whites (vertically) who hold a driver's license
or ID in each county in Texas.  If these two quantities were known for
a county, they would appear in the plot as a single dot, and the set
of counties as a set of dots.  However, because individual level
census data is not available, and driver's license and census data are
not available a the individual level in the same database, we cannot
know any of the exact points with precision; what the plot shows is
that the information hidden from us by the problems with the database
is directly quantifiable: it turns each county's dot into a line.  We
can think of the uncertainty in ecological inference as the dot as
being smeared into a line. That smearing represents a loss of
information, but much information is retained (and that is uniquely
captured by our methods of ecological inference).  In particular, we
know for certain that the true point representing the percent of
whites and non-whites who hold a driver's license or ID falls
somewhere on the line; we do not know where on the basis of just the
information from that county, but we know it must be on the line.

\begin{figure}[htb]
\begin{centering}
\includegraphics[scale=.75]{figs/CVAP_f3_dl_white_ex.png}
\caption{\label{tomog}This tomography plot displays the white vs. non-white proportions of citizens (CVAP)
living in Texas who hold a driver's license.}
\end{centering}
\end{figure}

For example, consider the bold red line in the plot (which we
highlight for expository purposes).  We know, based on the observed
data from that county (the percent with a license or ID and the
percent White), that the point in this plot (representing where the
White and non-White share of license or ID holders is) for this county
must be some point on the line, but we do not know exactly where on
the line this point falls. For this line, we know that the fraction of
Whites that hold a valid driver's license or ID must fall somewhere
between 0.5 (50 percent) and 0.6 (60 percent). We get these numbers by
projecting the line downwards to the horizontal axis. If instead we
project the line to the left (vertical) axis, we can see that, for
this particular county, the range of possible values for the percent
of non-Whites holding a valid driver's license or ID could be anywhere
from 0\% to 100\%.  That estimate is better than the method of
ecological regression, which often gives answers outside that
interval, but still we can see that this county is informative with
respect only to Whites, not non-Whites.

In this way, each line on a tomography plot captures exactly what we
do and do not know about the proportion of people with driver's
licenses or IDs in each county --- without any uncertainty (assuming
only that the data from the State of Texas and the U.S.\ Census Bureau
are measured correctly). Our statistical method then uses all this
available information as well as the statistical information from
ecological regression. Lines that are relatively steep in this
particular tomography plot convey a lot of information about the
percent of Whites who hold a valid driver's license or ID.  Lines that
are relatively flat convey a great deal of information about the
percent of non-Whites that hold a valid driver's license or ID.  Lines
that cut off the top right or bottom left corner of the plot are
informative about both quantities.

The tomography plots also reflect what we know about the racial composition
of Texas. If a county contains more than 65\% of a
particular racial group (which we use as an arbitrary cutoff for
graphical clarity), then the line on the tomography plot that
corresponds to that precinct is color coded to represent the majority
group (see the legend in the plot). If no groups comprise 65\% or more
of the county, then the line is represented as black.

Even without making any (ecological inference) modeling assumptions,
we can see some aspects of the results and thus draw conclusions
directly from the data.  The relatively flat blue lines, with high
percentages of nonwhite populations cross the vertical axis above
about 0.5 (percent nonwhite IDs).  However, the steep red lines, with
high percentages of whites meet the horizontal above about 0.7.  This
difference is clear evidence that nonwhites in these areas have lower
rates of having valid drivers' licenses or IDs on average than
whites.

Finally, the tomography plots also reflect the results of the
ecological inference statistical estimation, that includes information
on patterns across counties. The point estimate (i.e., the exact point
on the line that we estimate as the proportion of each group with a
valid driver's license or ID) as well as the confidence intervals are
colored yellow.  Taken together, these describe the overall estimate
for the proportion of license or ID holders in each racial group in Texas as
a whole. In the tomography plot for Figure \ref{tomog}, the mass of
yellow on the right means that our estimates indicate that whites held
valid driver's licenses or IDs at a higher rate than did their non-white
counterparts.

As importantly, the tomography plots convey the \emph{overall
  uncertainty} in the available data, and how our statistical
estimator uses that information to produce an estimate.  In this
particular example, the prevalence of flat lines on the upper half of
the graph and steep lines near the right convey a great deal of
information about where and what we are uncertain about.  The
statistical estimator assumes that there is some relationship among
all the precincts in a district, and so in all likelihood there exists
a cluster of points on the plot somewhere such that when two lines
cross or approach one another, the true points on each are close to
each other on their respective lines.

To appropriately and completely judge all sources of uncertainty in
ecological inferences requires examining a plot like this for each
numerical quantity to be estimated.  The uncertainty estimates here
are far more informative and information rich than sampling based
confidence intervals or standard errors, which assume the model is
true and only summarize fundamental and estimation uncertainty.  Our
appendix includes these plots.

\subsection{Homogenous Counties}

As an intuitive way to check the logic of the results presented
without the use of statistical inference, we also compare the
proportion of citizens who hold driver's licenses or IDs in the most
homogenous white counties to the most homogenous Hispanic counties. We
restrict the sample to the top 10\% Texas counties in terms of white
share of the population and Hispanic share of the population. In the
most homogenous white counties (those that are 88.9\% white or more),
83.1\% of citizens hold a driver's license or ID; in the most
homogenous Hispanic counties (those that are 58.1\% Hispanic or more),
73.0\% of citizens hold a driver's license or ID.

\section{Concluding Remarks}

The evidence from this analysis clearly indicates that blacks and
Hispanics have drivers' licenses or IDs at lower rates than whites.
This is true regardless of whether we focus on nearly homogeneous
districts, the purely data oriented analysis of the tomography plot,
or the more encompassing ecological inference methodology used here.
\clearpage
\appendix
\renewcommand*\appendixpagename{\section*{Appendix}}
\appendixpage
\section{Tomography Plots}

\begin{figure}[htb]
\begin{centering}
\includegraphics[scale=.75]{figs/CVAP_f3_dl_white.png}
\caption{\label{tomog_white}This tomography plot displays the white vs. non-white proportions of citizens (CVAP)
living in Texas who hold a driver's license.}
\end{centering}
\end{figure}


\begin{figure}[htb]
\begin{centering}
\includegraphics[scale=.75]{figs/CVAP_f3_dl_black.png}
\caption{\label{tomog_black}This tomography plot displays the black vs. non-black proportions of citizens (CVAP)
living in Texas who hold a driver's license.}
\end{centering}
\end{figure}

\begin{figure}[htb]
\begin{centering}
\includegraphics[scale=.75]{figs/CVAP_f3_dl_hispanic.png}
\caption{\label{tomog_hispanic}This tomography plot displays the Hispanic vs. non-Hispanic proportions of citizens (CVAP)
living in Texas who hold a driver's license.}
\end{centering}
\end{figure}


\begin{figure}[htb]
\begin{centering}
\includegraphics[scale=.75]{figs/CVAP_f3_dl_other.png}
\caption{\label{tomog_other}This tomography plot displays the other vs. non-other proportions of citizens (CVAP)
living in Texas who hold a driver's license.}
\end{centering}
\end{figure}

\clearpage
\singlespace
\bibliographystyle{apsr} 
\bibsep=0in 
%\phantomsection
\addcontentsline{toc}{section}{References}
\bibliography{gk,gkpubs}

\end{document}
