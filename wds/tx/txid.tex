\documentclass[12pt]{article}
%\usepackage{Rd,Sweave,upquote}
\usepackage[reqno]{amsmath}
\usepackage{natbib,amssymb,amsthm,graphicx,verbatim,url,verbatim}
\usepackage[all]{xy}
\usepackage{vmargin}
\setpapersize{USletter}
\topmargin=0in
\usepackage{times}
\usepackage{dcolumn, booktabs}
\usepackage[dvipsnames,usenames]{color}
\usepackage{dcolumn}
\newcolumntype{.}{D{.}{.}{-1}}\newcolumntype{d}[1]{D{.}{.}{#1}}
%\usepackage[notref]{showkeys}
%\usepackage{endfloat}

\usepackage{color,setspace}
\definecolor{spot}{rgb}{0.6,0,0}
\usepackage[pdftex, bookmarksopen=true, bookmarksnumbered=true,
  pdfstartview=FitH, breaklinks=true, urlbordercolor={0 1 0},
  citebordercolor={0 0 1}, colorlinks=true, citecolor=spot, 
  linkcolor=spot, urlcolor=spot,
  pdfauthor={Gary King},
  pdftitle={}]{hyperref}

%squeeze
% == spacing between sections and subsections
%\usepackage[compact]{titlesec} 
% keep floats on same page
\renewcommand{\topfraction}{0.85}
\renewcommand{\textfraction}{0.1}
\renewcommand{\floatpagefraction}{0.75} % keep < \topfraction

\usepackage{soul}

\usepackage{appendix}
\usepackage{latexsym}
\newtheorem{claim}{Claim}
\newtheorem{proposition}{Proposition}
\newtheorem{theorem}{Theorem}
\newtheorem{lemma}{Lemma}
\newtheorem*{ass}{Assumption}
\newtheorem{question}{Question}
\newtheorem{remark}{Remark}
\newtheorem{definition}{Definition}
\newcommand{\bX}{\boldsymbol{X}}

\title{Estimates of Racial Differences in the Percent of Citizens to
  be Disenfranchised by the Texas Voter ID Law} \author{Gary
  King\thanks{Alfred J.  Weatherhead III University Professor, Harvard
    University, Institute for Quantitative Social Science, 1737
    Cambridge Street, Cambridge MA 02138; http://GKing.harvard.edu,
    king@harvard.edu, (617) 500-7570.}}

%\date{January 25, 2011}

\begin{document}
\maketitle
\tableofcontents
\clearpage
\doublespacing

\section{Introduction}

The purpose of this report is to estimate the differential percentages
of black, white, and Hispanic citizens of voting age who would be
disenfranchised (not allowed to vote) solely because they do not meet
the additional requirements of the new Texas Voter ID law.

The report from the Department of Justice’s expert Stephen
Ansolabehere examines the differential percentages of black, white,
and Hispanic \emph{registered voters} who would be disenfranchised by
the new law.  We supplement this information by attempting to estimate
the broader percentages disenfranchised of \emph{all} those in each
racial group who would have been eligible to vote without the new law,
including those who had and had not voted previously.  We also use
different assumptions and methods than Ansolabehere, the uncertainties
of which are likely uncorrelated with those from his report.  The
result is that the combination of the two reports should provide
stronger inferences than either alone.

\section{Background and Qualifications}

I am the Albert J Weatherhead III University Professor at Harvard
University, one of 24 with the designation of University Professor.  I
have taught at Harvard since 1987 and before that at New York
University.  I hold a Ph.D. from the University of Wisconsin.  I have
been elected a member of six honorary societies (National Academy of
Sciences 2010, American Statistical Association 2009, American
Association for the Advancement of Science 2004, American Academy of
Arts and Sciences 1998, Society for Political Methodology 2008, and
American Academy of Political and Social Science 2004), and have won
more than 30 ``best of'' awards for my articles, books, conference
papers, and software, among others.  I have published more than 130
journal articles, 15 open source software packages, and 8 books
spanning most aspects of political methodology, several fields of
political science, statistics, and other scholarly disciplines.  I
have served on more than 30 editorial boards.  The statistical methods
and software I developed are used extensively in litigation, academia,
government, consulting, and private industry.  More information on me
and copies of most of my work is available at
\url{http://gking.harvard.edu}.

I have been a expert witness or consultant (for governments and
partisans on both sides) in many U.S.\ states for legislative
redistricting and other matters.  Most recently, I was the expert for
the Arizona Independent Redistricting Commission.  I filed an Amici
Curiae brief in \emph{LULAC v. Perry} with three other scholars, which
was discussed and positively evaluated in three of the Supreme Court's
opinions, including the plurality decision.

\section{Methodological Approach}\label{s:methods}

In this section, we offer an overview of our methodological approach.
We will attempt to estimate the percentage of blacks, whites,
hispanics, and others who have valid voter identification, according
to the new Texas Voter ID law.  

From U.S.\ Census Data, we compute the percent of citizens' of voting
age who are black, white, hispanic, or other in each zip code area in
Texas.  Then from the State of Texas' driver's license database, we
compute the number of citizens of voting age (of any race) who have
valid identification, and from the combination of the two databases
the percent of citizens of voting age who have a valid ID.  

As there exists no data base to provide a complete cross-tabulation of
these results at the individual level, we use methods of ecological
inference --- which is the process of inferring from group level data
to information about individuals.  We summarize here the methods of
ecological inference we use.

In 1953, two methods of ecological inference were introduced --- the
method of bounds \citep{DunDav53} and ecological regression
\citep{Goodman53}. A special case of the method of bounds is known as
``homogeneous precinct analysis,'' which had been used in many court
cases: this approach seeks out ethnically homogeneous precincts (100\%
black or 100\% white, or 100\% Hispanic) because for those precincts
we know for certain the behavior of interest (in this case the percent
of those with a valid ID) of that one ethnic group.  The assumption
behind this method is that the percent with an ID observed in
homogeneous precincts is identical to that in other areas. The
advantage of this method is that it yields completely certain
information about some subgroups of citizens in some areas; the
disadvantage is that the relatively few who live in racially
homogeneous precincts may have starkly different proportions of IDs
than the vast majority of the population who live in (at least
partially) heterogeneous areas.  The method of bounds is more general
than homogeneous precinct analysis because the former can also provide
some information about heterogeneous precincts; it does this in the
form of ``bounds'' or ranges into which the fraction of a minority
group having IDs must fall.

The second method of ecological inference --- ecological regression
--- ignores the information revealed by the method of bounds and its
special case of homogeneous precinct analysis. Instead, ecological
regression uses statistical information across all precincts. For
example, if we find in areas with more African Americans that fewer
people have voter IDs, then we may be willing to infer that it is the
African Americans who have fewer IDs. The advantage of this approach
is that it uses some information from all precincts. The disadvantage
is that the information can be misleading: For example, also
consistent with the same evidence would be that the (perhaps poor)
whites who live in areas with high African Americans concentrations
are the ones with fewer IDs. In fact, as an indication of the common
problems with this method, ecological regression, unlike the method of
bounds, regularly gives impossible answers -- such as estimating the
percent of Hispanics voting for the Democrats as 120\% or $-54$\%.

The method of bounds (or homogeneous precinct analysis) and ecological
regression dominated the academic literature and courtroom expert
testimony from 1953 until 1997 when the approach known as EI was
introduced \citep{King97}. This approach was the first to combine the
deterministic information from the method of bounds with the
statistical information from ecological regression into a single set
of estimates. Thus, it uses the statistical information from all
precincts, the certain information from homogeneous precincts, and
other deterministic information known for certain from other precincts
(given in the form of ranges of estimates).  Impossible estimates are
never produced by this methodology, and all information from all
precincts are used in the analysis. Like any indirect method of
revealing information about individuals from aggregate data, EI is
also uncertain to a degree, but it uses more information than other
previous approaches.

Since EI was introduced, a variety of other methods have been
developed in the academic literature, virtually all of which follow
the same practice of including deterministic and statistical
information in the same model.  For example, \citet{RosJiaKin01}
extend King's method to large numbers of ethnic groups and candidates;
we use this method in our work as well as King's original method. Some
of the other methods of ecological inference have been collected in the
edited volume by \citet{KinRosTan04b}.

In practice, when we have data we can use to validate the methods,
ecological regression and homogeneous precinct analysis are often
inaccurate in many situations. Studies have shown that uncertainty
remains with King's and other subsequent methods --- which will always
be the case because some information is lost in the process of
aggregate --- but the estimates are usually superior. Differences
among the new methods that include both deterministic and statistical
information are, in comparison, relatively minor.

In this case, we use these newer methods to give estimates of the
proportion of each ethnic group that has a valid voter ID.  We also
use King's ``tomography plots'' and other diagnostic methods that help
us discern when adjustments in the methods need to be made, and how
much uncertainty remains.

The ecological inference methods in this case also have the advantage
of avoiding any errors that may exist in matching voter registration files

\section{Data}\label{s:data}

\singlespace
\bibliographystyle{apsr} 
\bibsep=0in 
%\phantomsection
\addcontentsline{toc}{section}{References}
\bibliography{gk,gkpubs}
\end{document}
