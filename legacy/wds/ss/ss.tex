\documentclass[11pt,titlepage]{article}
\usepackage[reqno]{amsmath}
\usepackage{amssymb}
\usepackage{epsf}
\usepackage{epsfig}
\usepackage{url}

% === additional commands/packages/settings ===
\usepackage{graphicx,psfrag,natbib}
%\usepackage{setspace}
%\usepackage{vmargin}
%\setpapersize{USletter}
\usepackage{endfloat,endnotes}

\setlength{\oddsidemargin}{-0.1in} \setlength{\evensidemargin}{-0.1in}
\setlength{\textwidth}{6.4in}
\normalsize
% \citationstyle{apsr}
% \makeatletter
 \def\baselinestretch{2}\large\normalsize


% === dcolumn package ===
\usepackage{dcolumn}
\newcolumntype{.}{D{.}{.}{-1}}
\newcolumntype{d}[1]{D{.}{.}{#1}}

% === newcommands from sei.tex ===
\newcommand{\EI}{\ensuremath{{\mathfrak EI}}}
\newcommand{\Tiny}{\tiny}
\newcommand{\sump}{\sum_{i=1}^p}
\newcommand{\mean}{\frac{1}{p}\sump}
\newcommand{\ub}{\Dot{\beta}}
\newcommand{\ut}{\Dot{\theta}}
\newcommand{\bbeta}{{\mathfrak B}}
\newcommand{\btheta}{{\mathfrak T}}
\newcommand{\blambda}{{\mathfrak L}}
\newcommand{\bbetau}{\breve{\mathfrak B}}
\newcommand{\bV}{{\cal V}}
\newcommand{\sigmau}{\breve{\sigma}}
\newcommand{\Sigmau}{\breve{\Sigma}}
\newcommand{\rhou}{\breve{\rho}}
\newcommand{\psiu}{\breve{\psi}}
\newcommand{\Eu}{\breve{\text{E}}}
\newcommand{\Vu}{\breve{\text{V}}}
\newcommand{\Bb}{B^b}
\newcommand{\Bw}{B^w}
\newcommand{\NbD}{{N_i^{bD}}}
\newcommand{\NwD}{{N_i^{wD}}}
\newcommand{\NbR}{{N_i^{bR}}}
\newcommand{\NwR}{{N_i^{wR}}}
\newcommand{\NbN}{{N_i^{bN}}}
\newcommand{\NwN}{{N_i^{wN}}}
\newcommand{\tp}{{\mbox{\Tiny{+}}}}
\newcommand{\NpD}{{N_i^D}}
\newcommand{\NpR}{{N_i^{R}}}
\newcommand{\NpN}{{N_i^{N}}}
\newcommand{\Nbp}{{N_i^{b}}}
\newcommand{\Nwp}{{N_i^{w}}}
\newcommand{\Npp}{{N_i}}
\newcommand{\Nppp}{{N}}
\newcommand{\Nbpp}{{N^{b}}}
\newcommand{\Nwpp}{{N^{w}}}
\newcommand{\sumpN}{\sump\Npp}
\newcommand{\NsumpN}{\frac{1}{\Nppp}\sumpN}
\newcommand{\nbp}{{N_i^{bT}}}
\newcommand{\nwp}{{N_i^{wT}}}
\newcommand{\npp}{{N_i^{T}}}
\newcommand{\NbV}{{N_i^{bT}}}
\newcommand{\NwV}{{N_i^{wT}}}
\newcommand{\NpV}{{N_i^{T}}}
\newcommand{\xb}{\bar{X}}
\newcommand{\wb}{\bar{W}}
\newcommand{\tb}{\bar{T}}
\newcommand{\cx}{{\mathbb X}}
\newcommand{\cw}{{\mathbb W}}
\newcommand{\ct}{{\mathbb T}}
\newcommand{\cpij}{{\mathbb P}^*_{ij}}
\newcommand{\cpi}{{\mathbb P}^*_i}
\newcommand{\cwd}{{\dot{\mathbb W}}}
\newcommand{\ctd}{{\dot{\mathbb T}}}
\newcommand{\cxd}{{\dot{\mathbb X}}}
\newcommand{\cxb}{\bar{\mathbb X}}
\newcommand{\cz}{{\mathbb G}}
\newcommand{\ch}{{\mathbb H}}
\newcommand{\cm}{{\mathbb M}}
\newcommand{\E}{{\textbf{E}}}
\newcommand{\V}{{\textbf{V}}}
\newcommand{\C}{{\textbf{C}}}
\newcommand{\rE}{{\text{E}}}
\newcommand{\rV}{{\text{V}}}
\newcommand{\rC}{{\text{C}}}
\newcommand{\TN}{\text{TN}}
\newcommand{\BN}{\text{BN}}
\newcommand{\N}{\text{N}}
\renewcommand{\P}{\text{P}}
\newcommand{\D}{\textbf{D}}
\newcommand{\R}{\ensuremath{\textbf{R}}}
\newcommand{\Rr}{\ensuremath{\{\textbf{R}\diagdown r}\}}
\newcommand{\bC}{\ensuremath{\textbf{C}}}
\newcommand{\Cc}{\ensuremath{\{\textbf{C}\diagdown c}\}}
\newcommand{\one}{{\mathbf{1}}}
\newcommand{\Q}{\ensuremath{\overset{\vspace{3em}}{\text{\Huge ?}}}}
\newlength{\padsp}
\settowidth{\padsp}{$(\beta^w_i=\nwp/\Nwp)$}
\newcommand{\padbw}{\hspace*\padsp}
\newcommand{\bkappa}{\boldsymbol{\kappa}}

\newcommand{\bb}{\beta^{\text{bad}}}
\newcommand{\bg}{\beta^{\text{good}}}
\renewcommand{\bibitem}{\vskip 2pt\par\hangindent\parindent\hskip-\parindent}

\title{Analyzing Second-Stage Ecological Regressions\thanks{Our thanks
    to Michael Herron and Kenneth Shotts for helpful comments on an
    earlier draft.}}

\author{Christopher Adolph\thanks{Department of Government, Harvard
    University. (Littauer Center, North Yard, Harvard University,
    Cambridge MA 02138; \texttt{http://chris.adolph.name},
    \texttt{cadolph@fas.harvard.edu}).}
\and %
Gary King\thanks{Department of Government, Harvard University and
  World Health Organization (Center for Basic Research in the Social
  Sciences, 34 Kirkland Street, Harvard University, Cambridge MA
  02138; \texttt{http://GKing.Harvard.Edu}, \texttt{King@Harvard.Edu},
  (617) 495-2027).}  }

\begin{document}
\maketitle
%\baselineskip=1.57\baselineskip

\section{Introduction}

We take this opportunity to comment on Herron and Shotts (2002;
hereinafter HS) because of its interesting and productive ideas and
because of the potential to affect the way a considerable body of
practical research is conducted.  This article, and the literature
referenced therein, is based on the suggestions in three paragraphs in
King (1997: 289--90).  Since these paragraphs were not summarized in
HS, we thought they might be a useful place to start:
\begin{quotation}
  If a second stage analysis is conducted, least squares regression
  should probably not be used in most cases, even though it may not be
  particularly misleading.  The best first approach is usually to
  display a scatterplot of the explanatory variable (or variables)
  horizontally and (say) an estimate of $\beta_i^b$ or $\beta_i^w$
  vertically.  In many cases, this plot will be sufficient evidence to
  complete the second stage analysis.
  
  If it proves useful to have more of a formal statistical approach,
  and many of the actual values of $\beta_i^b$ fall near zero or one,
  then some method should be used that takes this into account.  The
  data could be transformed, via a logit or probit transformation, or
  [the ``extended model''] could be applied\ldots.  Whatever method is
  chosen, the researcher should be careful to include the fact that
  some estimates of $\beta_i^b$ are more uncertain than others.
    
  In practice, a weighted least squares linear regression may be
  sufficient in many applications, with weights based on the standard
  error of $\beta_i^b$ (or other quantity of interest).  Researchers
  should be careful in applying this simplified method here, and
  should verify its assumptions with scatterplots. \ldots This is not
  as theoretically elegant a procedure as the more formal set up in
  [the ``extended model''] but it is simple, relatively robust, and
  probably complete enough to be of use in many applications.
\end{quotation}  

Clearly the only logically consistent model that has been offered for
the issue at hand is the EI extended model that allows the second
stage covariates to be included in the EI estimation procedure.  EI
software now includes a feature that allows for first differences to
be computed if those covariates are included, which should make the
extended model somewhat easier to understand and use.  (We discuss how
to use the extended model in Section \ref{s:exten}.)

Any second stage analysis is by definition a second best procedure,
when judged conditional on the model.  Still, second stage analyses
--- if they give approximately correct answers --- have the advantage
of being easier to understand, use, and evaluate (since an estimate of
the dependent variable in the second stage is observable), more
numerically stable (hence allowing more covariates to be included and
studied), computationally faster (allowing more analyses to be
examined), more amenable to diagnostics in verifying assumptions
(since the actual estimated data points can be observed and the fit
can be checked directly), and possibly more robust to certain types of
misspecification.  The question at hand is if and when such a
procedure can be used to produce answers that are approximately
correct.

HS's contribution is in pointing out that precinct-level estimates
from EI regress to the mean.  This ``shrinkage'' property is indeed a
characteristic of EI, and it is also a characteristic of every other
Bayesian model.  Shrinkage results in optimal estimates, that is, with
the smallest possible mean square error.  Thus, we agree with the
implication of HS's article that the best possible estimate of
$\beta_i^b$ (e.g., the fraction of blacks who vote), under HS's
assumptions, is that produced by EI.  But HS also make the interesting
and correct point that using Bayesian mean posterior estimates with
this property, like those given by EI, as dependent variables in least
squares regression can, under some circumstances, produce biased
estimates.  No one disputes that second stage regressions are
approximations or that they cause problems from some theoretical
perspectives; the question at hand is when they cause problems that
affect real applied research.

HS study second stage regression based on least squares and conclude
that their bias-adjusted slope is preferable.  The article does not
mention whether the constant term should be adjusted.  Their slope
adjustment is based on linear approximations within the classical
linear econometric theory framework.  The problems with HS's approach
that we discuss here concern assuming linearity when modeling an
inherently nonlinear and bounded relationship (i.e., their algebra by
definition miss the information in the bounds), omitting the constant
correction from the paper, never computing the correction they propose
when conducting simulations, and missing the fact that weighted least
squares corrects problems they raise.  We show how filling in this
missing information leads back to the suggestions from the paragraphs
quoted above.

In addition to making second-stage analyses possible by providing the
first precinct-level estimates from an ecological inference model, the
primary advance of EI was in resolving the half-century long debate
between supporters of Goodman's (1959) unbounded linear regression
approach and Duncan-Davis (1953) method of bounds by incorporating
all information from both into the same model.  HS fall back to
Goodman's unbounded approach and so miss the highly informative
deterministic information in the aggregate data.

We replicated HS's simulations without trouble.  We then examined how
well their adjusted regression procedure fit the observed data based
on the estimated $\hat\beta_i^b$ and the true (normally unobserved)
$\beta_i^b$.  We find that correcting the slope but not the constant
is considerably worse than an unadjusted regression of $\hat\beta_i^b$
on $Z_i$ and the other procedures discussed in HS's paper and the
literature we have examined.  This is true even if we knew the true
value of the adjustment.  Unless $Z$ has zero mean, the regression
line from this method tends to miss the cloud of true $\beta_i^b$
points by a wide margin.

We therefore begin by extending HS's slope adjustment procedure,
within their linear econometric framework, by developing an adjustment
for the constant term.  We follow the procedure that we believe Herron
and Shotts would have used if they had tried.\footnote{Indeed, after
  we finished a draft, Herron and Shotts told us that they had the
  same constant correction in a draft version of their paper but took
  it out in the final version to save space.  Since HS suggest
  correcting the slope, and do not mention constant term corrections,
  when we refer to the ``HS partially adjusted procedure,'' we are
  referring to what readers would conclude that Herron and Shotts
  advocate, even though we now know (from our subsequent
  correspondence with them) that they also believe the corrections
  discussed in their paper are incomplete for real applications.}  As
it turns out, this fully adjusted second stage regression method
dominates the slope-only procedure.  Indeed, the partially adjusted
procedure in HS is never called for.

Since the only estimators HS offer for their partial adjustment
procedure either assume knowledge of the truth being estimated or
assume that some unknown parameters can be estimated without error, we
develop an estimator without these flaws and apply it to create full
adjustments.  We find that this fully empirical version of the full
adjustment procedure is, with two partial exceptions, dominated by
(unadjusted) weighted least squares.  For one, when the bounds are
wide, the true adjustment factor could make a noticable difference,
but the adjustment itself cannot be reliably estimated and so the
procedure cannot be applied.  But in this case, of course, researchers
should not be running EI in the first place because of extreme model
dependence.  For the other, nonlinear relationships obviously cannot
be well modeled by any linear second stage regression, and so in that
case a logistic (or other nonlinear) regression procedure is best.
Thus, the only situation in which the adjustment would lead to
improvements is when the bounds are wide and the true value of the
adjustment is known, which describes the Monte Carlo procedures HS
used to evaluate their method, but of course not any real application.
(Even in this unrealistic situation, the full correction takes an hour
to run, compared to a few seconds for weighted least squares.)  Since
researchers can easily detect which situation applies from the
available aggregate data, they can always take appropriate action.

We conclude that researchers with narrow enough bounds to run first
stage EI should first examine a scatterplot of the covariate
horizontally by the \emph{bounds} on $\beta_i^b$ vertically (such as
King, 1997: Fig.13.2, p.238), since this scatterplot requires no
assumptions at all.  We can sometimes learn a lot from such a graph,
including especially a check for nonlinearities (which normally occurs
when the points are near zero or one).  If nonlinearities are not
apparent, then weighted least squares should be used since it offers
approximately unbiased estimates in second stage regressions.  We also
suggest a scatterplot of the covariate horizontally by the estimate
$\hat\beta_i^b$ vertically to examine the data being run.

\section{A Fully Adjusted Second Stage Regression Model}
\label{s:fulladj}

We try to stick to HS's notation where possible and present the fully
adjusted method with their adjustment as a special case.\footnote{Our
  notation deviates from HS only to ensure logical consistency.  For
  example, their Equations (7) and (8) have different dependent
  variables, $\beta_i^b$ and $\hat\beta_i^b$, set equal to the same
  entire right side of the equation in both cases
  ($\alpha+\gamma'Z_i+\nu_i$).  Later in their paper, they allow
  $\gamma$ to differ between the two (calling the first $\gamma_R$ and
  the second $\gamma_U$), but do not change the constant term (or
  error terms).  We fix these issues and others.}

First, let the expectation of $\beta_i^b$ conditional on $Z_i$ be
approximated by
\begin{equation}
  \label{true}
  E(\beta_i^b)=\alpha_R+\gamma_R Z_i,
\end{equation}
so that estimates of $\alpha_R$ and $\gamma_R$ are the immediate goal
of the analysis.  Also let the expectation of $\hat\beta_i^b$
conditional on $Z_i$ be
\begin{equation}
  \label{est}
  E(\hat\beta_i^b)=\alpha_U+\gamma_U Z_i,
\end{equation}
which is of course well estimated by least squares.  HS then assumes
that the error in estimating $\hat\beta_i^b$ by EI is a linear
function of $\beta_i^b$:
\begin{equation}
  \label{hserr}
  E(\hat\beta_i^b - \beta_i^b) = \delta_0^b + \delta_1^b\beta_i^b.
\end{equation}
Then solving (\ref{hserr}) for $E(\hat\beta_i^b)$ and substituting the
result into (\ref{est}) gives a more informative version of (\ref{true}):
\begin{equation}
  \label{adj}
  E(\beta_i^b) = \left(\frac{\alpha_U-\delta_0^b}{1+\delta_1^b}\right)
  + \left(\frac{\gamma_U}{1+\delta_1^b}\right)Z_i.
\end{equation}
Thus, we know that the quantities of interest, $\alpha_R$ and
$\gamma_R$, can be expressed as the intercept and slope of
(\ref{adj}).  If we can estimate the components of each, we can derive
a consistent estimator, at least when HS's linearity and unboundedness
assumptions are not too far off.

\section{An Improved Estimation Procedure}

HS offer an estimation algorithm for correcting the slope term in
their Section 7.2.  This procedure is intuitive and we can easily
generalize it to provide a correction for the intercept as well.
Unfortunately, the procedure itself is flawed for two other reasons.
First, it conditions on the point estimate for the parameters of the
truncated bivariate normal, $\breve\psi$, and of $\delta_0^b$ and
$\delta_1^b$, and thus assumes the absence of estimation uncertainty.
Ignoring uncertainty would bias standard errors and confidence
intervals, of course, which perhaps is why HS do not calculate these.
However, since their estimation procedure is nonlinear (due to the
ratios in (\ref{adj})), ignoring estimation uncertainty also affects
their point estimates in finite samples.  Second, the procedure calls
for drawing only a single simulation of $(\beta_i^b,\beta_i^w)$ for
each observation.  As a result, the estimate includes substantial
Monte Carlo approximation error.  The error can be eliminated by
running their entire procedure many times and averaging.  Although
fixing these problems would not have substantially changed the
estimates presented in their paper, they will matter in some
applications.  We therefore develop and use a new estimation algorithm
that corrects these problems, as well as providing the ability to
compute standard errors and confidence intervals, which were not
available in HS's version.\footnote{To define our revised estimation
  algorithm, let a symbol with a tilde denote a value of that quantity
  randomly drawn from its posterior density.  For a given $X$ and $T$:
  (1) run EI on $X$ and $T$; (2) regress $\hat\beta_i^b$ (which comes
  from EI) on $Z_i$,x yielding the estimated intercept $\hat\alpha_U$
  and slope $\hat\gamma_U$; (3) draw $\tilde\psiu$ from its posterior
  provided by EI; (4) take $p$ draws of
  $(\tilde\beta_i^b,\tilde\beta_i^w)$ from a truncated normal density
  with parameter vector $\tilde\psiu$, (5) compute a new $T_i$ as
  $\tilde T_i=\tilde\beta_i^bX_i+\tilde\beta_i^w(1-X_i)$; (6) run EI
  on $X_i$ and $\tilde T_i$ to yield estimates
  ($\hat{\tilde\beta}_i^b,\hat{\tilde\beta}_i^w$) for all $i$; (7)
  regress ($\hat{\tilde\beta}_i^b-\beta_i^b$) on $\beta_i^b$ to
  estimate $\delta_0^b$ and $\delta_1^b$, which we label
  $\hat{\tilde\delta}_0$ and $\hat{\tilde\delta}_1$; (8) draw
  $\tilde\alpha_U$ and $\tilde\gamma_U$ from the posterior provided by
  LS; (9) compute the adjusted intercept as
  $(\tilde\alpha_U-\hat{\tilde\delta}_0^b)/(1+\hat{\tilde\delta}_1^b)$
  and adjusted slope as $\tilde\gamma_U/(1+\hat{\tilde\delta}_1^b)$;
  (10) repeat steps (3)-(9) a sufficient number of times to eliminate
  Monte Carlo approximation error; (11) and average the simulations in
  (10) to get point estimates, take their standard deviation for
  standard errors, or sort them and use percentile values for
  confidence intervals.}

The accurate procedure that fully represents the uncertainty of HS's
corrections is computationally slow (taking about an hour to do one
run).  Also, the standard errors for the fully adjusted method appear
larger than those obtained from LS by about 70\% and WLS by about
30\%.  For the intercepts, which under full adjustment combine the
uncertainty of three parameters, the standard errors are on average 34
times larger than LS, and 25 times larger than WLS.  Thus, any
reduction in bias that may occur is probably outweighed by the
substantial increase in variance.  This is an intuitive result given
that the bias adjustment takes the form of a ratio, and both the
numerator (which is LS) and denominator contain estimation
uncertainty.  (We confirmed this result with a small number of Monte
Carlo experiments.)  It is also consistent with the experience of
others trying to create bias adjustments in a variety of models
outside the field of ecological inference.

\section{HS's Monte Carlo Simulation Procedure}

We explain in this section that HS's Monte Carlo procedure is
inappropriate for evaluating second stage regressions.  The procedure
is to draw the true value of the dependent variable, $\beta_i^b$, from
the EI model without covariates.  Then they create the covariate for
the second stage explanatory variable $Z_i$ endogenously as equal to
$\beta_i^b$ plus random noise.  This procedure has three flaws.

The first flaw is that the Monte Carlo procedure can be interpreted in
three logicially inconsistent ways and, although HS do not discuss
which interpretation was intended, none of the three make the
procedure valid.  The first, and in our view most plausible,
interpretation is that by creating $Z_i$ with random error, the
procedure induces immense errors-in-variables attenuation bias, quite
apart from any attenuation bias that may occur due to the Bayesian
shrinkage in the EI estimate.

This problem can be seen clearly by studying the parameters of the
model HS created from which to draw their Monte Carlo data.  In this
model, the slope of the coefficient on the covariate in the second
stage regression is 1 (in their notation,
$\gamma_R=1$).\footnote{Under this first interpretation of the HS
  Monte Carlo setup, $E(\beta_i^b)=\alpha_R+\gamma_R Z_i$, where
  $Z_i=\beta_i^b+\tau_i$ and $E(\tau_i)=0$ and so
  $E(Z_i|\beta_i^b)=\beta_i^b$.  Hence
  $E(\beta_i^b)=\alpha_R+\gamma_RE(\beta_i^b+\tau_i) =
  \alpha_R+\gamma_R\beta_i^b$, which implies that $\alpha_R=0$ and
  $\gamma_R=1$.}  However, the estimates of this slope from their
simulations of the \emph{true} $\beta_i^b$ (i.e., without any
attenuation bias in the dependent variable at all) regressed on $Z$
gives a drastically biased estimate.  This slope estimate does not
appear in their article, but we were able to to replicate their Table
2 exactly, and in our Table \ref{t:hsrep} present these numbers.  As
can be seen, whereas the theoretical value of $\gamma_R$ is 1
according to this interpretation, in each case their estimates
indicate that $\hat\gamma_R$ is never larger than 0.19 and on average
about 0.07.  Thus, since the unadjusted method is unable to recover
the coefficients without shrinkage when using the true value of the
dependent variable, this Monte Carlo setup is inappropriate for
assessing a dependent variable that is estimated (by EI or otherwise).
\begin{table}[tb]
\label{t:hsrep}
\begin{center}
\begin{tabular}{c|c}
Model Parameters & \multicolumn{1}{c}{HS Estimates} \\
$\breve\psi$  &  $\hat\gamma_R$ \\\hline
(0.5, 0.5, 0.1, 0.1, 0)        &       0.04\\  
(0.75, 0.5, 0.1, 0.1, 0)       &       0.04  \\
(0.75, 0.75, 0.1, 0.1, 0)      &       0.04  \\
(0.9, 0.9, 0.1, 0.1, 0)        &       0.02  \\
(0.5, 0.5, 0.32, 0.1, 0)       &       0.19  \\
(0.6, 0.6, 0.1, 0.32, 0)       &       0.04  \\
(0.9, 0.1, 0.32, 0.32, 0)      &       0.15  \\
(0.5, 0.5, 0.1, 0.1, 0.3)      &       0.04  \\
\hline
\end{tabular}
\end{center}
\caption{Comparing the true slope on $Z$ in a 
second-stage regression in the HS Monte Carlo 
procedure with the estimate based on $\beta_i^b$ 
(rather than $\hat\beta_i^b$) as the dependent variable.  The true
value of $\gamma_R$ depends on the interpretation used (described in
the text).  This table replicates the results of simulations presented in
HS's Table 2.  All results are averages over 100 simulations.  
The difference and ratios presented in HS's Table 2 were  
successfully replicated, and are not shown here.}
\end{table}

A second interpretation of the HS Monte Carlo procedure is that
$\gamma_R$ is the causal effect of $Z_i$ on $\beta_i^b$.  Since many
second stage regressions are designed to be causal, this is often the
most appropriate intepretation.  Since $Z_i$ was created endogenously,
no matter how one exogenously changes $Z_i$, $\beta_i^b$ will not
budge, and hence, by this interpretation, the Monte Carlo procedure is
setting the causal effect to zero: $\gamma_R=0$.  Thus, because the
numbers in Table \ref{t:hsrep} are uniformly greater than zero, we
know that regressing the true $\beta_i^b$ on $Z_i$ overestimates
$\gamma_R$.  Therefore, since the unadjusted method is unable to
recover the true coefficients when using the true value of the
dependent variable (i.e., even though no shrinkage occurs or
estimation error in the dependent variable exists), the Monte Carlo
setup under this second interpretation is also inappropriate for
assessing a dependent variable that is estimated (by EI or otherwise).
  
A third way to interpret the true value in HS's simulations is
conditional on \emph{each} randomly generated set of $\tau_i$'s
($i=1,\dots,n$), and hence conditional on each randomly generated
$Z_i$.  By this interpretation, each random draw of a set of $n$
observations from the Monte Carlo data generating process produces a
\emph{different} true value of $\gamma_R$, the value of which is
neither set nor observed by the researcher.  This value can be
estimated by a LS regression of the true $\beta_i^b$ on $Z_i$, and it
can also be estimated by EI-R.  Even if we assume that the LS
estimator is better (because it uses the true $\beta_i^b$), comparing
the two estimators does not reveal which is closer to the unknown
$\gamma_R$.  The Monte Carlo procedure can only reveal how close the
estimators are to each other, since the target is unknown and changing
over iterations and the estimators are correlated.  By this
interpretation, of course, the procedure would miss the whole point of
running Monte Carlo experiments in the first place --- creating a
world where we know the true quantity being estimated and then seeing
how good an estimator is at recovering the known parameter value.
  
Although by this last interpretation we cannot know the different true
$\gamma_R$ in each iteration, we can compute its expected value, i.e.,
the estimand implied by the LS estimator of $\beta_i^b$ on $Z_i$.
However, computing the average $\gamma_R$ requires nonlinear
approximations,\footnote{That is, under this third interpretation
  $\gamma_R=C(\beta_i^b,Z_i)/V(Z_i) =
  C(\beta_i^b,\beta_i^b+\tau_i)/[V(\beta_i^b)+V(\tau_i)] =
  [1+1/4V(\beta_i^b)]^{-1}$, since $V(\tau_i)$ was set at 1/4.  In
  this expression, $V(\beta_i^b)$, in turn, is a nonlinear function of
  the parameters of the truncated bivariate normal.} which in any
event were not computed in HS and so the result was not offered as a
standard of comparison.  Since the Monte Carlo design under this
interpretation bears little resemblence to how second stage
regressions are normally thought of, the standard would have little
relevance.  In any event, the advantage of proper Monte Carlo
experiments is that these after-the-fact guesses can be avoided from
the start.

Whichever interpretation one has of the HS Monte Carlo procedure, it
also suffers from two other serious problems.  First, it artificially
rules out the possibility of nonlinear relationships between
$\beta_i^b$ and $Z_i$ created by the bounds.  That is, by constructing
the explanatory variable, $Z_i$, from the bounded $\beta_i^b$'s (plus
a normal disturbance), the procedure artificially restricts the
relationship between the two to be linear and to never be affected by
the [0,1] bounds on $\beta_i^b$.  Within this framework, testing the
robustness of the HS adjustment to the sort of nonlinear relationships
which crop up sometimes in applied research is impossible.

Finally, \emph{HS's Monte Carlo simulations do not test the actual
  adjustment procedure they propose}.  All the results they present in
their Section 6 rely on the \emph{true} $\beta_i^b$'s to estimate
$\delta_0^b$ and $\delta_1^b$, even though this is the one quantity
users of second-stage regressions by definition lack.  Appropriately,
HS recommend a different estimation procedure (in their Section 7.2)
for use when $\beta_i^b$ is unknown, but leave this procedure
untested.  Hence, HS's article offers no direct evidence that
adjustment would be less biased or more efficient than unadjusted
least squares in practice.


\section{An Improved Monte Carlo Simulation} \label{s:alt}

Most of our conclusions below do not depend on changing the simulation
method, but we do so in order to make the results more coherent.  To
simulate, we follow the logic of the extended EI model.  Thus, we
first fix $X$, the covariates $Z$ (to values uncorrelated with $X$),
the values for the intercept and the slope parameter on $Z$, and the
variance and covariance parameters of the truncated bivariate normal.
Then, for each simulation, we draw the $\beta_i^b$'s from the extended
EI model conditional on $X_i$ and $Z_i$ (without mean centering).  The
assumption that $X_i$ and $Z_i$ are uncorrelated enables us to run a
(first-stage) basic EI model (i.e., with no covariates) without
inducing aggregation bias.  (The assumption is of course less
important when the bounds are more informative, but we retain it for
simplicity in the simulations below.)  With this setup, unless the
relationship is clearly nonlinear, a regression of the true
$\beta_i^b$ on $Z_i$ recovers the intercept and slope coefficients
accurately, effectively correcting the problem with the HS
procedure.

We use this Monte Carlo setup to illustrate four prototypical
situations that in our experience map out the space of applications in
which the various second-stage methods work in different ways, at
least when we follow the parameter values chosen in HS.  (That is, all
the simulations we have run look like these plots or, roughly
speaking, convex combinations of them.)

First, when ecological data have very wide bounds, EI (and any method
of ecological inference) will be sensitive to modeling assumptions.
In many applications with data like these, no ecological inference
should be conducted unless one has some special auxiliary information
about the model assumptions.  If one nevertheless proceeds to the
second stage, then, since shrinkage probably exists in the
$\beta_i^b$'s, the true (unobserved) value of the full adjustment
would make for an improvement over least squares using the estimated
$\hat\beta_i^b$'s.  Of course, the true adjustment is not known and
needs to be estimated.  Unfortunately, it cannot be estimated
reliably.  The reason is that EI models $\beta_i^b$ as a random effect
constrained to be within the precinct-level bounds. If $Z_i$ is not in
the EI first stage (which is true by definition, since if it were
included we wouldn't need a second stage), then \emph{the only
  information in $\hat\beta_i^b$ that could be predicted by $Z_i$
  comes from the bounds}.  The same is true of the adjustment
procedure: \emph{the only information with which to estimate
  $\delta_0^b$ and $\delta_1^b$ comes from the bounds}.  If the bounds
are relatively uninformative, as we are assuming in this first
prototypical case, then there is little information with which to
estimate the adjustment.  Of course, this should not be a surprise: an
unbounded random effect variable must be unrelated to all measured
variables except by chance.

Figure \ref{f:wide} plots the covariate $Z_i$ horizontally by the true
$\beta_i^b$ (in the left graph) and the estimated $\hat\beta_i^b$ (in
the right graph) vertically.  Note how the unadjusted least squares
line (marked LS) fits the estimated points well (in the right graph)
but is attenuated for the true points (in the left).  Since the
bounds are all very wide, the variances are almost constant and so the
weighted least squares (marked WLS) line is practically on top of the
LS line.  The line representing the fully adjusted method using the
\emph{true} values of $\delta_0^b$ and $\delta_1^b$ (true adjustment
is marked TA) fits the true points well, and would correct for the
attenuation.  The actual fully adjusted method (marked FA) as
estimated from the data also appears, but due to the wide bounds it is
not a good estimate and indeed is worse than LS and WLS.  Figure
\ref{f:wide} also plots the partially adjusted method that a
researcher might implement based on HS's article (marked PA), which is
more biased than any of the alternatives, a problem that grows in
severity as the mean of $Z$ departs from zero (The HS and FA lines are
not exactly parallel because HS is calculated by assuming the estimate
$\hat\psiu$ is known exactly, whereas FA uses our estimation
procedure.)  Since the partially adjusted method is never better than
full adjustment, and often dramatically worse, we do not consider it
further.
\begin{figure}[t]
  \begin{center}
    \epsfig{file=adolph.fig1.eps, width=5.5in}
    \caption{Data with Wide Bounds. Plot of $Z_i$ horizontally by
      the estimated $\hat\beta_i^b$ vertically (in the right graph)
      and the true $\beta_i^b$ (in the left graph), with fits for the
      partially adjusted procedure (PA) in HS, least squares (LS) and
      weighted least squares (WLS) almost on top of one another, the
      fully adjusted method (FA), and the full adjustment based on the
      true values of $\delta_0^b$ and $\delta_1^b$ (TA).  Clearly TA
      fits the true points best, but is unfortunately badly estimated
      by FA.  In this example, insufficient information exists in the
      bounds with which to make ecological inferences at all.  Data
      were generated from the extended EI model with $X \sim
      \textrm{Uniform}(0,0.2)$, $Z \sim \textrm{Normal}(0.5,0.01)$,
      $\breve\bbeta_i^b = Z_i - 0.1$, $\breve\bbeta_i^w = Z_i - 0.1$,
      $\sigmau_b = 0.05$, $\sigmau_w = 0.05$, and $\rhou = 0$.}
    \label{f:wide}
  \end{center}
\end{figure}

Second, when the bounds are at least somewhat informative (that is,
when few of the bounds are extremely wide), we are in the situation
where we would be more likely to trust ecological inferences using EI
(or another method that takes into account the information in the
precinct-level bounds).  When in addition the relationship is
approximately (or locally) linear, we find that least squares and
weighted least squares usually do as well as, and often better than,
the fully adjusted procedure.  Figure \ref{f:narrow} gives one example
where least squares, weighted least squares, and the fully adjusted
method all give approximately the same estimates.
\begin{figure}[t]
  \begin{center}
    \epsfig{file=adolph.fig2.eps, width=5.5in}
    \caption{Data with Informative Bounds. Plot of the $Z_i$ horizontally by
      the estimated $\hat\beta_i^b$ vertically (in the right graph)
      and the true $\beta_i^b$ (in the left graph) with fits for least
      squares (LS) and weighted least squares (WLS) appearing almost on
      top of one another, and the fully adjusted method (FA).  Note
      how all three methods give almost the same answer. Data were
      generated from the extended EI model, with $X \sim
      \textrm{Uniform}(0.2,1)$, $Z \sim \textrm{Normal}(0,0.01)$,
      $\breve\bbeta_i^b = Z_i + 0.44$, $\breve\bbeta_i^w = Z_i +
      0.68$, $\sigmau_b = 0.05$, $\sigmau_w = 0.05$, and $\rhou = 0$.}
    \label{f:narrow}
  \end{center}
\end{figure}

Third, when some observations have wide bounds and others have narrow
bounds, and $\hat\beta_i^b$ is an approximate (or locally) linear
function of $Z_i$, (unadjusted) weighted least squares regression will
often be substantially less biased than least squares, and
approximately equivalent to or better than the fully adjusted
procedure.\footnote{Like all weighted regressions, this procedure
  would have higher variance than LS. HS studied consistency, and
  implicitly bias, but did not address other properties, such as
  efficiency.}  This is contrary to
HS's claims that WLS would not make a difference; what they missed by
applying a linear regression framework to this problem with bounds and
nonlinearity is that the degree of attenuation is greater when the
bounds are wider --- as can be seen by the differences between Figures
\ref{f:wide} and \ref{f:narrow} --- and so the weights are correlated
with the attenuation bias and can at least partially correct for it.

Figure \ref{f:mixed} provides an example of this phenomenon.  We
generated the data for this figure from the same model as Figure
\ref{f:narrow}, changing only the parameter values so that the points
were affected by the bounds.  The effect of the wide bounds on some
observations can be seen by the attenuation in the set of points
forming a flatter slope in the right graph (as compared to the left
graph which has no such feature).  As a result, the least squares (LS)
line is a good deal flatter than it should be (as judged by the fit to
the points in the left graph) but the (unadjusted) weighted least
squares (WLS) line corrects for most of the attenuation.  The fully
adjusted (FA) line, in contrast, over-corrects for attenuation.
\begin{figure}[t]
  \begin{center}
    \epsfig{file=adolph.fig3.eps, width=5.5in}
    \caption{Data with Narrow and Wide Bounds. Plot of the $Z_i$ 
      horizontally by the estimated $\hat\beta_i^b$ vertically (in the
      right graph) and the true $\beta_i^b$ (in the left graph) with
      fits for least squares (LS), weighted least squares (WLS), and
      the fully adjusted method (FA).  Note how (unadjusted) WLS
      corrects for most of the attenuation bias.  Data were generated
      from the extended EI model, with $X \sim
      \frac{1}{2}\textrm{Uniform}(0,0.2) +
      \frac{1}{2}\textrm{Uniform}(0.8,1)$, $Z \sim
      \textrm{Normal}(0,0.01)$, $\breve\bbeta_i^b = Z_i + 0.44$,
      $\breve\bbeta_i^w = Z_i + 0.68$, $\sigmau_b = 0.05$, $\sigmau_w
      = 0.05$, and $\rhou = 0$.}
    \label{f:mixed}
  \end{center}
\end{figure}

Finally, when the relationship is nonlinear, as is often observably
the case because of the bounds, then any (adjusted or unadjusted)
linear second stage procedure can produce impossible results.  In this
situation, a scatterplot or an appropriate nonlinear procedure would
be better.  The fully adjusted procedure in this situation often
produces more out of bounds predictions than the unadjusted procedure.
(In this case, WLS is also inappropriate, both because the assumption
of linearity does not hold, and because $\hat\beta_i^b$'s at the
extremes have standard errors of zero or nearly so.  Giving extra
weight to these observations tends to bias the estimate of the slope
downwards.)  Figure \ref{f:nonlinear} illustrates these issues.  In
this example, we also include a nonlinear model by using a loess
regression of a logit transformation of $\hat\beta_i^b$,
$\ln(\hat\beta_i^b/(1-\hat\beta_i^b))$, on $Z_i$ and using simulation
to compute the regression line.  This line (marked loess) clearly
gives a far better fit than any of the other methods.  It is also the
only method that does not extend above 1 or below 0 for $\beta_i^b$
(i.e., into the impossible region) for some values of $Z_i$.  (A
linear regression on the logit scale would also stay out of the
impossible region but the fit would not be as much of an improvement.)
\begin{figure}[t]
  \begin{center}
    \epsfig{file=adolph.fig4.eps, width=5.5in}
    \caption{Data with a Nonlinear Relationship.  Plot of $Z_i$ 
      horizontally by the estimated $\hat\beta_i^b$ vertically (in the
      right graph) and the true $\beta_i^b$ (in the left graph) with
      fits for least squares (LS) and the fully adjusted method (FA)
      almost on top of one another, weighted least squares (WLS), and
      the better fitting loess regression on the logistic scale
      (loess).  Note how all the linear methods give out of bounds
      predictions.  Data were generated from the extended EI model,
      with $X \sim \textrm{Uniform}(0,1)$, $Z \sim
      \textrm{Normal}(0,4)$, $\breve\bbeta_i^b = Z_i + 1.16$,
      $\breve\bbeta_i^w = Z_i + 1.16$, $\sigmau_b = 0.3$, $\sigmau_w =
      0.3$, and $\rhou = 0$.}
    \label{f:nonlinear}
  \end{center}
\end{figure}

\section{The Extended Model} \label{s:exten}

As the only self-consistent ``second-stage'' approach, the extended
model should probably see more use.  We therefore pause briefly here
to discuss an important issue about how to use it.  

One apparently obvious, but flawed, way to use the extended model is
to study the effects of the explantory variables only by looking at
the truncated normal parameter estimates ($\alpha^b$ and $\alpha^w$ in
King, 1997: 170) and their standard errors.  The problem with this
approach is that it does not include the robustness of the bounds that
comes from conditioning on $T_i$.

To explain, consider a simpler case: estimating the district-wide
fraction of blacks who vote, $B^b$ without covariates.  If the model
holds (and the number of people per precinct is constant over
precincts), a consistent and efficient estimator of this quantity is
as follows.  (1) Run EI to estimate the five parameters of the
truncated bivariate normal.  Since they are on the untruncated scale,
(2) compute from them (analytically or by simulation) the truncated
parameters, which of course includes the mean of the precinct
parameters and hence our estimate.

Suppose, however, that the model is not exactly right.  Then we would
also want to condition on $T$ so that the precinct-level bounding
information can be included in this estimate.  To do this, use an
alternative estimator: (1) Run EI to estimate the five parameters of
the truncated bivariate normal.  Then (2) condition on $T$ and compute
estimates of the fraction of blacks who vote in each precinct,
$\beta_i^b$, by drawing (as in King, 1997) from the posterior density,
$\P(\beta_i^b|T_i)$, all the mass of which falls within the known
bounds.  Finally, (3) average the precinct estimates to produce an
estimate of $B^b$, as desired originally.

Because it includes the precinct-level bounds, the second estimator is
clearly more sensitive to misspecification than the first.  And since
dealing with misspecification is the key issue in making ecological
inferences in real research, we see little reason to use the first
estimator.  An equivalent problem applies in estimating and
interpreting $\alpha^b$ and $\alpha^w$ directly: the estimates do not
include information from the precinct-level bounds.  Although
conditional on the model, they are consistent and efficient, they will
be more sensitive to misspecification.  Thus, we suggest the same
approach to estimating effects from the extended model: (1) Estimate
the parameters of the truncated bivariate normal and $\alpha^b$ and
$\alpha^w$.  (2) Condition on $T_i$ and compute the conditional
densities, $\P(\beta_i^b|T_i)$.  (3) Either display all the densities
as a function of $Z$, such as via a scatterplot of simulations from
these densities vertically by $Z_i$ horizontally, or summarize them in
some way.  The result will be much less sensitive to misspecification.

\section{Concluding Suggestions}

As suggested in King (1997) and quoted above, ``The best first
approach is usually to display a scatterplot of the explanatory
variable (or variables) horizontally and (say) an estimate of
$\beta_i^b$ or $\beta_i^w$ vertically.  In many cases, this plot will
be sufficient evidence to complete the second stage analysis.''  This
approach remains accurate.  Indeed \emph{show us the data} is a good
general motto for any statistical analysis, especially those with
complex nonlinear and bounded variables such as those resulting from
ecological inference.  To this scatterplot, we would suggest adding
information on the bounds.  This can be done by adding a thin vertical
line representing the bounds on $\beta_i^b$ for each
$(\hat\beta_i^b,Z_i)$ point plotted (e.g., King, 1997: Figure 13.2).
From this figure, we can then see all the information in the data,
precisely how informative the bounds are, and whether the bounds are
of constant or variable width.

For researchers who wish a simple approximation to estimating a second
stage relationship instead of the extended model, the information
provided in this article provides a guide.  If enough information to
run a first stage EI model exists and a scatterplot does not indicate
nonlinearity, then weighted least squares is the best approach.
Researchers can easily tell which is appropriate by examining the
situations discussed in in Section \ref{s:alt}.

If a more formal statistical approach seems desirable, then a good
method must go beyond classical linear econometric theory.  It must
take into account (a) the nonlinear nature of the problem, (b) the
bounded nature of the second stage dependent variable with the width
of the bounds varying over observations, (c) the heteroskedasticity
and the correlation between bounds and the shrinkage, and (d) the
effect of any possible logical inconsistency of the first and second
stages of the analysis (since, of course, two-step statistical methods
need not be logically consistent to work well; see, e.g., Meng, 1994).
At present, the only model that has been proposed with all these
properties is the extended EI model that allows covariates to be
included as part of the EI estimation procedure (King, 1997: Ch.9).
HS are correct that this extended model is sometimes only weakly
identified, but that is only when the bounds are not narrow enough and
$X$ is included among the covariates or highly related to $Z$.  In
other cases, with narrow bounds or even wide ones when $Z$ is
unrelated to $X$, the extended model can be strongly identified and so
can be used in many cases without problem.  Imai and King (2002)
demonstrate how to compute first differences and other quantities of
interest from the extended EI model, and they report on extensions of
the EI software to make this possible.

Econometric theory and the classical linear regression framework works
well for what it was designed for.  However, in models with nonlinear
relationships or sample spaces or parameter spaces that are highly and
differentially bounded, such as in ecological inference problems,
political methodologists need to look elsewhere or develop their own
methods.  

HS have made an important contribution by highlighting what turns out
to be the shrinkage property of Bayesian point estimates like those
provided by EI.  We are in their debt for pointing this out and
stimulating the ideas and discussions offered herein.

\section*{References}
%\mbox{} \baselineskip=6pt 
%\parskip=1.5\baselineskip plus 4pt minus 4pt
%\vspace{-\parskip}

\bibitem Duncan, Otis Dudley. and Beverly Davis. 1953. ``An Alternative to
  Ecological Correlation,'' \emph{American Sociological Review}, 18:
  665--6.

\bibitem Goodman, Leo. 1959. ``Some Alternatives to Ecological
  Correlation,'' \emph{American Journal of Sociology}, 64: 610--24.
  
\bibitem Imai, Kosuke and Gary King. 2002. ``Did Illegally Counted
  Overseas Absentee Ballots Decide the 2000 U.S. Presidential
  Election?'' \url{http://gking.harvard.edu/preprints.shtml#ballots}.

\bibitem King, Gary. 1987. \emph{A Solution to the Ecological
    Inference Problem: Reconstructing Individual Behavior from
    Aggregate Data}, Princeton: Princeton University Press.
  
\bibitem Meng, X.L.\ 1994. ``Multiple-imputation Inferences with
  Uncongenial Sources of Input,'' \emph{Statistical Science}, 9, 4:
  538--573.
\end{document}
